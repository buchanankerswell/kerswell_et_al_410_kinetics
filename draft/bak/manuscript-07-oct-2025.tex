%%
% Copyright (c) 2017 - 2024, Pascal Wagler;
% Copyright (c) 2014 - 2024, John MacFarlane
%
% All rights reserved.
%
% Redistribution and use in source and binary forms, with or without
% modification, are permitted provided that the following conditions
% are met:
%
% - Redistributions of source code must retain the above copyright
% notice, this list of conditions and the following disclaimer.
%
% - Redistributions in binary form must reproduce the above copyright
% notice, this list of conditions and the following disclaimer in the
% documentation and/or other materials provided with the distribution.
%
% - Neither the name of John MacFarlane nor the names of other
% contributors may be used to endorse or promote products derived
% from this software without specific prior written permission.
%
% THIS SOFTWARE IS PROVIDED BY THE COPYRIGHT HOLDERS AND CONTRIBUTORS
% "AS IS" AND ANY EXPRESS OR IMPLIED WARRANTIES, INCLUDING, BUT NOT
% LIMITED TO, THE IMPLIED WARRANTIES OF MERCHANTABILITY AND FITNESS
% FOR A PARTICULAR PURPOSE ARE DISCLAIMED. IN NO EVENT SHALL THE
% COPYRIGHT OWNER OR CONTRIBUTORS BE LIABLE FOR ANY DIRECT, INDIRECT,
% INCIDENTAL, SPECIAL, EXEMPLARY, OR CONSEQUENTIAL DAMAGES (INCLUDING,
% BUT NOT LIMITED TO, PROCUREMENT OF SUBSTITUTE GOODS OR SERVICES;
% LOSS OF USE, DATA, OR PROFITS; OR BUSINESS INTERRUPTION) HOWEVER
% CAUSED AND ON ANY THEORY OF LIABILITY, WHETHER IN CONTRACT, STRICT
% LIABILITY, OR TORT (INCLUDING NEGLIGENCE OR OTHERWISE) ARISING IN
% ANY WAY OUT OF THE USE OF THIS SOFTWARE, EVEN IF ADVISED OF THE
% POSSIBILITY OF SUCH DAMAGE.
%%

%%
% This is the Eisvogel pandoc LaTeX template.
%
% For usage information and examples visit the official GitHub page:
% https://github.com/Wandmalfarbe/pandoc-latex-template
%%

% Options for packages loaded elsewhere
\PassOptionsToPackage{unicode}{hyperref}
\PassOptionsToPackage{hyphens}{url}
\PassOptionsToPackage{dvipsnames,svgnames,x11names,table}{xcolor}
%
\documentclass[
  paper=a4,
  ,captions=tableheading
]{scrartcl}
\usepackage{amsmath,amssymb}
% Use setspace anyway because we change the default line spacing.
% The spacing is changed early to affect the titlepage and the TOC.
\usepackage{setspace}
\setstretch{1.2}
\usepackage{iftex}
\ifPDFTeX
  \usepackage[T1]{fontenc}
  \usepackage[utf8]{inputenc}
  \usepackage{textcomp} % provide euro and other symbols
\else % if luatex or xetex
  \usepackage{unicode-math} % this also loads fontspec
  \defaultfontfeatures{Scale=MatchLowercase}
  \defaultfontfeatures[\rmfamily]{Ligatures=TeX,Scale=1}
\fi
\usepackage{lmodern}
\ifPDFTeX\else
  % xetex/luatex font selection
\fi
% Use upquote if available, for straight quotes in verbatim environments
\IfFileExists{upquote.sty}{\usepackage{upquote}}{}
\IfFileExists{microtype.sty}{% use microtype if available
  \usepackage[]{microtype}
  \UseMicrotypeSet[protrusion]{basicmath} % disable protrusion for tt fonts
}{}
\makeatletter
\@ifundefined{KOMAClassName}{% if non-KOMA class
  \IfFileExists{parskip.sty}{%
    \usepackage{parskip}
  }{% else
    \setlength{\parindent}{0pt}
    \setlength{\parskip}{6pt plus 2pt minus 1pt}}
}{% if KOMA class
  \KOMAoptions{parskip=half}}
\makeatother
\usepackage{xcolor}
\definecolor{default-linkcolor}{HTML}{A50000}
\definecolor{default-filecolor}{HTML}{A50000}
\definecolor{default-citecolor}{HTML}{4077C0}
\definecolor{default-urlcolor}{HTML}{4077C0}
\usepackage[margin=2.5cm,includehead=true,includefoot=true,centering,]{geometry}
\usepackage{listings}
\newcommand{\passthrough}[1]{#1}
\renewcommand\lstlistingname{Code Block}
\renewcommand\lstlistlistingname{Code Block}
\lstset{defaultdialect=[5.3]Lua}
\lstset{defaultdialect=[x86masm]Assembler}
\usepackage{longtable,booktabs,array}
\usepackage{calc} % for calculating minipage widths
% Correct order of tables after \paragraph or \subparagraph
\usepackage{etoolbox}
\makeatletter
\patchcmd\longtable{\par}{\if@noskipsec\mbox{}\fi\par}{}{}
\makeatother
% Allow footnotes in longtable head/foot
\IfFileExists{footnotehyper.sty}{\usepackage{footnotehyper}}{\usepackage{footnote}}
\makesavenoteenv{longtable}
% add backlinks to footnote references, cf. https://tex.stackexchange.com/questions/302266/make-footnote-clickable-both-ways
\usepackage{footnotebackref}
\usepackage{graphicx}
\makeatletter
\newsavebox\pandoc@box
\newcommand*\pandocbounded[1]{% scales image to fit in text height/width
  \sbox\pandoc@box{#1}%
  \Gscale@div\@tempa{\textheight}{\dimexpr\ht\pandoc@box+\dp\pandoc@box\relax}%
  \Gscale@div\@tempb{\linewidth}{\wd\pandoc@box}%
  \ifdim\@tempb\p@<\@tempa\p@\let\@tempa\@tempb\fi% select the smaller of both
  \ifdim\@tempa\p@<\p@\scalebox{\@tempa}{\usebox\pandoc@box}%
  \else\usebox{\pandoc@box}%
  \fi%
}
% Set default figure placement to htbp
% Make use of float-package and set default placement for figures to H.
% The option H means 'PUT IT HERE' (as  opposed to the standard h option which means 'You may put it here if you like').
\usepackage{float}
\floatplacement{figure}{H}
\makeatother
\setlength{\emergencystretch}{3em} % prevent overfull lines
\providecommand{\tightlist}{%
  \setlength{\itemsep}{0pt}\setlength{\parskip}{0pt}}
\setcounter{secnumdepth}{5}
% definitions for citeproc citations
\NewDocumentCommand\citeproctext{}{}
\NewDocumentCommand\citeproc{mm}{%
  \begingroup\def\citeproctext{#2}\cite{#1}\endgroup}
\makeatletter
 % allow citations to break across lines
 \let\@cite@ofmt\@firstofone
 % avoid brackets around text for \cite:
 \def\@biblabel#1{}
 \def\@cite#1#2{{#1\if@tempswa , #2\fi}}
\makeatother
\newlength{\cslhangindent}
\setlength{\cslhangindent}{1.5em}
\newlength{\csllabelwidth}
\setlength{\csllabelwidth}{3em}
\newenvironment{CSLReferences}[2] % #1 hanging-indent, #2 entry-spacing
  {\begin{list}{}{%
   \setlength{\itemindent}{0pt}
   \setlength{\leftmargin}{0pt}
   \setlength{\parsep}{0pt}
   % turn on hanging indent if param 1 is 1
   \ifodd #1
    \setlength{\leftmargin}{\cslhangindent}
    \setlength{\itemindent}{-1\cslhangindent}
   \fi
   % set entry spacing
   \setlength{\itemsep}{#2\baselineskip}}}
  {\end{list}}
\usepackage{calc}
\newcommand{\CSLBlock}[1]{\hfill\break\parbox[t]{\linewidth}{\strut\ignorespaces#1\strut}}
\newcommand{\CSLLeftMargin}[1]{\parbox[t]{\csllabelwidth}{\strut#1\strut}}
\newcommand{\CSLRightInline}[1]{\parbox[t]{\linewidth - \csllabelwidth}{\strut#1\strut}}
\newcommand{\CSLIndent}[1]{\hspace{\cslhangindent}#1}
\ifLuaTeX
\usepackage[bidi=basic]{babel}
\else
\usepackage[bidi=default]{babel}
\fi
\babelprovide[main,import]{american}
% get rid of language-specific shorthands (see #6817):
\let\LanguageShortHands\languageshorthands
\def\languageshorthands#1{}
\makeatletter
\@ifpackageloaded{subcaption}{}{\usepackage{subcaption}}
\@ifpackageloaded{caption}{}{\usepackage{caption}}
\captionsetup[subfigure]{margin=0.5em}
\AtBeginDocument{%
\renewcommand*\figurename{Figure}
\renewcommand*\tablename{Table}
}
\AtBeginDocument{%
\renewcommand*\listfigurename{List of Figures}
\renewcommand*\listtablename{List of Tables}
}
\newcounter{pandoccrossref@subfigures@footnote@counter}
\newenvironment{pandoccrossrefsubfigures}{%
\setcounter{pandoccrossref@subfigures@footnote@counter}{0}
\begin{figure}\centering%
\gdef\global@pandoccrossref@subfigures@footnotes{}%
\DeclareRobustCommand{\footnote}[1]{\footnotemark%
\stepcounter{pandoccrossref@subfigures@footnote@counter}%
\ifx\global@pandoccrossref@subfigures@footnotes\empty%
\gdef\global@pandoccrossref@subfigures@footnotes{{##1}}%
\else%
\g@addto@macro\global@pandoccrossref@subfigures@footnotes{, {##1}}%
\fi}}%
{\end{figure}%
\addtocounter{footnote}{-\value{pandoccrossref@subfigures@footnote@counter}}
\@for\f:=\global@pandoccrossref@subfigures@footnotes\do{\stepcounter{footnote}\footnotetext{\f}}%
\gdef\global@pandoccrossref@subfigures@footnotes{}}
\newcommand*\listoflistings\lstlistoflistings
\AtBeginDocument{%
\renewcommand*{\lstlistlistingname}{List of Listings}
}
\makeatother
\usepackage{bookmark}
\IfFileExists{xurl.sty}{\usepackage{xurl}}{} % add URL line breaks if available
\urlstyle{same}
\hypersetup{
  pdftitle={Displaced and Faded},
  pdfauthor={Kerswell B.; Wheeler J.; Gassmöller R.},
  pdflang={en-US},
  pdfsubject={Mantle Convection},
  pdfkeywords={mantle convection, phase changes, geodynamics, numerical
modeling},
  colorlinks=true,
  linkcolor={default-linkcolor},
  filecolor={default-filecolor},
  citecolor={default-citecolor},
  urlcolor={default-urlcolor},
  breaklinks=true,
  pdfcreator={LaTeX via pandoc with the Eisvogel template}}
\title{Displaced and Faded}
\usepackage{etoolbox}
\makeatletter
\providecommand{\subtitle}[1]{% add subtitle to \maketitle
  \apptocmd{\@title}{\par {\large #1 \par}}{}{}
}
\makeatother
\subtitle{How thermodynamics and kinetics collude to complicate seismic
structures in Earth's mantle}
\author{Kerswell B. \and Wheeler J. \and Gassmöller R.}
\date{07 October 2025}



%%
%% added
%%


%
% for the background color of the title page
%
\usepackage{pagecolor}
\usepackage{afterpage}
\usepackage[margin=2.5cm,includehead=true,includefoot=true,centering]{geometry}

%
% break urls
%
\PassOptionsToPackage{hyphens}{url}

%
% When using babel or polyglossia with biblatex, loading csquotes is recommended
% to ensure that quoted texts are typeset according to the rules of your main language.
%
\usepackage{csquotes}

%
% captions
%
\definecolor{caption-color}{HTML}{777777}
\usepackage[font={stretch=1.2}, textfont={color=caption-color}, position=top, skip=4mm, labelfont=bf, singlelinecheck=false, justification=justified]{caption}
\setcapindent{0em}

%
% blockquote
%
\definecolor{blockquote-border}{RGB}{221,221,221}
\definecolor{blockquote-text}{RGB}{119,119,119}
\usepackage{mdframed}
\newmdenv[rightline=false,bottomline=false,topline=false,linewidth=3pt,linecolor=blockquote-border,skipabove=\parskip]{customblockquote}
\renewenvironment{quote}{\begin{customblockquote}\list{}{\rightmargin=0em\leftmargin=0em}%
\item\relax\color{blockquote-text}\ignorespaces}{\unskip\unskip\endlist\end{customblockquote}}

%
% Source Sans Pro as the default font family
% Source Code Pro for monospace text
%
% 'default' option sets the default
% font family to Source Sans Pro, not \sfdefault.
%
\ifnum 0\ifxetex 1\fi\ifluatex 1\fi=0 % if pdftex
    \usepackage[default]{sourcesanspro}
  \usepackage{sourcecodepro}
  \else % if not pdftex
    \usepackage[default]{sourcesanspro}
  \usepackage{sourcecodepro}

  % XeLaTeX specific adjustments for straight quotes: https://tex.stackexchange.com/a/354887
  % This issue is already fixed (see https://github.com/silkeh/latex-sourcecodepro/pull/5) but the
  % fix is still unreleased.
  % TODO: Remove this workaround when the new version of sourcecodepro is released on CTAN.
  \ifxetex
    \makeatletter
    \defaultfontfeatures[\ttfamily]
      { Numbers   = \sourcecodepro@figurestyle,
        Scale     = \SourceCodePro@scale,
        Extension = .otf }
    \setmonofont
      [ UprightFont    = *-\sourcecodepro@regstyle,
        ItalicFont     = *-\sourcecodepro@regstyle It,
        BoldFont       = *-\sourcecodepro@boldstyle,
        BoldItalicFont = *-\sourcecodepro@boldstyle It ]
      {SourceCodePro}
    \makeatother
  \fi
  \fi

%
% heading color
%
\definecolor{heading-color}{RGB}{40,40,40}
\addtokomafont{section}{\color{heading-color}}
% When using the classes report, scrreprt, book,
% scrbook or memoir, uncomment the following line.
%\addtokomafont{chapter}{\color{heading-color}}

%
% variables for title, author and date
%
\usepackage{titling}
\title{Displaced and Faded}
\author{Kerswell B., Wheeler J., Gassmöller R.}
\date{07 October 2025}

%
% tables
%

\definecolor{table-row-color}{HTML}{F5F5F5}
\definecolor{table-rule-color}{HTML}{999999}

%\arrayrulecolor{black!40}
\arrayrulecolor{table-rule-color}     % color of \toprule, \midrule, \bottomrule
\setlength\heavyrulewidth{0.3ex}      % thickness of \toprule, \bottomrule
\renewcommand{\arraystretch}{1.3}     % spacing (padding)


%
% remove paragraph indentation
%
\setlength{\parindent}{0pt}
\setlength{\parskip}{6pt plus 2pt minus 1pt}
\setlength{\emergencystretch}{3em}  % prevent overfull lines

%
%
% Listings
%
%


%
% general listing colors
%
\definecolor{listing-background}{HTML}{E5E5E5}
\definecolor{listing-rule}{HTML}{B3B2B3}
\definecolor{listing-numbers}{HTML}{B3B2B3}
\definecolor{listing-text-color}{HTML}{000000}
\definecolor{listing-keyword}{HTML}{435489}
\definecolor{listing-keyword-2}{HTML}{994d00}
\definecolor{listing-keyword-3}{HTML}{9137CB}
\definecolor{listing-identifier}{HTML}{5F2E3A}
\definecolor{listing-symbol}{HTML}{990000}
\definecolor{listing-string}{HTML}{006B6B}
\definecolor{listing-comment}{HTML}{3A5F2E}

\lstdefinestyle{eisvogel_listing_style}{
  numbers          = left,
  xleftmargin      = 2.7em,
  framexleftmargin = 2.5em,
  backgroundcolor  = \color{listing-background},
  basicstyle       = \color{listing-text-color}\linespread{1.0}%
                      \lst@ifdisplaystyle%
                      \small%
                      \fi\ttfamily{},
  breaklines       = true,
  frame            = single,
  framesep         = 0.19em,
  rulecolor        = \color{listing-rule},
  frameround       = ffff,
  tabsize          = 4,
  aboveskip        = 1.0em,
  belowskip        = 0.1em,
  abovecaptionskip = 0em,
  belowcaptionskip = 1.0em,
  sensitive        = true,
  showstringspaces = false,
  escapeinside     = {/*@}{@*/}, % Allow LaTeX inside these special comments
  numberstyle      = \color{listing-numbers}
}

\lstdefinelanguage{bash}{
  morekeywords    = {if, fi, then, else, elif, for, while, do, done, case, esac,
                     function, select, set, until, readonly, declare, let, eval, exit,
                     break, continue, shift, return, subsection, end},
  sensitive       = true,
  morecomment     = [l]{\#},
  morestring      = [b]",
  morestring      = [b]',
  stringstyle     = \color{listing-string},
  commentstyle    = \color{listing-comment},
  keywordstyle    = \color{listing-keyword}\bfseries,
  keywordstyle    = {[2]\color{listing-keyword-2}\bfseries},
  literate        =
    % symbols
    {,}{{\textcolor{listing-symbol}{,}}}1
    {;}{{\textcolor{listing-symbol}{;}}}1
    {!}{{\textcolor{listing-symbol}{!}}}1
    {\\}{{\textcolor{listing-symbol}{\textbackslash}}}1
    {=}{{\textcolor{listing-symbol}{=}}}1
    {~}{{\textcolor{listing-symbol}{\textasciitilde}}}1
    {=~}{{\textcolor{listing-symbol}{=\textasciitilde}}}2
    {[}{{\textcolor{listing-symbol}{[}}}1
    {]}{{\textcolor{listing-symbol}{]}}}1
    {\{}{{\textcolor{listing-symbol}{\{}}}1
    {\}}{{\textcolor{listing-symbol}{\}}}}1
    {|}{{\textcolor{listing-symbol}{|}}}1
    {<}{{\textcolor{listing-symbol}{<}}}1
    {>}{{\textcolor{listing-symbol}{>}}}1
    {*}{{\textcolor{listing-symbol}{*}}}1
    % keywords
    {\&>}{{\textcolor{listing-keyword-2}{\&>}}}2
    {/dev/null}{{\textcolor{listing-keyword-2}{/dev/null}}}9
    {\&}{{\textcolor{listing-keyword-2}{\&}}}1
    {--}{{\textcolor{listing-keyword-2}{--}}}2
    {~}{{\textcolor{listing-keyword-2}{~}}}1
    {-}{{\textcolor{listing-keyword-2}{-}}}1
    {-e}{{\textcolor{listing-keyword-2}{-e}}}2
    {-j}{{\textcolor{listing-keyword-2}{-j}}}2
    {-d}{{\textcolor{listing-keyword-2}{-d}}}2
    {-D}{{\textcolor{listing-keyword-2}{-D}}}2
    {-x}{{\textcolor{listing-keyword-2}{-x}}}2
    {-f}{{\textcolor{listing-keyword-2}{-f}}}2
    {-q}{{\textcolor{listing-keyword-2}{-q}}}2
    {-y}{{\textcolor{listing-keyword-2}{-y}}}2
    {-r}{{\textcolor{listing-keyword-2}{-r}}}2
    {-v}{{\textcolor{listing-keyword-2}{-v}}}2
    {-rf}{{\textcolor{listing-keyword-2}{-rf}}}3
    {-type}{{\textcolor{listing-keyword-2}{-type}}}5
    {-name}{{\textcolor{listing-keyword-2}{-name}}}5
    {á}{{\'a}}1 {é}{{\'e}}1 {í}{{\'i}}1 {ó}{{\'o}}1 {ú}{{\'u}}1
    {Á}{{\'A}}1 {É}{{\'E}}1 {Í}{{\'I}}1 {Ó}{{\'O}}1 {Ú}{{\'U}}1
    {à}{{\`a}}1 {è}{{\`e}}1 {ì}{{\`i}}1 {ò}{{\`o}}1 {ù}{{\`u}}1
    {À}{{\`A}}1 {È}{{\`E}}1 {Ì}{{\`I}}1 {Ò}{{\`O}}1 {Ù}{{\`U}}1
    {ä}{{\"a}}1 {ë}{{\"e}}1 {ï}{{\"i}}1 {ö}{{\"o}}1 {ü}{{\"u}}1
    {Ä}{{\"A}}1 {Ë}{{\"E}}1 {Ï}{{\"I}}1 {Ö}{{\"O}}1 {Ü}{{\"U}}1
    {â}{{\^a}}1 {ê}{{\^e}}1 {î}{{\^i}}1 {ô}{{\^o}}1 {û}{{\^u}}1
    {Â}{{\^A}}1 {Ê}{{\^E}}1 {Î}{{\^I}}1 {Ô}{{\^O}}1 {Û}{{\^U}}1
    {œ}{{\oe}}1 {Œ}{{\OE}}1 {æ}{{\ae}}1 {Æ}{{\AE}}1 {ß}{{\ss}}1
    {ç}{{\c c}}1 {Ç}{{\c C}}1 {ø}{{\o}}1 {å}{{\r a}}1 {Å}{{\r A}}1
    {€}{{\EUR}}1 {£}{{\pounds}}1 {«}{{\guillemotleft}}1
    {»}{{\guillemotright}}1 {ñ}{{\~n}}1 {Ñ}{{\~N}}1 {¿}{{?`}}1
    {…}{{\ldots}}1 {≥}{{>=}}1 {≤}{{<=}}1 {„}{{\glqq}}1 {“}{{\grqq}}1
    {”}{{''}}1
}

\lstset{style=eisvogel_listing_style}


%
% header and footer
%
\usepackage[headsepline,footsepline]{scrlayer-scrpage}

\newpairofpagestyles{eisvogel-header-footer}{
  \clearpairofpagestyles
  \ihead*{Displaced and Faded}
  \chead*{}
  \ohead*{07 October 2025}
  \ifoot*{Kerswell B., Wheeler J., Gassmöller R.}
  \cfoot*{}
  \ofoot*{\thepage}
  \addtokomafont{pageheadfoot}{\upshape}
}
\pagestyle{eisvogel-header-footer}



%%
%% end added
%%

\begin{document}

%%
%% begin titlepage
%%
\begin{titlepage}
\newgeometry{left=6cm}
\definecolor{titlepage-color}{HTML}{2E7A40}
\newpagecolor{titlepage-color}\afterpage{\restorepagecolor}
\newcommand{\colorRule}[3][black]{\textcolor[HTML]{#1}{\rule{#2}{#3}}}
\begin{flushleft}
\noindent
\\[-1em]
\color[HTML]{FFFFFF}
\makebox[0pt][l]{\colorRule[FFFFFF]{1.3\textwidth}{2pt}}
\par
\noindent

{
  \setstretch{1.4}
  \vfill
  \noindent {\huge \textbf{\textsf{Displaced and Faded}}}
    \vskip 1em
  {\Large \textsf{How thermodynamics and kinetics collude to complicate
seismic structures in Earth's mantle}}
    \vskip 2em
  \noindent {\Large \textsf{Kerswell B., Wheeler J., Gassmöller R.}}
  \vfill
}


\textsf{07 October 2025}
\end{flushleft}
\end{titlepage}
\restoregeometry
\pagenumbering{arabic}

%%
%% end titlepage
%%

% \maketitle


{
\hypersetup{linkcolor=}
\setcounter{tocdepth}{3}
\tableofcontents
\newpage
}
\section*{Abstract}\label{sec:abstract}
\addcontentsline{toc}{section}{Abstract}

The seismic expression of Earth's 410 km discontinuity varies from
sharp, high-amplitude interfaces to broad weak signals---patterns not
fully explained by equilibrium thermodynamics. Laboratory studies show
that the olivine \(\Leftrightarrow\) wadsleyite transition is strongly
rate-limited, implying that kinetics may strongly influence seismic
structures, but uncertainties in growth and nucleation span several
orders of magnitude. To quantify the effects of kinetics on seismic
structures, we embed a growth-controlled kinetic model into compressible
geodynamic simulations of mantle plumes and slabs. We ask how
thermodynamic, kinetic, and dynamic feedbacks shape the 410, what
timescales and lengthscales characterize these processes, and how
seismic observables can be used to constrain kinetic parameters. Our
results show that plumes follow smooth, monotonic trends: sluggish
kinetics broaden and uplift the 410, whereas faster kinetics sharpen it
to seismically resolvable widths. Slabs instead display thresholded
behavior: moderately-slow kinetics generate metastable wedges, broad
phase transition zones, and stagnation at the 410, while both ultra-slow
(\(\dot{X}\) \textless{} 0.02 Ma\(^{-1}\)) and ultra-fast (\(\dot{X}\)
\textgreater{} 1 Ma\(^{-1}\)) kinetics promote high-contrast seismic
discontinuities. We identify kinetic thresholds of approximately
\(\dot{X}\) \textless{} 1 Ma\(^{-1}\) that separate metastable, slab
ponding regimes from equilibrated, slab penetrating ones, providing a
direct link between kinetic rates and seismic signatures in subduction
zones. These results reconcile observed lateral variability in 410
topography and sharpness across hotspots and subduction zones,
establishing the 410 as a sensitive seismological probe of phase
transition kinetics. More broadly, they demonstrate that incorporating
kinetics into compressible geodynamic simulations is essential for
connecting mineral-scale processes to mantle-scale dynamics, and for
constraining phase transitions kinetics from seismic data.

\cleardoublepage

\section*{Definition of Symbols}\label{sec:symbols}
\addcontentsline{toc}{section}{Definition of Symbols}

{\def\LTcaptype{} % do not increment counter
\begin{longtable}[]{@{}
  >{\raggedright\arraybackslash}p{(\linewidth - 6\tabcolsep) * \real{0.4688}}
  >{\raggedright\arraybackslash}p{(\linewidth - 6\tabcolsep) * \real{0.2188}}
  >{\raggedright\arraybackslash}p{(\linewidth - 6\tabcolsep) * \real{0.1562}}
  >{\raggedright\arraybackslash}p{(\linewidth - 6\tabcolsep) * \real{0.1562}}@{}}
\toprule\noalign{}
\begin{minipage}[b]{\linewidth}\raggedright
Parameter
\end{minipage} & \begin{minipage}[b]{\linewidth}\raggedright
Symbol
\end{minipage} & \begin{minipage}[b]{\linewidth}\raggedright
Unit
\end{minipage} & \begin{minipage}[b]{\linewidth}\raggedright
Equations
\end{minipage} \\
\midrule\noalign{}
\endhead
\bottomrule\noalign{}
\endlastfoot
Activation enthalpy & \(H^{\ast}\) & J mol\(^{-1}\) &
\ref{eq:growth-rate} \\
Activation volume & \(V^{\ast}\) & m\(^3\) mol\(^{-1}\) &
\ref{eq:growth-rate} \\
Compressibility (reference) & \(\bar{\beta}\) & Pa\(^{-1}\) &
\ref{eq:density-ala} \\
Density & \(\rho\) & kg m\(^{-3}\) &
\ref{eq:navier-stokes-no-inertia}--\ref{eq:continuity-expanded},
\ref{eq:density-ala-expansion}--\ref{eq:density-ala} \\
Density (reference) & \(\bar{\rho}\) & kg m\(^{-3}\) &
\ref{eq:adiabatic-pressure}--\ref{eq:density-ala} \\
Density (dynamic) & \(\hat{\rho}\) & kg m\(^{-3}\) & - \\
Deviatoric stess tensor & \(\sigma^{\prime}\) & Pa &
\ref{eq:navier-stokes-no-inertia}, \ref{eq:energy} \\
Deviatoric strain rate tensor & \(\dot{\epsilon}^{\prime}\) & s\(^{-1}\)
& \ref{eq:energy} \\
Dislocation density & \(D\) & m\(^{-2}\) & \ref{eq:growth-rate} \\
Gas constant & \(R\) & J K\(^{-1}\) mol\(^{-1}\) &
\ref{eq:growth-rate} \\
Grain size & \(d\) & 1 m & \ref{eq:growth-rate} \\
Gravitational acceleration & \(g\) & m s\(^{-2}\) &
\ref{eq:navier-stokes-no-inertia},
\ref{eq:adiabatic-temperature}--\ref{eq:adiabatic-pressure} \\
Growth rate & \(\dot{x}\) & m s\(^{-1}\) & \ref{eq:growth-rate} \\
Kinetic prefactor & \(A\) & m s\(^{-1}\) K\(^{-1}\) ppm\(_{OH}^{-n}\) &
\ref{eq:growth-rate} \\
Latent heat & \(Q_L\) & J kg\(^{-1}\) & \ref{eq:energy} \\
Molar entropy & \(\bar{S}\) & J mol\(^{-1}\) K\(^{-1}\) &
\ref{eq:excess-gibbs} \\
Molar Gibbs free-energy & \(\bar{G}\) & J mol\(^{-1}\) &
\ref{eq:excess-gibbs} \\
Molar volume & \(\bar{V}\) & m\(^{3}\) mol\(^{-1}\) &
\ref{eq:excess-gibbs} \\
Nucleation site factor & \(S\) & m\(^{-1}\) & \ref{eq:growth-rate} \\
Phase transition rate & \(\dot{X}\) & s\(^{-1}\) &
\ref{eq:phase-transition-rate} \\
Pressure & \(P\) & Pa & \ref{eq:navier-stokes-no-inertia},
\ref{eq:energy}, \ref{eq:growth-rate} \\
Pressure (reference) & \(\bar{P}\) & K & \ref{eq:adiabatic-pressure} \\
Pressure (dynamic) & \(\hat{P}\) & Pa & \ref{eq:density-ala},
\ref{eq:excess-gibbs} \\
Specific heat capacity (reference) & \(\bar{C}_p\) & J kg\(^{-1}\)
K\(^{-1}\) & \ref{eq:energy}, \ref{eq:adiabatic-temperature} \\
Temperature & \(T\) & K & \ref{eq:energy}, \ref{eq:growth-rate} \\
Temperature (reference) & \(\bar{T}\) & K &
\ref{eq:adiabatic-temperature}, \ref{eq:rheological-model} \\
Temperature (dynamic) & \(\hat{T}\) & K & \ref{eq:density-ala},
\ref{eq:excess-gibbs}, \ref{eq:rheological-model} \\
Thermal conductivity (reference) & \(\bar{k}\) & W m\(^{-1}\) K\(^{-1}\)
& \ref{eq:energy} \\
Thermal expansivity (reference) & \(\bar{\alpha}\) & Pa\(^{-1}\) &
\ref{eq:energy}, \ref{eq:adiabatic-temperature}, \ref{eq:density-ala} \\
Time & \(t\) & s &
\ref{eq:continuity-compressible}--\ref{eq:continuity-expanded},
\ref{eq:volume-fraction}, \ref{eq:phase-transition-rate} \\
Velocity & \(\vec{u}\) & m s\(^{-1}\) &
\ref{eq:continuity-compressible}--\ref{eq:continuity-expanded},
\ref{eq:composition} \\
Viscosity & \(\eta\) & Pa s & \ref{eq:rheological-model} \\
Viscosity (reference) & \(\bar{\eta}\) & Pa s &
\ref{eq:rheological-model} \\
Viscosity exponent & \(B\) & - & \ref{eq:rheological-model} \\
Volume fraction & \(X\) & - & \ref{eq:volume-fraction},
\ref{eq:phase-transition-rate}--\ref{eq:composition} \\
Water content & \(C_{OH}\) & ppm & \ref{eq:growth-rate} \\
Water content exponent & \(n\) & - & \ref{eq:growth-rate} \\
\end{longtable}
}

\cleardoublepage

\section{Introduction}\label{sec:introduction}

The mantle transition zone (MTZ) hosts two major seismic discontinuities
near 410 and 660 km depth, linked to polymorphic phase transitions of
olivine and garnet (\citeproc{ref-katsura2004}{{Katsura et al.}, 2004};
\citeproc{ref-ringwood1975}{Ringwood, 1975}). Global seismic studies
reveal that the depth, sharpness, and amplitude of the 410 vary widely:
some regions display sharp, high-amplitude discontinuities, whereas
others exhibit broad, displaced, or weakened signals
(\citeproc{ref-deuss2009}{Deuss, 2009};
\citeproc{ref-lawrence2008}{Lawrence \& Shearer, 2008};
\citeproc{ref-schmerr2007}{Schmerr \& Garnero, 2007}). Such variability
cannot be explained by equilibrium thermodynamics alone, which predicts
discontinuity topography primarily from temperature and Clapeyron
slopes. Additional factors---notably phase transition kinetics, water
content, and compressibility---likely play central roles
(\citeproc{ref-fukao2013}{Fukao \& Obayashi, 2013};
\citeproc{ref-perrillat2016}{Perrillat et al., 2016};
\citeproc{ref-rubie1994}{Rubie \& Ross II, 1994}).

Mineral-physics experiments demonstrate that the olivine
\(\Leftrightarrow\) wadsleyite transition is strongly rate-limited, with
growth and nucleation controlled by temperature, pressure, water, and
deformation state (\citeproc{ref-kubo2004}{Kubo et al., 2004};
\citeproc{ref-ledoux2023}{{Ledoux et al.}, 2023};
\citeproc{ref-ohuchi2022}{Ohuchi et al., 2022}). Under cold slab
conditions, sluggish kinetics can permit metastable persistence of
olivine well below 410 km, influencing slab stagnation and deep
seismicity (\citeproc{ref-fukao2009}{Fukao et al., 2009}). In hot plume
environments, slow kinetics can broaden and uplift the discontinuity,
potentially explaining reduced amplitudes beneath hotspots
(\citeproc{ref-chambers2005}{Chambers et al., 2005};
\citeproc{ref-jenkins2016}{Jenkins et al., 2016}). Yet kinetic
parameters remain poorly constrained, spanning several orders of
magnitude, and feedbacks introduced by coupling kinetic rate laws to
compressible mantle flow simulations are incompletely understood.

Here we address this gap by embedding a growth-controlled kinetic model
within compressible geodynamic simulations of plumes and slabs using the
numerical geodynamic software ASPECT. We ask: 1) how do feedbacks among
thermodynamics, kinetics, and compressibility shape the 410? 2) What are
the characteristic timescales and lengthscales of these feedbacks? 3)
What are the implications for mantle flow and seismic expression? And 4)
can seismic observations be used to constrain effective kinetic
parameters?

We explore these questions through a systematic suite of numerical
simulations that vary kinetic parameters across the experimentally
constrained ranges of uncertainties. We evaluate the resulting phase
transition zone (PTZ) displacements, widths, and seismic velocity
contrasts. Our results show that plumes display predictable power-law
scalings between kinetics and PTZ structure, while slabs exhibit abrupt
thresholds separating metastable ponding from penetrative behaviour, and
broad PTZs from sharp seismic discontinuities. Together, these findings
demonstrate that phase transition kinetics provide a first-order control
on MTZ structure, help reconcile observed 410 variability, and offer a
pathway to constrain phase transition rates and kinetic parameters from
seismic data.

\cleardoublepage

\section{Methods}\label{sec:methods}

\subsection{Governing Equations for Compressible Mantle
Flow}\label{sec:governing-equations}

Mantle flow is simulated by using the finite-element geodynamic code
ASPECT (v3.0.0, \citeproc{ref-aspect-doi-v3.0.0}{Bangerth et al.,
2024a}, \citeproc{ref-aspectmanual}{2024b};
\citeproc{ref-clevenger2021}{Clevenger \& Heister, 2021};
\citeproc{ref-fraters2019}{{Fraters et al.}, 2019};
\citeproc{ref-fraters2020}{Fraters, 2020};
\citeproc{ref-gassmoller2018}{Gassmöller et al., 2018};
\citeproc{ref-heister2017}{Heister et al., 2017};
\citeproc{ref-kronbichler2012}{Kronbichler et al., 2012}) to find the
velocity \(\vec{u}\), pressure \(P\), and temperature \(T\) fields that
satisfy the following equations:

\begin{equation}
  \nabla P - \nabla \cdot \sigma^{\prime} = \rho\, g
  \label{eq:navier-stokes-no-inertia}
\end{equation}

\begin{equation}
  \frac{\partial \rho}{\partial t} + \nabla \cdot (\rho\, \vec{u}) = 0
  \label{eq:continuity-compressible}
\end{equation}

\begin{equation}
  \rho\, \bar{C}_p \left(\frac{\partial T}{\partial t} + \vec{u} \cdot \nabla T \right) - \nabla \cdot \left(\bar{k}\, \nabla T \right) = \sigma^{\prime} : \dot{\epsilon}^{\prime} + \bar{\alpha}\, T \left(\vec{u} \cdot \nabla P \right) + Q_L
  \label{eq:energy}
\end{equation}

where \(\sigma^{\prime}\) is the deviatoric stress tensor, \(\rho\) is
density, \(g\) is gravitational acceleration, \(t\) is time,
\(\bar{C}_p\), \(\bar{k}\), \(\bar{\alpha}\) are the reference specific
heat capacity, thermal conductivity, and thermal expansivity,
respectively (see Section \ref{sec:adiabatic-reference-conditions}), and
\(Q_L\) is the latent heat released or absorbed during phase
transitions. Equations \ref{eq:navier-stokes-no-inertia} and
\ref{eq:continuity-compressible} together describe the buoyancy-driven
flow of a nearly-incompressible isotropic fluid with negligible inertia
and Equation \ref{eq:energy} describes the conduction, advection, and
production (or consumption) of thermal energy
(\citeproc{ref-schubert2001}{Schubert et al., 2001}). Note that the
pressure \(P\) in this context is equal to the mean normal stress and is
positive under compression:
\(P = - \frac{\sigma_{xx} + \sigma_{yy}}{2}\) (see Appendix
\ref{sec:momentum-derivation}).

The fully compressible form of the continuity equation above (Equation
\ref{eq:continuity-compressible}) can result in a dynamic feedback
between density changes and pressures that cause numerical oscillations
if the the advection timestep is less than the viscous relaxation
timescale (\citeproc{ref-curbelo2019}{Curbelo et al., 2019}). Therefore
the continuity equation is reformulated using the \emph{projected
density approximation} (PDA, \citeproc{ref-gassmoller2020}{Gassmöller et
al., 2020}) by applying the product rule to
\(\nabla \cdot (\rho\, \vec{u})\) and multiplying both sides of Equation
\ref{eq:continuity-compressible} by \(\frac{1}{\rho}\) to get to the
following expression:

\begin{equation}
  \frac{1}{\rho} \frac{\partial \rho}{\partial t} + \nabla \cdot \vec{u} + \left(\frac{1}{\rho} \nabla \rho \right) \cdot \vec{u} = 0
  \label{eq:continuity-expanded}
\end{equation}

Adopting the PDA allows us to retain the coupling between density,
pressure, temperature, and phase transitions, while mitigating the
numerical instabilities associated with the fully compressible form of
the continuity equation (\citeproc{ref-gassmoller2020}{Gassmöller et
al., 2020}). The PDA formulation is therefore particularly appropriate
for our models, which include dynamic thermal effects as well as an
olivine \(\Leftrightarrow\) wadsleyite phase transition that would
otherwise be neglected by incompressible forms of the continuity
equation (e.g., Boussinesq). This formulation unique to ASPECT provides
a stable and internally consistent framework for solving the governing
equations---ensuring that compressibility-related feedbacks were
accurately represented without compromising numerical robustness.

\subsection{Numerical Setup}\label{sec:numerical-setup}

\subsubsection{Adiabatic Reference
Conditions}\label{sec:adiabatic-reference-conditions}

In order to effectively converge on a solution, we needed to initialize
our ASPECT simulations with reasonable guesses for the
pressure-temperature (PT) fields and material properties in Earth's
upper mantle (\citeproc{ref-aspectmanual}{Bangerth et al., 2024b};
\citeproc{ref-heister2017}{Heister et al., 2017};
\citeproc{ref-kronbichler2012}{Kronbichler et al., 2012}). For this
purpose, we began by evaluating entropy changes over a PT range of
1573--1973 K and 0.001--25 GPa (Figure \ref{fig:isentrope}) using the
Gibbs free-energy minimization software Perple\_X (v.7.0.9,
\citeproc{ref-connolly2009}{Connolly, 2009}). We assumed a dry pyrolitic
bulk composition after Green et al. (\citeproc{ref-green1979}{1979}) and
phase equilibria were evaluated in the
Na\(_2\)O‐CaO‐FeO‐MgO‐Al\(_2\)O\(_3\)‐SiO\(_2\) (NCFMAS) chemical system
with thermodynamic data and solution models of Stixrude \&
Lithgow-Bertelloni (\citeproc{ref-stixrude2022}{2022}). Equations of
state were included for solid solution phases: olivine, plagioclase,
spinel, clinopyroxene, wadsleyite, ringwoodite, perovskite,
ferropericlase, high‐pressure C2/c pyroxene, orthopyroxene, akimotoite,
post‐perovskite, Ca‐ferrite, garnet, and Na‐Al phase.

\begin{figure}
\centering
\includegraphics[width=0.75\linewidth,height=\textheight,keepaspectratio,alt={Entropy (a) and density (b) changes in Earth's upper mantle under thermodynamic equilibrium. Material properties were computed with Perple\_X using the equations of state and thermodynamic data of Stixrude \& Lithgow-Bertelloni (2022). The black box indicates the approximate PT range of our ASPECT simulations, while the white line indicates the isentropic adiabat used to calculate reference material properties.}]{../figs/PYR-material-table.png}
\caption{Entropy (a) and density (b) changes in Earth's upper mantle
under thermodynamic equilibrium. Material properties were computed with
Perple\_X using the equations of state and thermodynamic data of
Stixrude \& Lithgow-Bertelloni (\citeproc{ref-stixrude2022}{2022}). The
black box indicates the approximate PT range of our ASPECT simulations,
while the white line indicates the isentropic adiabat used to calculate
reference material properties.}\label{fig:isentrope}
\end{figure}

We then determined the mantle adiabat by applying the Newton--Raphson
algorithm to find corresponding temperatures for each pressure such that
entropy remains constant (white line in Figure \ref{fig:isentrope}).
Material properties were evaluated at each PT point along the isentrope
to construct the adiabatic reference conditions shown in Figure
\ref{fig:material-property-profiles}. These reference conditions serve
three main purposes: 1) initializing the PT fields and material
properties in our ASPECT simulations (see Section
\ref{sec:initialization-and-boundary-conditions}), 2) updating the
material model during the simulations (see Section
\ref{sec:material-model}), and 3) serving as a basis for computing
``dynamic'' quantities, such as the dynamic temperature
\(\hat{T} = T - \bar{T}\), dynamic pressure \(\hat{P} = P - \bar{P}\),
and dynamic density \(\hat{\rho} = \rho - \bar{\rho}\), that quantify
how much the approximate numerical solution deviates from the reference
adiabatic conditions (i.e., a non-convecting ambient mantle).

\begin{figure}
\centering
\pandocbounded{\includegraphics[keepaspectratio,alt={Reference material properties used in our ASPECT simulations. Profiles were computed using the BurnMan software (Cottaar et al., 2014; Myhill et al., 2023) and were based on the equations of state and thermodynamic data of Stixrude \& Lithgow-Bertelloni (2022) for pure Mg\_\{0.9\}Fe\_\{0.01\} olivine (ol) and wadsleyite (wad).}]{../figs/material-property-profiles.png}}
\caption{Reference material properties used in our ASPECT simulations.
Profiles were computed using the BurnMan software
(\citeproc{ref-cottaar2014}{Cottaar et al., 2014};
\citeproc{ref-myhill2023}{Myhill et al., 2023}) and were based on the
equations of state and thermodynamic data of Stixrude \&
Lithgow-Bertelloni (\citeproc{ref-stixrude2022}{2022}) for pure
Mg\(_{0.9}\)Fe\(_{0.01}\) olivine (ol) and wadsleyite
(wad).}\label{fig:material-property-profiles}
\end{figure}

\subsubsection{Initialization and Boundary
Conditions}\label{sec:initialization-and-boundary-conditions}

Our ASPECT simulations were initialized with pure
Mg\(_{0.9}\)Fe\(_{0.1}\) olivine and wadsleyite within a 396 \(\times\)
264 km rectangular model domain (Figure \ref{fig:initial-setup}).
``Surface'' PT conditions of 10 GPa and 1706 K were applied at the top
boundary such that the olivine \(\Leftrightarrow\) wadsleyite transition
occurs approximately half-way down the model. The initial PT fields were
then computed by numerically integrating the following equations:

\begin{equation}
  \frac{d\bar{T}}{dy} = \frac{\bar{\alpha}\, \bar{T}\, g}{\bar{C}_p}
  \label{eq:adiabatic-temperature}
\end{equation}

\begin{equation}
  \frac{d\bar{P}}{dy} = \bar{\rho}\, g
  \label{eq:adiabatic-pressure}
\end{equation}

where \(d\bar{P}/dy\) and \(d\bar{T}/dy\) are adiabatic PT profiles
applied uniformly across the model domain, \(g\) is gravitational
acceleration, and the density \(\bar{\rho}\), thermal expansivity
\(\bar{\alpha}\), and specific heat capacity \(\bar{C}_p\) were
determined from the adiabatic reference conditions shown in Figure
\ref{fig:material-property-profiles}. Normal Guassian-shaped thermal
anomalies of \(\leq\) 500 K were then applied at the top and bottom
boundaries for slab and plume simulations, respectively.

\begin{figure}
\centering
\includegraphics[width=0.7\linewidth,height=\textheight,keepaspectratio,alt={Initial setup and boundary conditions for slab (top) and plume (bottom) simulations. The thermal anomalies and prescribed inflows for slab and plume models were essentially mirrored, except for a horizontal boundary velocity of \textbackslash vec\{u\}\_x = 0.5 cm/yr applied to slab models. Traction boundary conditions (\textbackslash sigma\_\{xy\} and \textbackslash sigma\_\{yy\}) ensure that outflows must be driven by dynamic pressures. The top boundary has a constant ``surface'' PT of 10 GPa and 1706 K.}]{../figs/initial-setup.png}
\caption{Initial setup and boundary conditions for slab (top) and plume
(bottom) simulations. The thermal anomalies and prescribed inflows for
slab and plume models were essentially mirrored, except for a horizontal
boundary velocity of \(\vec{u}_x\) = 0.5 cm/yr applied to slab models.
Traction boundary conditions (\(\sigma_{xy}\) and \(\sigma_{yy}\))
ensure that outflows must be driven by dynamic pressures. The top
boundary has a constant ``surface'' PT of 10 GPa and 1706
K.}\label{fig:initial-setup}
\end{figure}

Velocity and traction boundary conditions were set to ensures that
outflow of material from the model must be driven by dynamic pressures
generated by convection and/or volume changes due to the olivine
\(\Leftrightarrow\) wadsleyite phase transition. For slab models, a
prescribed inflow velocity of \(\vec{u}_x\) = 0.5 and \(\vec{u}_y\) =
-1.5 cm/yr was applied at the top boundary. All other boundaries (right,
bottom, left) were prescribed a constant horizontal velocity of
\(\vec{u}_x\) = 0.5 cm/yr and constant vertical stress equal to the
initial hydrostatic pressure \(\bar{P}\) determined by numerical
integration of Equation \ref{eq:adiabatic-pressure}. Velocity and
traction boundary conditions for plume models essentially mirror those
of slab models with one key change: plume models have zero horizontal
velocity at all boundaries.

\subsubsection{Material Model}\label{sec:material-model}

\paragraph{Material Properties}\label{sec:material-properties}

Material properties were updated during our ASPECT simulations by
referencing the adiabatic reference conditions shown in Figure
\ref{fig:material-property-profiles}. Besides density, no PT corrections
were applied to material properties, effectively assuming that
deviations in material properties from a non-convecting ambient mantle
were negligible. For density, however, we applied a dynamic PT
correction through a first-order Taylor expansion
(\citeproc{ref-gassmoller2020}{Gassmöller et al., 2020};
\citeproc{ref-jarvis1980}{Jarvis \& Mckenzie, 1980}):

\begin{equation}
  \rho \approx \bar{\rho} + \left(\frac{\partial \bar{\rho}}{\partial P} \right)_T \Delta P + \left(\frac{\partial \bar{\rho}}{\partial T} \right)_P \Delta T
  \label{eq:density-ala-expansion}
\end{equation}

Equation \ref{eq:density-ala-expansion} is rewritten using standard
thermodynamic relations
\(\beta = \frac{1}{\rho} \left(\frac{\partial \rho}{\partial P}\right)_T\)
and
\(\alpha = -\frac{1}{\rho} \left(\frac{\partial \rho}{\partial T}\right)_P\)
to get the expression:

\begin{equation}
  \rho = \bar{\rho} \left(1 + \bar{\beta}\, \hat{P} - \bar{\alpha}\, \hat{T} \right)
  \label{eq:density-ala}
\end{equation}

where \(\bar{\rho}\), \(\, \bar{\beta}\), \(\, \bar{\alpha}\), are the
adiabatic reference density, compressibility, and thermal expansivity,
respectively, and \(\Delta P = \hat{P} = P - \bar{P}\) and
\(\Delta T = \hat{T} = T - \bar{T}\) are the dynamic PT. Note that the
reference thermal conductivity \(\bar{k}\) = 4.0 Wm\(^{-1}\)K\(^{-1}\)
is constant in all our numerical experiments.

\paragraph{Phase Transition
Kinetics}\label{sec:phase-transition-kinetics}

The kinetics of the olivine \(\Leftrightarrow\) wadsleyite phase
transition were governed entirely by interface growth, as nucleation was
assumed to saturate rapidly and did not limit the phase transition
(\citeproc{ref-cahn1956}{Cahn, 1956}; \citeproc{ref-hosoya2005}{Hosoya
et al., 2005}). Following Faccenda \& Dal Zilio
(\citeproc{ref-faccenda2017}{2017}), the transformed volume fraction is
given by:

\begin{equation}
  X = 1 - \exp\left(-S\, \dot{x}\, t \right)
  \label{eq:volume-fraction}
\end{equation}

where \(X\) is the volume fraction of the product phase (olivine or
wadsleyite), \(S\) is a geometric factor that accounts for nucleation
sites, \(\dot{x}\) is the interface growth rate, and \(t\) is the
elapsed time after site saturation. For inter-crystalline grain-boundary
controlled growth, \(S = 6.67/d\), where \(d\) is grain size. For
intra-crystalline dislocation-controlled growth, \(S = 2\sqrt{D}\),
where \(D\) is dislocation density (after
\citeproc{ref-mohiuddin2018}{Mohiuddin \& Karato, 2018}).

Since we assumed growth-controlled kinetics, the expression for the
interface growth rate \(\dot{x}\) determined the overall phase
transition rate (\citeproc{ref-turnbull1956}{Turnbull, 1956}):

\begin{equation}
  \dot{x} = A\, T\, C_{OH}^n \exp\left(-\frac{H^{\ast} + P V^{\ast}}{R\, T}\right) \left(1 - \exp\left[-\frac{\Delta G}{R\, T}\right] \right)
  \label{eq:growth-rate}
\end{equation}

where \(A\) is a kinetic prefactor, \(C_{OH}\) is the concentration of
water in the reactant phase, \(n\) is the water content exponent,
\(H^{\ast}\) is activation enthalphy, \(V^{\ast}\) is activation volume,
\(P\) is pressure, \(T\) is temperature, \(R\) is the gas constant, and
\(\Delta G\) is the excess Gibbs free-energy difference between olivine
and wadsleyite. The excess Gibbs free-energy \(\Delta G\) is
approximated by:

\begin{equation}
  \Delta G \approx \Delta \bar{G} + \hat{P}\, \Delta \bar{V} - \hat{T}\, \Delta \bar{S}
  \label{eq:excess-gibbs}
\end{equation}

where \(\Delta \bar{G}\), \(\Delta \bar{V}\), and \(\Delta \bar{S}\) are
the Gibbs free-energy, volume, and entropy differences between olivine
and wadsleyite along the adiabatic reference profile (Figure
\ref{fig:phase-transition-kinetics-profile}), and \(\hat{P}\) and
\(\hat{T}\) are the dynamic PT.

In this formulation, the time evolution of the olivine
\(\Leftrightarrow\) wadsleyite phase transition is fully described by
the interplay of pressure, temperature, and growth parameters (Table
\ref{tbl:growth-parameters}), without explicit consideration of
nucleation kinetics (\citeproc{ref-cahn1956}{Cahn, 1956};
\citeproc{ref-faccenda2017}{Faccenda \& Dal Zilio, 2017};
\citeproc{ref-hosoya2005}{Hosoya et al., 2005};
\citeproc{ref-rubie1994}{Rubie \& Ross II, 1994}). The macro-scale
olivine \(\Leftrightarrow\) wadsleyite phase transition rate was
therefore computed by taking the time derivative of Equation
\ref{eq:volume-fraction}:

\begin{equation}
  \frac{dX}{dt} = \dot{X} = S\, \dot{x}\, \left(1 - X \right)
  \label{eq:phase-transition-rate}
\end{equation}

\paragraph{Operator Splitting}\label{sec:operator-splitting}

Since the phase transition rate \(\dot{X}\) is slower than the advection
timescale in our ASPECT simulations, we employ a first-order operator
splitting scheme to decouple advection from phase-change kinetics. In
this approach, the phase fraction is updated in two sequential steps
within each overall time step \(\Delta t\):

\begin{equation}
  \frac{\partial X}{\partial t} + \vec{u} \cdot \nabla X = 0
  \label{eq:composition}
\end{equation}

\begin{enumerate}
\def\labelenumi{\arabic{enumi}.}
\tightlist
\item
  \textbf{Advection step:} Solve the transport of material
  \(\left(\vec{u} \cdot \nabla X \right)\) without phase changes over
  the time interval \(\Delta t\) to yield an intermediate composition
  \(X^\ast\)
\item
  \textbf{Reaction step:} starting from \(X^\ast\), integrate Equation
  \ref{eq:phase-transition-rate} over the same time interval using a
  smaller sub-step \(\delta t \le \Delta t\) to obtain the updated
  composition \(X^{n+1}\)
\end{enumerate}

This operator splitting scheme ensures numerical stability while
accurately capturing slow kinetics without restricting the convective
timestep (\citeproc{ref-aspectmanual}{Bangerth et al., 2024b}).

\begin{figure}
\centering
\includegraphics[width=0.8\linewidth,height=\textheight,keepaspectratio,alt={Reference thermodynamic properties used in our ASPECT simulations. Profiles were computed using the BurnMan software (Cottaar et al., 2014; Myhill et al., 2023) and are based on the equations of state and thermodynamic data of Stixrude \& Lithgow-Bertelloni (2022) for pure Mg\_\{0.9\}Fe\_\{0.01\} olivine (ol) and wadsleyite (wad).}]{../figs/phase-transition-kinetics-profile.png}
\caption{Reference thermodynamic properties used in our ASPECT
simulations. Profiles were computed using the BurnMan software
(\citeproc{ref-cottaar2014}{Cottaar et al., 2014};
\citeproc{ref-myhill2023}{Myhill et al., 2023}) and are based on the
equations of state and thermodynamic data of Stixrude \&
Lithgow-Bertelloni (\citeproc{ref-stixrude2022}{2022}) for pure
Mg\(_{0.9}\)Fe\(_{0.01}\) olivine (ol) and wadsleyite
(wad).}\label{fig:phase-transition-kinetics-profile}
\end{figure}

\subsubsection{Rheological Model}\label{sec:rheological-model}

Our ASPECT simulations use a simple rheological model where mantle
viscosity is modified by a depth-dependent piecewise function:

\begin{equation}
  \bar{\eta} =
  \begin{cases}
    1\, \eta_0, & y \leq 132 \text{km} \\
    1\, \eta_0, & y > 132 \text{km}
  \end{cases}
  \label{eq:piecewise-viscosity-function}
\end{equation}

where \(\eta_0\) is the nominal background viscosity of the upper mantle
(\citeproc{ref-karato2008}{Karato, 2008};
\citeproc{ref-ranalli1995}{Ranalli, 1995}). A thermal dependency is then
implemented through an exponential term:

\begin{equation}
  \eta = \bar{\eta} \exp \left(-B\, \frac{\hat{T}}{\bar{T}} \right)
  \label{eq:rheological-model}
\end{equation}

where \(B\) is the thermal viscosity exponent factor, and \(\hat{T}\) is
the dynamic temperature.

\begin{longtable}[]{@{}
  >{\raggedright\arraybackslash}p{(\linewidth - 18\tabcolsep) * \real{0.1000}}
  >{\raggedleft\arraybackslash}p{(\linewidth - 18\tabcolsep) * \real{0.1000}}
  >{\raggedleft\arraybackslash}p{(\linewidth - 18\tabcolsep) * \real{0.1000}}
  >{\raggedleft\arraybackslash}p{(\linewidth - 18\tabcolsep) * \real{0.1000}}
  >{\raggedleft\arraybackslash}p{(\linewidth - 18\tabcolsep) * \real{0.1000}}
  >{\raggedleft\arraybackslash}p{(\linewidth - 18\tabcolsep) * \real{0.1000}}
  >{\raggedleft\arraybackslash}p{(\linewidth - 18\tabcolsep) * \real{0.1000}}
  >{\raggedleft\arraybackslash}p{(\linewidth - 18\tabcolsep) * \real{0.1000}}
  >{\raggedleft\arraybackslash}p{(\linewidth - 18\tabcolsep) * \real{0.1000}}
  >{\raggedleft\arraybackslash}p{(\linewidth - 18\tabcolsep) * \real{0.1000}}@{}}
\caption{\label{tbl:growth-parameters}Full list of kinetic and
rheological parameters used in plume and slab simulations. Kinetic
parameter values are consistent with the range of experimental data from
Hosoya et al. (\citeproc{ref-hosoya2005}{2005}). Units are \(A\): m
s\(^{-1}\) K\(^{-1}\) ppm\(_{OH}^{-n}\), \(H^{\ast}\): J mol\(^{-1}\),
\(V^{\ast}\): m\(^3\) mol\(^{-1}\), \(d\): m, \(D\): m\(^{-2}\),
\(C_{OH}\): ppm, \(n\): --, \(\eta_0\): Pa s, \(B\):
--.}\label{tbl:growth-parameters}\tabularnewline
\toprule\noalign{}
\begin{minipage}[b]{\linewidth}\raggedright
Type
\end{minipage} & \begin{minipage}[b]{\linewidth}\raggedleft
\(A\)
\end{minipage} & \begin{minipage}[b]{\linewidth}\raggedleft
\(H^{\ast}\)
\end{minipage} & \begin{minipage}[b]{\linewidth}\raggedleft
\(V^{\ast}\)
\end{minipage} & \begin{minipage}[b]{\linewidth}\raggedleft
\(d\)
\end{minipage} & \begin{minipage}[b]{\linewidth}\raggedleft
\(D\)
\end{minipage} & \begin{minipage}[b]{\linewidth}\raggedleft
\(C_{OH}\)
\end{minipage} & \begin{minipage}[b]{\linewidth}\raggedleft
\(n\)
\end{minipage} & \begin{minipage}[b]{\linewidth}\raggedleft
\(\eta_0\)
\end{minipage} & \begin{minipage}[b]{\linewidth}\raggedleft
\(B\)
\end{minipage} \\
\midrule\noalign{}
\endfirsthead
\toprule\noalign{}
\begin{minipage}[b]{\linewidth}\raggedright
Type
\end{minipage} & \begin{minipage}[b]{\linewidth}\raggedleft
\(A\)
\end{minipage} & \begin{minipage}[b]{\linewidth}\raggedleft
\(H^{\ast}\)
\end{minipage} & \begin{minipage}[b]{\linewidth}\raggedleft
\(V^{\ast}\)
\end{minipage} & \begin{minipage}[b]{\linewidth}\raggedleft
\(d\)
\end{minipage} & \begin{minipage}[b]{\linewidth}\raggedleft
\(D\)
\end{minipage} & \begin{minipage}[b]{\linewidth}\raggedleft
\(C_{OH}\)
\end{minipage} & \begin{minipage}[b]{\linewidth}\raggedleft
\(n\)
\end{minipage} & \begin{minipage}[b]{\linewidth}\raggedleft
\(\eta_0\)
\end{minipage} & \begin{minipage}[b]{\linewidth}\raggedleft
\(B\)
\end{minipage} \\
\midrule\noalign{}
\endhead
\bottomrule\noalign{}
\endlastfoot
plume & \(e^{-15}\) & 274 & 3e-6 & 0.005 & - & 60 & 3.2 & 1e21 & 1 \\
plume & \(e^{-16}\) & 274 & 3e-6 & 0.005 & - & 60 & 3.2 & 1e21 & 1 \\
plume & \(e^{-17}\) & 274 & 3e-6 & 0.005 & - & 60 & 3.2 & 1e21 & 1 \\
plume & \(e^{-18}\) & 274 & 3e-6 & 0.001 & - & 60 & 3.2 & 1e21 & 1 \\
plume & \(e^{-18}\) & 274 & 3e-6 & 0.005 & - & 60 & 3.2 & 1e21 & 1 \\
plume & \(e^{-18}\) & 274 & 3e-6 & 0.01 & - & 60 & 3.2 & 1e21 & 1 \\
plume & \(e^{-18}\) & 274 & 3e-6 & 0.005 & - & 80 & 3.2 & 1e21 & 1 \\
plume & \(e^{-18}\) & 274 & 3e-6 & 0.005 & - & 100 & 3.2 & 1e21 & 1 \\
plume & \(e^{-18}\) & 274 & 3e-6 & 0.005 & - & 120 & 3.2 & 1e21 & 1 \\
plume & \(e^{-18}\) & 274 & 3e-6 & 0.005 & - & 140 & 3.2 & 1e21 & 1 \\
plume & \(e^{-18}\) & 274 & 3e-6 & - & 1e6 & 60 & 3.2 & 1e21 & 1 \\
plume & \(e^{-18}\) & 274 & 3e-6 & - & 1e8 & 60 & 3.2 & 1e21 & 1 \\
plume & \(e^{-18}\) & 274 & 3e-6 & - & 1e10 & 60 & 3.2 & 1e21 & 1 \\
plume & \(e^{-18}\) & 274 & 3e-6 & - & 1e12 & 60 & 3.2 & 1e21 & 1 \\
plume & \(e^{-18}\) & 300 & 3e-6 & 0.005 & - & 60 & 3.2 & 1e21 & 1 \\
plume & \(e^{-19}\) & 274 & 3e-6 & 0.005 & - & 60 & 3.2 & 1e21 & 1 \\
plume & \(e^{-20}\) & 274 & 3e-6 & 0.005 & - & 60 & 3.2 & 1e21 & 1 \\
plume & \(e^{-21}\) & 274 & 3e-6 & 0.005 & - & 60 & 3.2 & 1e21 & 1 \\
slab & \(e^{-15}\) & 274 & 3e-6 & 0.005 & - & 60 & 3.2 & 1e21 & 1 \\
slab & \(e^{-16}\) & 274 & 3e-6 & 0.005 & - & 60 & 3.2 & 1e21 & 1 \\
slab & \(e^{-17}\) & 274 & 3e-6 & 0.005 & - & 60 & 3.2 & 1e21 & 1 \\
slab & \(e^{-18}\) & 274 & 3e-6 & 0.001 & - & 60 & 3.2 & 1e21 & 1 \\
slab & \(e^{-18}\) & 274 & 3e-6 & 0.005 & - & 60 & 3.2 & 1e21 & 1 \\
slab & \(e^{-18}\) & 274 & 3e-6 & 0.01 & - & 60 & 3.2 & 1e21 & 1 \\
slab & \(e^{-18}\) & 274 & 3e-6 & 0.005 & - & 80 & 3.2 & 1e21 & 1 \\
slab & \(e^{-18}\) & 274 & 3e-6 & 0.005 & - & 100 & 3.2 & 1e21 & 1 \\
slab & \(e^{-18}\) & 274 & 3e-6 & 0.005 & - & 120 & 3.2 & 1e21 & 1 \\
slab & \(e^{-18}\) & 274 & 3e-6 & 0.005 & - & 140 & 3.2 & 1e21 & 1 \\
slab & \(e^{-18}\) & 274 & 3e-6 & - & 1e6 & 60 & 3.2 & 1e21 & 1 \\
slab & \(e^{-18}\) & 274 & 3e-6 & - & 1e8 & 60 & 3.2 & 1e21 & 1 \\
slab & \(e^{-18}\) & 274 & 3e-6 & - & 1e10 & 60 & 3.2 & 1e21 & 1 \\
slab & \(e^{-18}\) & 274 & 3e-6 & - & 1e12 & 60 & 3.2 & 1e21 & 1 \\
slab & \(e^{-18}\) & 300 & 3e-6 & 0.005 & - & 60 & 3.2 & 1e21 & 1 \\
slab & \(e^{-19}\) & 274 & 3e-6 & 0.005 & - & 60 & 3.2 & 1e21 & 1 \\
slab & \(e^{-20}\) & 274 & 3e-6 & 0.005 & - & 60 & 3.2 & 1e21 & 1 \\
slab & \(e^{-21}\) & 274 & 3e-6 & 0.005 & - & 60 & 3.2 & 1e21 & 1 \\
\end{longtable}

\cleardoublepage

\section{Results}\label{sec:results}

\subsection{Simulation Snapshots: Slabs and
Plumes}\label{sec:simulation-snapshots}

Figures \ref{fig:slab-comp-set2} and \ref{fig:plume-comp-set2}
illustrate how thermodynamics and phase transition kinetics collude to
complicate seismic structures in a convecting mantle. These
``snapshots'', taken after 100 Myr of evolution, provide context for the
quantitative analysis presented below.

In slab models (Figure \ref{fig:slab-comp-set2}), sluggish kinetics
allow extensive metastable olivine to persist, producing displaced PTZs
with faded \(\hat{T}\), \(\hat{\rho}\), and \(V_p\) gradients. As
kinetics accelerate, the olivine \(\Leftrightarrow\) wadsleyite
transition progresses more rapidly within the slab interior, sharpening
the PTZ and generating narrower, higher-contrast \(V_p\) gradients
closer to the nominal (equilibrium) transition depth. Faster kinetics
also promotes continuous slab descent through the transition zone, while
sluggish kinetics favor stagnation and ponding.

Plume models (Figure \ref{fig:plume-comp-set2}) exhibit complementary
behavior. Sluggish kinetics delay the wadsleyite \(\Leftrightarrow\)
olivine reaction within hot upwellings, yielding broad PTZs displaced
upward by tens of kilometers. Moderately fast kinetics concentrate the
phase transition into a narrower vertical interval, strengthening
\(\hat{\rho}\) anomalies and \(V_p\) gradients. Under fast kinetic
regimes, plumes display sharp reaction fronts confined to thin PTZ
interfaces, with the largest density anomalies and most strongly focused
seismic contrasts. Rapid conversion also enhances the vertical velocity
of plume heads and penetration through the PTZ.

These visualizations demonstrate how the kinetics of the olivine
\(\Leftrightarrow\) wadsleyite transition govern whether slabs and
plumes generate diffuse, low-amplitude anomalies or sharp, high-contrast
seismic signatures. They also demonstrate that phase transition
kinetics---in addition to standard thermodynamic considerations (i.e.,
Clapeyron slopes)---play a crucial role in developing PTZ topography.
Larger sets of visualized simulation outputs for the models in Figures
\ref{fig:slab-comp-set2}--\ref{fig:plume-comp-set2} are shown in
Appendix \ref{sec:simulation-snapshots-continued}.

\begin{figure}
\centering
\pandocbounded{\includegraphics[keepaspectratio,alt={Slab model with sluggish (a--c: slab-lnk18-Ha300-d5mm-060ppm), moderately-sluggish (d--f: slab-lnk18-Ha274-d5mm-060ppm), and fast (g--i: slab-lnk18-Ha274-D1e12-060ppm) kinetics after 100 Ma evolution. Panels show dynamic temperature \textbackslash hat\{T\} (left column), dynamic density \textbackslash hat\{\textbackslash rho\} (middle column), and pressure-wave velocity V\_p (right column).}]{../figs/simulation/2d_box/compositions/slab-slow-default-fast-set2-composition-0100.png}}
\caption{Slab model with sluggish (a--c: slab-lnk18-Ha300-d5mm-060ppm),
moderately-sluggish (d--f: slab-lnk18-Ha274-d5mm-060ppm), and fast
(g--i: slab-lnk18-Ha274-D1e12-060ppm) kinetics after 100 Ma evolution.
Panels show dynamic temperature \(\hat{T}\) (left column), dynamic
density \(\hat{\rho}\) (middle column), and pressure-wave velocity
\(V_p\) (right column).}\label{fig:slab-comp-set2}
\end{figure}

\begin{figure}
\centering
\pandocbounded{\includegraphics[keepaspectratio,alt={Plume model with sluggish (a--c: plume-lnk18-Ha300-d5mm-060ppm), moderately-sluggish (d--f: plume-lnk18-Ha274-d5mm-060ppm), and fast (g--i: plume-lnk18-Ha274-D1e12-060ppm) kinetics after 100 Ma evolution. Panels show dynamic temperature \textbackslash hat\{T\} (left column), dynamic density \textbackslash hat\{\textbackslash rho\} (middle column), and pressure-wave velocity V\_p (right column).}]{../figs/simulation/2d_box/compositions/plume-slow-default-fast-set2-composition-0100.png}}
\caption{Plume model with sluggish (a--c:
plume-lnk18-Ha300-d5mm-060ppm), moderately-sluggish (d--f:
plume-lnk18-Ha274-d5mm-060ppm), and fast (g--i:
plume-lnk18-Ha274-D1e12-060ppm) kinetics after 100 Ma evolution. Panels
show dynamic temperature \(\hat{T}\) (left column), dynamic density
\(\hat{\rho}\) (middle column), and pressure-wave velocity \(V_p\)
(right column).}\label{fig:plume-comp-set2}
\end{figure}

\cleardoublepage

\subsection{Phase Transition Zone: Displacement and
Width}\label{sec:ptz-displacement-width}

Figures \ref{fig:ptz-plumes}--\ref{fig:ptz-slabs} summarize the
quantitative relationships among PTZ displacement, effective PTZ width,
and maximum phase transition rate \(\dot{X}_{\mathrm{max}}\) in plume
and slab models after 100 Ma of evolution (Table
\ref{tbl:centerline-profile-results}). PTZ width and displacement were
evaluated from phase fraction field \(X\) along a vertical profile
through the center of the model domain, where width was defined as the
difference between the depths at \(X\) = 0.9 and \(X\) = 0.1, and
displacement was defined as the offset between the nominal equilibrium
reaction depth and the depth at \(X\) = 0.9. The maximum phase
transition rate, \(\dot{X}_{\mathrm{max}}\), was evaluated from the
phase transition rate field \(\dot{X}\) along the same vertical profile.
The results highlight contrasting responses of the PTZ in plume and slab
models to variations in phase transition kinetics.

Plume models (Figure \ref{fig:ptz-plumes}) show well-defined monotonic
trends. PTZ width increases linearly with upward displacement, while
both width and displacement decrease smoothly (as power-laws) with
increasing \(\dot{X}_{\mathrm{max}}\). These consistent relationships
demonstrate simple scaling between PTZ structure and phase transition
kinetics for plumes.

\begin{figure}
\centering
\pandocbounded{\includegraphics[keepaspectratio,alt={Quantitative relationships between PTZ structure and phase transition rate in plume models after 100 Ma. Plume PTZs widen linearly with upward displacement (a), and both width (b) and displacement (c) decrease monotonically with increasing phase transition rate.}]{../figs/ptz-plumes.png}}
\caption{Quantitative relationships between PTZ structure and phase
transition rate in plume models after 100 Ma. Plume PTZs widen linearly
with upward displacement (a), and both width (b) and displacement (c)
decrease monotonically with increasing phase transition
rate.}\label{fig:ptz-plumes}
\end{figure}

In contrast to plume models, slab models (Figure \ref{fig:ptz-slabs})
show nonlinear and non-monotonic behavior. PTZ width first broadens with
downward displacement but narrows again once displacement exceeds
\textasciitilde-50 km, forming a quadratic trend. The quadratic trend in
Figure \ref{fig:ptz-slabs}a points to a threshold effect, where strong
thermodynamic driving forces accelerate reactions and sharpen the PTZ at
displacements \(\leq\) -50 km. This threshold effect is also evident in
Figures \ref{fig:ptz-slabs}b--\ref{fig:ptz-slabs}c, where PTZ width
first increases and then decreases with increasing
\(\dot{X}_{\mathrm{max}}\) as a complex power law (quadratic in log-log
space), and displacement versus \(\dot{X}_{\mathrm{max}}\) shows a
weaker power-law correlation than plume models.

\begin{figure}
\centering
\pandocbounded{\includegraphics[keepaspectratio,alt={Quantitative relationships between PTZ structure and phase transition rate in slab models after 100 Ma. Slab PTZs broaden with moderate downward displacement but narrow again at displacements \textbackslash leq -50 km (a). Width increases, then decreases with phase transition rate, but not as a simple power law (b), while displacement versus phase transition rate shows a weaker correlation than plume models (c). Outliers (in red) deviate from the quadratic trend in (a). Solid lines exclude outliers, while dashed lines were fit to the entire dataset.}]{../figs/ptz-slabs.png}}
\caption{Quantitative relationships between PTZ structure and phase
transition rate in slab models after 100 Ma. Slab PTZs broaden with
moderate downward displacement but narrow again at displacements
\(\leq\) -50 km (a). Width increases, then decreases with phase
transition rate, but not as a simple power law (b), while displacement
versus phase transition rate shows a weaker correlation than plume
models (c). Outliers (in red) deviate from the quadratic trend in (a).
Solid lines exclude outliers, while dashed lines were fit to the entire
dataset.}\label{fig:ptz-slabs}
\end{figure}

In summary, plume PTZs follow linear and power-law scalings that are
smooth and predictable, whereas slab PTZs exhibit quadratic and curved
relationships shaped by threshold behavior. These contrasting trends
underscore the different ways phase transition kinetics govern PTZ
structure in upwellings versus subducting slabs.

\cleardoublepage

\section{Discussion}\label{sec:discussion}

Our numerical simulations show that growth-controlled phase-transition
kinetics, when coupled to realistic thermodynamics and a compressible
treatment of mantle flow, exert a first-order control on the geometry
and seismic expression of the 410 km phase transition. Our results
demonstrate that plume and slab flow regimes respond in systematically
different ways: plumes produce monotonic, smoothly varying PTZ scalings
with kinetics, whereas slabs show thresholded, non-monotonic behavior
(Figures \ref{fig:ptz-plumes}--\ref{fig:ptz-slabs}; Table
\ref{tbl:centerline-profile-results}). The implications of such
contrasting relationships in slabs versus plumes are discussed below.

\subsection{Uncertainties and Model
Limitations}\label{sec:uncertainties-and-model-limitations}

The kinetic parameters employed here span several orders of magnitude
(Table \ref{tbl:growth-parameters}), reflecting the substantial
uncertainties in experimental constraints on olivine \(\Leftrightarrow\)
wadsleyite transition rates. Laboratory studies yield activation
enthalpies \(H^{\ast}\) ranging from \textasciitilde200--500 kJ/mol and
kinetic prefactors \(A\) varying by 6--8 orders of magnitude depending
on water content, grain size, and deformation mechanism
(\citeproc{ref-hosoya2005}{Hosoya et al., 2005};
\citeproc{ref-kubo2004}{Kubo et al., 2004};
\citeproc{ref-rubie1994}{Rubie \& Ross II, 1994}). Moreover, our choice
of grain-boundary versus dislocation-controlled growth mechanisms
significantly impacts the nucleation site factor \(S\) and thus the
overall transition rate (\citeproc{ref-mohiuddin2018}{Mohiuddin \&
Karato, 2018}). These parameter uncertainties are the principal source
of quantitative uncertainty in PTZ widths, displacements and associated
seismic contrasts in our numerical simulations.

We assumed rapid nucleation saturation and growth-limited kinetics
(Equations \ref{eq:volume-fraction}--\ref{eq:phase-transition-rate}),
which is appropriate for many experimental conditions
(\citeproc{ref-cahn1956}{Cahn, 1956};
\citeproc{ref-faccenda2017}{Faccenda \& Dal Zilio, 2017};
\citeproc{ref-hosoya2005}{Hosoya et al., 2005}). However, experiments
and in-situ studies show that nucleation can be important---especially
at low temperatures and in coarse-grained or dry lithologies---and can
slow the net phase transition relative to a pure growth-limited
formulation (\citeproc{ref-kubo2004}{Kubo et al., 2004};
\citeproc{ref-perrillat2016}{Perrillat et al., 2016};
\citeproc{ref-rubie1994}{Rubie \& Ross II, 1994}). Recent in-situ X-ray
and acoustic experiments and microstructure studies
(\citeproc{ref-ledoux2023}{{Ledoux et al.}, 2023};
\citeproc{ref-ohuchi2022}{Ohuchi et al., 2022}) further document complex
nucleation/growth microstructures (including nanocrystalline, incoherent
products) that can limit effective phase transition rates under some
PT--deformation paths. These results indicate that our assumption of
saturated nucleation is an approximation that will tend to decrease
metastability and sharpen PTZs in cold slabs.

Assuming pure Mg-rich end-members neglects Fe-partitioning and
minor-element effects that can shift equilibrium depths by
\textasciitilde10--20 km and alter kinetics
(\citeproc{ref-katsura2004}{{Katsura et al.}, 2004};
\citeproc{ref-perrillat2016}{Perrillat et al., 2016}). Likewise, our
assumption of a relatively dry mantle neglects any potential sharpening
and/or shifting of the 410 km seismic discontinuity due to variable
water contents in olivine and wadsleyite (\citeproc{ref-chen2002}{Chen
et al., 2002}). Our simple temperature-dependent viscosity also omits
grain-size evolution, stress-dependent rheologies and anisotropic
deformation, all of which can modify strain localization and hence the
thermodynamic/kinetic feedbacks controlling PTZ structure
(\citeproc{ref-karato2001}{Karato et al., 2001}). These omissions should
be addressed in future higher-fidelity models---for now they imply that
the quantitative thresholds reported above are model-dependent and
should be interpreted as first-order effects.

\subsection{Implications for Subduction
Dynamics}\label{sec:implications-for-subduction-dynamics}

The quadratic relationship between PTZ width and displacement in slabs
(Figure \ref{fig:ptz-slabs}a) suggests a critical threshold near -50 km
displacement where thermodynamic forces overcome kinetic barriers. Above
this threshold---where kinetics are sufficiently sluggish---metastable
olivine persists extensively, consistent with deep earthquake
observations attributed to transformational faulting in metastable
wedges (\citeproc{ref-green1995}{Green \& Houston, 1995};
\citeproc{ref-ishii2021}{Ishii \& Ohtani, 2021};
\citeproc{ref-kirby1996}{Kirby et al., 1996};
\citeproc{ref-sindhusuta2025}{Sindhusuta et al., 2025}). Our results
indicate that \(\dot{X}_{\mathrm{max}}\) \textless{} 0.2 Ma\(^{-1}\)
produce extensive downward PTZ displacement that promote slab stagnation
at the 410. In contrast, \(\dot{X}_{\mathrm{max}}\) \textgreater{} 1
Ma\(^{-1}\) show small PTZ displacements that enable continuous
penetration into the lower mantle. Thus, in our models the transition
between a metastable, ponding regime and effective penetration occurs
over roughly one to two orders of magnitude in
\(\dot{X}_{\mathrm{max}}\) (Table \ref{tbl:centerline-profile-results}).

These mechanistic thresholds have observational relevance. Regions where
seismic tomography and receiver functions show slab flattening or
stagnation above the transition zone (e.g.,
\citeproc{ref-fukao2013}{Fukao \& Obayashi, 2013}) are compatible with
low effective phase transition rates and extensive metastability in our
models. At the same time, transformational faulting and deep earthquakes
in metastable wedges remain consistent with substantial metastable
volumes and sharp local stress concentrations produced during delayed
phase transitions (\citeproc{ref-ohuchi2022}{Ohuchi et al., 2022}).
Furthermore, the non-monotonic width-rate relationship in slabs (Figure
\ref{fig:ptz-slabs}b) implies that intermediate kinetics (0.05--0.15
Ma\(^{-1}\)) generate the broadest PTZs, potentially explaining the
variable thickness of the 410 discontinuity beneath different subduction
zones (\citeproc{ref-han2021}{Han et al., 2021};
\citeproc{ref-jiang2015}{Jiang et al., 2015}; \citeproc{ref-lee2014}{Lee
et al., 2014}; \citeproc{ref-shen2020}{Shen \& Zhan, 2020};
\citeproc{ref-vanstiphout2019}{Van Stiphout et al., 2019}). Our work
therefore demonstrates that kinetics provides an important control on
the expression of 410 seismic discontinuity, particularly where cold
slabs create conditions far from equilibrium.

\subsection{Seismic Structure and Discontinuity
Sharpness}\label{sec:seismic-structure-and-discontinuity-sharpness}

The correlation between \(\dot{X}_{\mathrm{max}}\) and PTZ structure has
direct implications for interpreting seismic discontinuities. Sharp
interfaces are most readily detected with SS precursors and
receiver-function stacks. Global SS/PP precursor studies emphasize
preferential sensitivity to very thin PTZs (a few km,
\citeproc{ref-shearer2000}{Shearer, 2000};
\citeproc{ref-chambers2005}{Chambers et al., 2005};
\citeproc{ref-deuss2009}{Deuss, 2009}), but actual resolution depends on
frequency content, signal-to-noise, and processing strategy.
High-frequency and regional receiver-function approaches demonstrate
that structures as thin as \textasciitilde5 km---and occasionally
thinner---can be resolved under favorable conditions
(\citeproc{ref-dokht2016}{Dokht et al., 2016};
\citeproc{ref-frazer2023}{Frazer \& Park, 2023};
\citeproc{ref-helffrich1996}{Helffrich \& Wood, 1996};
\citeproc{ref-wei2017}{Wei \& Shearer, 2017}).

In our models, thin and easily detectable discontinuities correspond to
\(\dot{X}_{\mathrm{max}}\) \textgreater{} 4 Ma\(^{-1}\) in plumes and
\textgreater{} 1 Ma\(^{-1}\) in slabs (Figures
\ref{fig:ptz-plumes}--\ref{fig:ptz-slabs}; Table
\ref{tbl:centerline-profile-results}). This implies that regions with
sharp, high-amplitude 410 signals may reflect either fast kinetics that
maintain reactions close to equilibrium or large effective displacements
across the PTZ. Observationally, sharp 410s are reported beneath several
plume regions, but sharpness is not uniquely diagnostic of kinetics:
composition, anisotropy, and imaging method also play important roles
(\citeproc{ref-deuss2009}{Deuss, 2009};
\citeproc{ref-lawrence2008}{Lawrence \& Shearer, 2008}).

Discontinuities observed near 500--600 km depth provide an additional
cautionary example. These features have been attributed to akimotoite
formation or other mid-transition-zone reactions
(\citeproc{ref-cottaar2016}{Cottaar \& Deuss, 2016}), and alternative
explanations include \(\beta \Leftrightarrow \gamma\) olivine
transitions or compositional heterogeneity
(\citeproc{ref-deuss2001}{Deuss \& Woodhouse, 2001};
\citeproc{ref-saikia2008}{Saikia et al., 2008};
\citeproc{ref-tauzin2017}{Tauzin et al., 2017}). Because the olivine
\(\alpha \Leftrightarrow \beta\) transition occurs much shallower
(\textasciitilde410 km), we do not interpret 500--600 km signals as
delayed wadsleyite formation. Conversely, broad or absent
discontinuities within the upper transition zone may indicate sluggish
kinetics, in addition to thermal or compositional influences.

The systematic upward displacement and broadening of the 410 produced by
sluggish kinetics in hot upwellings (Figure \ref{fig:ptz-plumes}) offers
a kinetic complement to purely thermal explanations for
reduced-amplitude or displaced discontinuities beneath hotspots. Global
SS and receiver-function compilations show reduced amplitude and complex
expressions of the 410 beneath many hotspots
(\citeproc{ref-deuss2009}{Deuss, 2009}), and high-resolution
receiver-function and precursor studies have reported topography and
thickness anomalies on scales of tens of kilometres
(\citeproc{ref-agius2017}{Agius et al., 2017};
\citeproc{ref-chambers2005}{Chambers et al., 2005};
\citeproc{ref-glasgow2024}{Glasgow et al., 2024};
\citeproc{ref-jenkins2016}{Jenkins et al., 2016}). Accounting for
kinetically controlled broadening and displacement (10--30 km in many of
our plume cases) reduces the remaining discrepancy between thermal
predictions and the observed total uplift or weakening.

Finally, observed lateral variations in apparent 410 thickness---ranging
from \textasciitilde5--30 km in some Pacific regions {[}Alex Song et al.
(\citeproc{ref-alex2004}{2004}); g@schmerr2007{]}---can be produced
either by thermal/compositional heterogeneity or by spatial changes in
effective kinetics (\(\dot{X}_{\mathrm{max}}\)) or both. Because our
model suite spans \textasciitilde2--3 orders of magnitude in
\(\dot{X}_{\mathrm{max}}\) (Table \ref{tbl:centerline-profile-results}),
the observed variability is broadly consistent with realistic variations
in both kinetics and background mantle properties.

\subsection{Constraining Kinetic Parameters from Seismic
Observations}\label{sec:constraining-kinetic-parameters-from-seismic-observations}

The distinct and relatively simple scaling laws we find for plumes
(linear width--displacement and power-law
width--\(\dot{X}_{\mathrm{max}}\) scalings; Figure \ref{fig:ptz-plumes})
provide a potential pathway to infer effective phase transition rates
from seismic observables, provided independent constraints on thermal
structure are available (e.g., from tomography or surface heat-flow). In
slabs the relation is more complex and thresholded (Figure
\ref{fig:ptz-slabs}), but the threshold behaviour itself---the depth and
abruptness of the catastrophic conversion---is a diagnostic that can
distinguish slow vs.~fast kinetic regimes. Operationally, two
complementary approaches are promising: thickness inversion coupled to
independent thermometry, and targeted regional tests.

High-resolution receiver-function or SS precursor maps
(\citeproc{ref-chambers2005}{Chambers et al., 2005};
\citeproc{ref-houser2010}{Houser \& Williams, 2010};
\citeproc{ref-lawrence2008}{Lawrence \& Shearer, 2008};
\citeproc{ref-schmerr2007}{Schmerr \& Garnero, 2007}) can produce
spatial maps of apparent PTZ thickness and displacement. Combining these
with tomographic temperature estimates and forward mapping from our
model scalings (Figures \ref{fig:ptz-plumes}--\ref{fig:ptz-slabs})
yields order-of-magnitude bounds on \(\dot{X}_{\mathrm{max}}\) and
discriminates growth mechanisms (grain-boundary vs.~dislocation control,
\citeproc{ref-hosoya2005}{Hosoya et al., 2005}). Mineral-physics
experiments that better quantify nucleation vs.~growth rates, the role
of water and Fe content, and microstructural inheritance
(\citeproc{ref-ledoux2023}{{Ledoux et al.}, 2023};
\citeproc{ref-ohuchi2022}{Ohuchi et al., 2022};
\citeproc{ref-perrillat2013}{Perrillat et al., 2013},
\citeproc{ref-perrillat2016}{2016}) are essential to shrink model
uncertainty and make seismic inversions for kinetics quantitative.
Combining such experiments with the forward model scalings we present
here is a roadmap to constrain effective mantle kinetics from seismic
data.

Moreover, regions where thermal structure is relatively constrained but
seismic expression varies laterally (for example, across slabs with
different ages or hydration states, or across hotspot swells vs.~ambient
mantle) are high-value tests. Systematic surveys that compare PTZ
thickness/topography in tectonically similar thermal settings but
different deformation histories (or water contents) can isolate kinetic
effects from purely thermal/compositional signals (e.g.,
\citeproc{ref-agius2017}{Agius et al., 2017};
\citeproc{ref-perrillat2022}{Perrillat et al., 2022};
\citeproc{ref-schmandt2012}{Schmandt, 2012};
\citeproc{ref-vanstiphout2019}{Van Stiphout et al., 2019}).

\cleardoublepage

\section{Conclusions}\label{sec:conclusions}

Seismic structure in Earth's upper mantle reflects a balance between
equilibrium thermodynamics and phase transition kinetics. This study set
out to quantify how these factors contribute to the expression of the
410 km seismic discontinuity (olivine \(\Leftrightarrow\) wadsleyite
transition) in plumes and slabs.

By coupling equilibrium thermodynamics to a growth-controlled kinetic
formulation within compressible mantle flow models, we systematically
explored the sensitivity of PTZ structure to kinetic rate parameters
across a broad experimental uncertainty range. Our plume simulations
reveal smooth, power-law scaling: sluggish kinetics broaden and uplift
the 410, while faster kinetics sharpen it to seismically detectable
thicknesses. Slab simulations, in contrast, display thresholded
dynamics: moderately slow kinetics yield metastable wedges, broad PTZs,
and slab stagnation at the 410 km PTZ, whereas both ultra-slow
(\(\dot{X}_{\mathrm{max}}\) \textless{} 0.02 Ma\(^{-1}\)) and ultra-fast
(\(\dot{X}_{\mathrm{max}}\) \textgreater{} 1 Ma\(^{-1}\)) kinetics
permit sharp seismic contrasts.

These results show that phase transition kinetics in compressibile
mantle flow can reproduce the observed diversity of 410 km discontinuity
sharpness and topography across the globe. They further indicate that
kinetic thresholds of \(\dot{X}_{\mathrm{max}}\) \textless{} 1
Ma\(^{-1}\) separate stagnant, metastable regimes from penetrating,
equilibrated ones. Such thresholds link mantle dynamics to effective
phase transition rates, suggesting that regional seismic observations
can be inverted---with supporting thermal constraints---to constrain
micro-scale kinetic parameters.

In summmary, our results demonstrate the utility of the 410 km
discontinuity as a potential seismological probe of kinetics in Earth's
upper mantle. Integrating seismic imaging, mineral physics, and forward
geodynamic models offers a path toward quantifying the role of kinetics
in mantle dynamics and in shaping the seismic expression of the MTZ.

\cleardoublepage

\section*{Acknowledgements}\label{acknowledgements}
\addcontentsline{toc}{section}{Acknowledgements}

This work was funded by the UKRI NERC Large Grant no. NE/V018477/1
awarded to John Wheeler at the University of Liverpool. All computations
were undertaken on Barkla2, part of the High Performance Computing
facilities at the University of Liverpool, who graciously provided
expert support. We thank the Computational Infrastructure for
Geodynamics (\url{geodynamics.org}) which is funded by the National
Science Foundation under award EAR-0949446 and EAR-1550901 for
supporting the development of ASPECT.

\section*{Data Availability}\label{data-availability}
\addcontentsline{toc}{section}{Data Availability}

All data, code, and relevant information for reproducing this work can
be found at
\url{https://github.com/buchanankerswell/kerswell_et_al_dynp}, and at
\ldots, the official Open Science Framework data repository. All code is
MIT Licensed and free for use and distribution (see license details).
ASPECT version 3.0.0, (\citeproc{ref-aspect-doi-v3.0.0}{Bangerth et al.,
2024a}, \citeproc{ref-aspectmanual}{2024b};
\citeproc{ref-clevenger2021}{Clevenger \& Heister, 2021};
\citeproc{ref-fraters2019}{{Fraters et al.}, 2019};
\citeproc{ref-fraters2020}{Fraters, 2020};
\citeproc{ref-gassmoller2018}{Gassmöller et al., 2018};
\citeproc{ref-heister2017}{Heister et al., 2017};
\citeproc{ref-kronbichler2012}{Kronbichler et al., 2012}) used in these
computations is freely available under the GPL v2.0 or later license
through its software landing page
\url{https://geodynamics.org/resources/aspect} or
\url{https://aspect.geodynamics.org} and is being actively developed on
GitHub and can be accessed via
\url{https://github.com/geodynamics/aspect}.

\cleardoublepage

\section*{References}\label{references}
\addcontentsline{toc}{section}{References}

\protect\phantomsection\label{refs}
\begin{CSLReferences}{1}{0}
\bibitem[\citeproctext]{ref-agius2017}
Agius, M., Rychert, C., Harmon, N., \& Laske, G. (2017). Mapping the
mantle transition zone beneath hawaii from ps receiver functions:
Evidence for a hot plume and cold mantle downwellings. \emph{Earth and
Planetary Science Letters}, \emph{474}, 226--236.

\bibitem[\citeproctext]{ref-alex2004}
Alex Song, T., Helmberger, D., \& Grand, S. (2004). Low-velocity zone
atop the 410-km seismic discontinuity in the northwestern united states.
\emph{Nature}, \emph{427}(6974), 530--533.

\bibitem[\citeproctext]{ref-aspect-doi-v3.0.0}
Bangerth, W., Dannberg, J., Fraters, M., Gassmöller, R., Glerum, A.,
Heister, T., et al. (2024a, December). ASPECT v3.0.0 (Version v3.0.0).
Zenodo. \url{https://doi.org/10.5281/zenodo.14371679}

\bibitem[\citeproctext]{ref-aspectmanual}
Bangerth, W., Dannberg, J., Fraters, M., Gassmöller, R., Glerum, A.,
Heister, T., et al. (2024b, December). {{ASPECT}: Advanced Solver for
Planetary Evolution, Convection, and Tectonics, User Manual}.
\url{https://doi.org/10.6084/m9.figshare.4865333}

\bibitem[\citeproctext]{ref-cahn1956}
Cahn, J. (1956). The kinetics of grain boundary nucleated reactions.
\emph{Acta Metallurgica}, \emph{4}(5), 449--459.

\bibitem[\citeproctext]{ref-chambers2005}
Chambers, K., Deuss, A., \& Woodhouse, J. (2005). Reflectivity of the
410-km discontinuity from PP and SS precursors. \emph{Journal of
Geophysical Research: Solid Earth}, \emph{110}(B2).

\bibitem[\citeproctext]{ref-chen2002}
Chen, J., Inoue, T., Yurimoto, H., \& Weidner, D. (2002). Effect of
water on olivine-wadsleyite phase boundary in the (mg, fe) 2SiO4 system.
\emph{Geophysical Research Letters}, \emph{29}(18), 22--1.

\bibitem[\citeproctext]{ref-clevenger2021}
Clevenger, T., \& Heister, T. (2021). Comparison between algebraic and
matrix-free geometric multigrid for a stokes problem on adaptive meshes
with variable viscosity. \emph{Numerical Linear Algebra with
Applications}, \emph{28}(5), e2375.

\bibitem[\citeproctext]{ref-connolly2009}
Connolly, J. (2009). The geodynamic equation of state: What and how.
\emph{Geochemistry, Geophysics, Geosystems}, \emph{10}(10).

\bibitem[\citeproctext]{ref-cottaar2016}
Cottaar, S., \& Deuss, A. (2016). Large-scale mantle discontinuity
topography beneath europe: Signature of akimotoite in subducting slabs.
\emph{Journal of Geophysical Research: Solid Earth}, \emph{121}(1),
279--292.

\bibitem[\citeproctext]{ref-cottaar2014}
Cottaar, S., Heister, T., Rose, I., \& Unterborn, C. (2014). BurnMan: A
lower mantle mineral physics toolkit. \emph{Geochemistry, Geophysics,
Geosystems}, \emph{15}(4), 1164--1179.

\bibitem[\citeproctext]{ref-curbelo2019}
Curbelo, J., Duarte, L., Alboussiere, T., Dubuffet, F., Labrosse, S., \&
Ricard, Y. (2019). Numerical solutions of compressible convection with
an infinite prandtl number: Comparison of the anelastic and anelastic
liquid models with the exact equations. \emph{Journal of Fluid
Mechanics}, \emph{873}, 646--687.

\bibitem[\citeproctext]{ref-deuss2009}
Deuss, A. (2009). Global observations of mantle discontinuities using SS
and PP precursors. \emph{Surveys in Geophysics}, \emph{30}(4), 301--326.

\bibitem[\citeproctext]{ref-deuss2001}
Deuss, A., \& Woodhouse, J. (2001). Seismic observations of splitting of
the mid-transition zone discontinuity in earth's mantle. \emph{Science},
\emph{294}(5541), 354--357.

\bibitem[\citeproctext]{ref-dokht2016}
Dokht, R., Gu, Y., \& Sacchi, M. (2016). Waveform inversion of SS
precursors: An investigation of the northwestern pacific subduction
zones and intraplate volcanoes in china. \emph{Gondwana Research},
\emph{40}, 77--90.

\bibitem[\citeproctext]{ref-faccenda2017}
Faccenda, M., \& Dal Zilio, L. (2017). The role of solid--solid phase
transitions in mantle convection. \emph{Lithos}, \emph{268}, 198--224.

\bibitem[\citeproctext]{ref-fraters2020}
Fraters, M. (2020, June). The geodynamic world builder (Version v0.3.0).
Zenodo. \url{https://doi.org/10.5281/zenodo.3900603}

\bibitem[\citeproctext]{ref-fraters2019}
{Fraters, M., Thieulot, C., van den Berg, A., \& Spakman, W.} (2019).
The geodynamic world builder: A solution for complex initial conditions
in numerical modeling. \emph{Solid Earth}, \emph{10}(5), 1785--1807.

\bibitem[\citeproctext]{ref-frazer2023}
Frazer, W., \& Park, J. (2023). High-resolution mid-mantle imaging with
multiple-taper SS-precursor estimates. \emph{Geophysical Journal
International}, \emph{233}(2), 1356--1371.

\bibitem[\citeproctext]{ref-fukao2013}
Fukao, Y., \& Obayashi, M. (2013). Subducted slabs stagnant above,
penetrating through, and trapped below the 660 km discontinuity.
\emph{Journal of Geophysical Research: Solid Earth}, \emph{118}(11),
5920--5938.

\bibitem[\citeproctext]{ref-fukao2009}
Fukao, Y., Obayashi, M., Nakakuki, T., \& Group, D. S. P. (2009).
Stagnant slab: A review. \emph{Annual Review of Earth and Planetary
Sciences}, \emph{37}(1), 19--46.

\bibitem[\citeproctext]{ref-gassmoller2018}
Gassmöller, R., Lokavarapu, H., Heien, E., Puckett, E., \& Bangerth, W.
(2018). Flexible and scalable particle-in-cell methods with adaptive
mesh refinement for geodynamic computations. \emph{Geochemistry,
Geophysics, Geosystems}, \emph{19}(9), 3596--3604.

\bibitem[\citeproctext]{ref-gassmoller2020}
Gassmöller, R., Dannberg, J., Bangerth, W., Heister, T., \& Myhill, R.
(2020). On formulations of compressible mantle convection.
\emph{Geophysical Journal International}, \emph{221}(2), 1264--1280.

\bibitem[\citeproctext]{ref-gerya2019}
Gerya, T. (2019). \emph{Introduction to numerical geodynamic modelling}.
Cambridge University Press.

\bibitem[\citeproctext]{ref-glasgow2024}
Glasgow, M., Zhang, H., Schmandt, B., Zhou, W., \& Zhang, J. (2024).
Global variability of the composition and temperature at the 410-km
discontinuity from receiver function analysis of dense arrays.
\emph{Earth and Planetary Science Letters}, \emph{643}, 118889.

\bibitem[\citeproctext]{ref-green1979}
Green, D., Jaques, L., \& Hibberson, W. (1979). Petrogenesis of
mid-ocean ridge basalts. In \emph{The earth: Its origin, structure and
evolution} (pp. 265--300). Academic Press.

\bibitem[\citeproctext]{ref-green1995}
Green, H., \& Houston, H. (1995). The mechanics of deep earthquakes.
\emph{Annual Review Of Earth And Planetary Sciences, Volume 23, Pp.
169-214.}, \emph{23}, 169--214.

\bibitem[\citeproctext]{ref-han2021}
Han, G., Li, J., Guo, G., Mooney, W., Karato, S., \& Yuen, D. (2021).
Pervasive low-velocity layer atop the 410-km discontinuity beneath the
northwest pacific subduction zone: Implications for rheology and
geodynamics. \emph{Earth and Planetary Science Letters}, \emph{554},
116642.

\bibitem[\citeproctext]{ref-heister2017}
Heister, T., Dannberg, J., Gassmöller, R., \& Bangerth, W. (2017). High
accuracy mantle convection simulation through modern numerical
methods--II: Realistic models and problems. \emph{Geophysical Journal
International}, \emph{210}(2), 833--851.

\bibitem[\citeproctext]{ref-helffrich1996}
Helffrich, G., \& Wood, B. (1996). 410 km discontinuity sharpness and
the form of the olivine \(\alpha\)-\(\beta\) phase diagram: Resolution
of apparent seismic contradictions. \emph{Geophysical Journal
International}, \emph{126}(2), F7--F12.

\bibitem[\citeproctext]{ref-hosoya2005}
Hosoya, T., Kubo, T., Ohtani, E., Sano, A., \& Funakoshi, K. (2005).
Water controls the fields of metastable olivine in cold subducting
slabs. \emph{Geophysical Research Letters}, \emph{32}(17).

\bibitem[\citeproctext]{ref-houser2010}
Houser, C., \& Williams, Q. (2010). Reconciling pacific 410 and 660 km
discontinuity topography, transition zone shear velocity patterns, and
mantle phase transitions. \emph{Earth and Planetary Science Letters},
\emph{296}(3-4), 255--266.

\bibitem[\citeproctext]{ref-ishii2021}
Ishii, T., \& Ohtani, E. (2021). Dry metastable olivine and slab
deformation in a wet subducting slab. \emph{Nature Geoscience},
\emph{14}(7), 526--530.

\bibitem[\citeproctext]{ref-jarvis1980}
Jarvis, G., \& Mckenzie, D. (1980). Convection in a compressible fluid
with infinite prandtl number. \emph{Journal of Fluid Mechanics},
\emph{96}(3), 515--583.

\bibitem[\citeproctext]{ref-jenkins2016}
Jenkins, J., Cottaar, S., White, R., \& Deuss, A. (2016). Depressed
mantle discontinuities beneath iceland: Evidence of a garnet controlled
660 km discontinuity? \emph{Earth and Planetary Science Letters},
\emph{433}, 159--168.

\bibitem[\citeproctext]{ref-jiang2015}
Jiang, G., Zhao, D., \& Zhang, G. (2015). Detection of metastable
olivine wedge in the western pacific slab and its geodynamic
implications. \emph{Physics of the Earth and Planetary Interiors},
\emph{238}, 1--7.

\bibitem[\citeproctext]{ref-karato2008}
Karato, S. (2008). Deformation of earth materials. \emph{An Introduction
to the Rheology of Solid Earth}, \emph{463}.

\bibitem[\citeproctext]{ref-karato2001}
Karato, S., Riedel, M., \& Yuen, D. (2001). Rheological structure and
deformation of subducted slabs in the mantle transition zone:
Implications for mantle circulation and deep earthquakes. \emph{Physics
of the Earth and Planetary Interiors}, \emph{127}(1-4), 83--108.

\bibitem[\citeproctext]{ref-katsura2004}
{Katsura, T., Yamada, H., Nishikawa, O., Song, M., Kubo, A., Shinmei,
T., et al.} (2004). Olivine-wadsleyite transition in the system (mg, fe)
2SiO4. \emph{Journal of Geophysical Research: Solid Earth},
\emph{109}(B2).

\bibitem[\citeproctext]{ref-kirby1996}
Kirby, S., Stein, S., Okal, E., \& Rubie, D. (1996). Metastable mantle
phase transformations and deep earthquakes in subducting oceanic
lithosphere. \emph{Reviews of Geophysics}, \emph{34}(2), 261--306.

\bibitem[\citeproctext]{ref-kronbichler2012}
Kronbichler, M., Heister, T., \& Bangerth, W. (2012). High accuracy
mantle convection simulation through modern numerical methods.
\emph{Geophysical Journal International}, \emph{191}(1), 12--29.

\bibitem[\citeproctext]{ref-kubo2004}
Kubo, T., Ohtani, E., \& Funakoshi, K. (2004). Nucleation and growth
kinetics of the \(\alpha\)-\(\beta\) transformation in Mg2SiO4determined
by in situ synchrotron powder x-ray diffraction. \emph{American
Mineralogist}, \emph{89}(2-3), 285--293.

\bibitem[\citeproctext]{ref-lawrence2008}
Lawrence, J., \& Shearer, P. (2008). Imaging mantle transition zone
thickness with SdS-SS finite-frequency sensitivity kernels.
\emph{Geophysical Journal International}, \emph{174}(1), 143--158.

\bibitem[\citeproctext]{ref-ledoux2023}
{Ledoux, E., Krug, M., Gay, J., Chantel, J., Hilairet, N., Bykov, M., et
al.} (2023). In-situ study of microstructures induced by the olivine to
wadsleyite transformation at conditions of the 410 km depth
discontinuity. \emph{American Mineralogist}, \emph{108}(12), 2283--2293.

\bibitem[\citeproctext]{ref-lee2014}
Lee, S., Rhie, J., Park, Y., \& Kim, K. (2014). Topography of the 410
and 660 km discontinuities beneath the korean peninsula and southwestern
japan using teleseismic receiver functions. \emph{Journal of Geophysical
Research: Solid Earth}, \emph{119}(9), 7245--7257.

\bibitem[\citeproctext]{ref-mohiuddin2018}
Mohiuddin, A., \& Karato, S. (2018). An experimental study of
grain-scale microstructure evolution during the olivine--wadsleyite
phase transition under nominally {``dry''} conditions. \emph{Earth and
Planetary Science Letters}, \emph{501}, 128--137.

\bibitem[\citeproctext]{ref-myhill2023}
Myhill, R., Cottaar, S., Heister, T., Rose, I., Unterborn, C., Dannberg,
J., \& Gassmöller, R. (2023). BurnMan--a python toolkit for planetary
geophysics, geochemistry and thermodynamics.

\bibitem[\citeproctext]{ref-ohuchi2022}
Ohuchi, T., Higo, Y., Tange, Y., Sakai, T., Matsuda, K., \& Irifune, T.
(2022). In situ x-ray and acoustic observations of deep seismic faulting
upon phase transitions in olivine. \emph{Nature Communications},
\emph{13}(1), 5213.

\bibitem[\citeproctext]{ref-perrillat2013}
Perrillat, J., Daniel, I., Bolfan-Casanova, N., Chollet, M., Morard, G.,
\& Mezouar, M. (2013). Mechanism and kinetics of the
\(\alpha\)--\(\beta\) transition in san carlos olivine Mg1. 8Fe0. 2SiO4.
\emph{Journal of Geophysical Research: Solid Earth}, \emph{118}(1),
110--119.

\bibitem[\citeproctext]{ref-perrillat2016}
Perrillat, J., Chollet, M., Durand, S., De Moortele, B. van, Chambat,
F., Mezouar, M., \& Daniel, I. (2016). Kinetics of the
olivine--ringwoodite transformation and seismic attenuation in the
earth's mantle transition zone. \emph{Earth and Planetary Science
Letters}, \emph{433}, 360--369.

\bibitem[\citeproctext]{ref-perrillat2022}
Perrillat, J., Tauzin, B., Chantel, J., Jonfal, J., Daniel, I., Jing,
Z., \& Wang, Y. (2022). Shear wave velocities across the
olivine--wadsleyite--ringwoodite transitions and sharpness of the 410 km
seismic discontinuity. \emph{Earth and Planetary Science Letters},
\emph{593}, 117690.

\bibitem[\citeproctext]{ref-ranalli1995}
Ranalli, G. (1995). \emph{Rheology of the earth}. Springer Science \&
Business Media.

\bibitem[\citeproctext]{ref-ringwood1975}
Ringwood, A. (1975). Composition and petrology of the earth's mantle.
\emph{MacGraw-Hill}, \emph{618}.

\bibitem[\citeproctext]{ref-rubie1994}
Rubie, D., \& Ross II, C. (1994). Kinetics of the olivine-spinel
transformation in subducting lithosphere: Experimental constraints and
implications for deep slab processes. \emph{Physics of the Earth and
Planetary Interiors}, \emph{86}(1-3), 223--243.

\bibitem[\citeproctext]{ref-saikia2008}
Saikia, A., Frost, D., \& Rubie, D. (2008). Splitting of the
520-kilometer seismic discontinuity and chemical heterogeneity in the
mantle. \emph{Science}, \emph{319}(5869), 1515--1518.

\bibitem[\citeproctext]{ref-schmandt2012}
Schmandt, B. (2012). Mantle transition zone shear velocity gradients
beneath USArray. \emph{Earth and Planetary Science Letters}, \emph{355},
119--130.

\bibitem[\citeproctext]{ref-schmerr2007}
Schmerr, N., \& Garnero, E. (2007). Upper mantle discontinuity
topography from thermal and chemical heterogeneity. \emph{Science},
\emph{318}(5850), 623--626.

\bibitem[\citeproctext]{ref-schubert2001}
Schubert, G., Turcotte, D., \& Olson, P. (2001). \emph{Mantle convection
in the earth and planets}. Cambridge University Press.

\bibitem[\citeproctext]{ref-shearer2000}
Shearer, P. (2000). Upper mantle seismic discontinuities.
\emph{Geophysical Monograph-American Geophysical Union}, \emph{117},
115--132.

\bibitem[\citeproctext]{ref-shen2020}
Shen, Z., \& Zhan, Z. (2020). Metastable olivine wedge beneath the japan
sea imaged by seismic interferometry. \emph{Geophysical Research
Letters}, \emph{47}(6), e2019GL085665.

\bibitem[\citeproctext]{ref-sindhusuta2025}
Sindhusuta, S., Chi, S., Foster, C., Officer, T., \& Wang, Y. (2025).
Numerical investigation into mechanical behavior of metastable olivine
during phase transformation: Implications for deep-focus earthquakes.
\emph{Journal of Geophysical Research: Solid Earth}, \emph{130}(2),
e2024JB030557.

\bibitem[\citeproctext]{ref-stixrude2022}
Stixrude, L., \& Lithgow-Bertelloni, C. (2022). Thermal expansivity,
heat capacity and bulk modulus of the mantle. \emph{Geophysical Journal
International}, \emph{228}(2), 1119--1149.

\bibitem[\citeproctext]{ref-tauzin2017}
Tauzin, B., Kim, S., \& Kennett, B. (2017). Pervasive seismic
low-velocity zones within stagnant plates in the mantle transition zone:
Thermal or compositional origin? \emph{Earth and Planetary Science
Letters}, \emph{477}, 1--13.

\bibitem[\citeproctext]{ref-turnbull1956}
Turnbull, D. (1956). Phase changes. In \emph{Solid state physics} (Vol.
3, pp. 225--306). Elsevier.

\bibitem[\citeproctext]{ref-vanstiphout2019}
Van Stiphout, A., Cottaar, S., \& Deuss, A. (2019). Receiver function
mapping of mantle transition zone discontinuities beneath alaska using
scaled 3-d velocity corrections. \emph{Geophysical Journal
International}, \emph{219}(2), 1432--1446.

\bibitem[\citeproctext]{ref-wei2017}
Wei, S., \& Shearer, P. (2017). A sporadic low-velocity layer atop the
410 km discontinuity beneath the pacific ocean. \emph{Journal of
Geophysical Research: Solid Earth}, \emph{122}(7), 5144--5159.

\end{CSLReferences}

\cleardoublepage

\section*{Appendix}\label{appendix}
\addcontentsline{toc}{section}{Appendix}

\subsection*{Stress, Pressure, and The Momentum
Equation}\label{sec:momentum-derivation}
\addcontentsline{toc}{subsection}{Stress, Pressure, and The Momentum
Equation}

The momentum equation in the following form is referred to as the
Navier-Stokes equation and describes the flow of a compressible viscous
fluid primarily by buoyancy forces:

\begin{equation}
  \rho \left(\frac{\partial \vec{u}}{\partial t} \right) = \nabla \cdot \sigma^{\prime} - \nabla P + \rho g
  \label{eq:navier-stokes-compressible}
\end{equation}

where \(\rho\) is density, \(\vec{u}\) is velocity, \(P\) is pressure,
\(\sigma^{\prime}\) is the deviatoric stress tensor, and \(g\) is
gravitational acceleration. In terms of classical mechanics, the
left-hand side is analogous to mass times acceleration \(ma\), and the
right-hand side are the forces \(F\) that are acting on the fluid.
Hence, the equation describes a balance between force and momentum
\(ma = F\) (\citeproc{ref-gerya2019}{Gerya, 2019}).

The forces in Equation \ref{eq:navier-stokes-compressible} include the
pressure gradient \(\nabla P\) which acts to drive the fluid from high
pressure to low pressure, the viscous forces
\(\nabla \cdot \sigma^{\prime}\) that dissipate energy by resisting
flow, and the buoyancy forces \(\rho g\) that drive convection (denser
fluids sink while lighter fluids rise). Because the flow of Earth's
mantle occurs at such slow rates, however, the inertial term
\(\left(\frac{\partial \vec{u}}{\partial t} \right)\) on the left-hand
side of Equation \ref{eq:navier-stokes-compressible} can be ignored, and
the momentum equation simplifies to:

\begin{equation}
  \nabla P - \nabla \cdot \sigma^{\prime} = \rho g
  \label{eq:navier-stokes-no-inertia-appendix}
\end{equation}

Equation \ref{eq:navier-stokes-no-inertia-appendix} describes a balance
between the buoyancy force and the pressure gradient minus the energy
dissipation due to deformation. Since the deviatoric stress tensor
\(\sigma^{\prime}\) can be described in terms of velocity (see Equation
\ref{eq:stress-deviatoric-component}), the primary unknowns in Equation
\ref{eq:navier-stokes-compressible} are pressure and velocity.

The complete stress tensor can be written as:

\begin{equation}
  \sigma_{ij} = \sigma^{\prime}_{ij} - P \delta_{ij}
  \label{eq:stress-complete}
\end{equation}

where \(\sigma_{ij}\) is total stress, \(\sigma^{\prime}_{ij}\) is the
deviatoric (non-hydrostatic) component of stress,
\(P = - \frac{\sigma_{xx} + \sigma_{yy} + \sigma_{zz}}{3}\) is the
hydrostatic component of stress, and \(\delta_{ij}\) is the Kroneker
delta:

\begin{equation}
  \delta_{ij} =
  \begin{cases}
    1, & \text{if} i = j \\
    0, & \text{if} i \neq j
  \end{cases}
\end{equation}

The hydrostatic component of stress acts equally in all directions and
therefore affects the fluid's volume (density) but does not change its
shape or cause it to flow. Note that the negative sign in Equation
\ref{eq:stress-complete} implies that pressure is positive under
compression (negative normal stress). This is a convention used in
geodynamics that differs from material sciences and other fields.

The deviatoric part of the stress tensor is responsible for deformation
and flow of the fluid and is equal to the total stress without the
hydrostatic stress component,
\(\sigma^{\prime}_{ij} = \sigma_{ij} + P \delta_{ij}\), or in full
matrix form:

\begin{equation}
  \begin{pmatrix}
  \sigma^{\prime}_{xx} & \sigma^{\prime}_{xy} & \sigma^{\prime}_{xz} \\
  \sigma^{\prime}_{yx} & \sigma^{\prime}_{yy} & \sigma^{\prime}_{yz} \\
  \sigma^{\prime}_{zx} & \sigma^{\prime}_{zy} & \sigma^{\prime}_{zz}
  \end{pmatrix} =
  \begin{pmatrix}
  \sigma_{xx} & \sigma_{xy} & \sigma_{xz} \\
  \sigma_{yx} & \sigma_{yy} & \sigma_{yz} \\
  \sigma_{zx} & \sigma_{zy} & \sigma_{zz}
  \end{pmatrix} +
  \begin{pmatrix}
  -\frac{\sigma_{xx} + \sigma_{yy} + \sigma_{zz}}{3} & 0 & 0 \\
  0 & -\frac{\sigma_{xx} + \sigma_{yy} + \sigma_{zz}}{3} & 0 \\
  0 & 0 & -\frac{\sigma_{xx} + \sigma_{yy} + \sigma_{zz}}{3}
  \end{pmatrix}
  \label{eq:stress-deviatoric}
\end{equation}

In practice, the deviatoric stress tensor \(\sigma^{\prime}\) is
computed by applying a constitutive relationship between stress and
strain to express \(\sigma^{\prime}\) in terms of velocity. In the
present case, we apply a generalized linear model that combines shear
deformation without rotation and volumetric deformation (dilation):

\begin{equation}
  \sigma^{\prime} = \eta \left(\nabla \vec{u} + \left(\nabla \vec{u} \right)^\intercal \right) - \left(\frac{2}{3} \eta - \zeta \right) \left(\nabla \cdot \vec{u} \right) I
\end{equation}

where \(\vec{u}\) is velocity, \(\eta\) is shear viscosity, and
\(\zeta\) is bulk viscosity. For nearly-incompressible fluids (very
small \(\zeta\)), the expression for deviatoric stress simplifies to:

\begin{equation}
  \sigma^{\prime} = 2 \eta \dot{\epsilon}^{\prime}
\end{equation}

where
\(\dot{\epsilon}^{\prime} = \frac{1}{2} \left(\nabla \vec{u} + \left(\nabla \vec{u} \right)^\intercal \right) - \frac{1}{3} \left(\nabla \cdot \vec{u} \right) I\)
is the deviatoric strain rate tensor. In full component form the
deviatoric stress tensor is:

\begin{equation}
  \sigma^{\prime}_{ij} = \eta \left(\frac{\partial u_i}{\partial x_j} + \frac{\partial u_j}{\partial x_i} \right) - \frac{2}{3} \eta \left(\frac{\partial u_x}{\partial x} + \frac{\partial u_y}{\partial y} + \frac{\partial u_z}{\partial z} \right) \delta_{ij}
  \label{eq:stress-deviatoric-component}
\end{equation}

Note that the deviatoric stress and strain rate tensors are symmetric
such that \(\sigma^{\prime}_{ij} = \sigma^{\prime}_{ji}\) and
\(\dot{\epsilon}^{\prime}_{ij} = \dot{\epsilon}^{\prime}_{ji}\), which
implies that there is zero rigid-body rotation in the fluid flow.
Because of this symmetry, the full matrix form the deviatoric stress
tensor can be written as:

\begin{equation}
  \sigma^{\prime} = \begin{pmatrix}
  2 \eta \frac{\partial u_x}{\partial x} - \frac{2}{3} \eta \left(\frac{\partial u_x}{\partial x} + \frac{\partial u_y}{\partial y} + \frac{\partial u_z}{\partial z} \right) & \eta \left(\frac{\partial u_x}{\partial y} + \frac{\partial u_y}{\partial x} \right) & \eta \left(\frac{\partial u_x}{\partial z} + \frac{\partial u_z}{\partial x} \right) \\
  \eta \left(\frac{\partial u_y}{\partial x} + \frac{\partial u_x}{\partial y} \right) & 2 \eta \frac{\partial u_y}{\partial y} - \frac{2}{3} \eta \left(\frac{\partial u_x}{\partial x} + \frac{\partial u_y}{\partial y} + \frac{\partial u_z}{\partial z} \right) & \eta \left(\frac{\partial u_y}{\partial z} + \frac{\partial u_z}{\partial y} \right) \\
  \eta \left(\frac{\partial u_z}{\partial x} + \frac{\partial u_x}{\partial z} \right) & \eta \left(\frac{\partial u_z}{\partial y} + \frac{\partial u_y}{\partial z} \right) & 2 \eta \frac{\partial u_z}{\partial z} - \frac{2}{3} \eta \left(\frac{\partial u_x}{\partial x} + \frac{\partial u_y}{\partial y} + \frac{\partial u_z}{\partial z} \right)
  \end{pmatrix}
\end{equation}

It is often useful to visualize the \emph{second invariant} of the
deviatoric stress tensor, which is independent of the coordinate
reference frame and quantifies the local deviation of stress from a
hydrostatic (non-convecting) state:

\begin{equation}
  \sigma_{\text{II}} = \sqrt{\frac{1}{2} \left(\text{tr}(\sigma^{\prime 2}) - \text{tr}(\sigma^{\prime})^2 \right)}
  \label{eq:second-invariant-definition}
\end{equation}

where \(\text{tr}(\sigma^{\prime 2}) = \Sigma \sigma^{\prime 2}_{ij}\)
and
\(\text{tr}(\sigma^{\prime})^2 = (\sigma^{\prime}_{xx} + \sigma^{\prime}_{yy} + \sigma^{\prime}_{zz})^2\).
Note that Equation \ref{eq:second-invariant-definition} uses the
convention that compressive stress is positive. It follows from Equation
\ref{eq:stress-complete} that the normal deviatoric stresses are:

\begin{equation}
  \begin{aligned}
    \sigma^{\prime}_{xx} &= \sigma_{xx} + P \\
    \sigma^{\prime}_{yy} &= \sigma_{yy} + P \\
    \sigma^{\prime}_{zz} &= \sigma_{zz} + P
  \end{aligned}
\end{equation}

and thus by definition
\(\text{tr}(\sigma^{\prime}) = \text{tr}(\sigma) + 3P = 0\), since
\(\text{tr}(\sigma) = -3P\). By this definition, Equation
\ref{eq:second-invariant-definition} can be written as:

\begin{equation}
  \begin{aligned}
    \sigma_{\text{II}} &= \sqrt{\frac{1}{2} \left(\text{tr}(\sigma^{\prime 2}) - 0 \right)} = \sqrt{\frac{1}{2} \text{tr}(\sigma^{\prime 2})} = \sqrt{\frac{1}{2} \sum_{i, j}\sigma^{\prime 2}_{ij}} \\
    \sigma_{\text{II}} &= \sqrt{\frac{1}{2} (\sigma^{\prime 2}_{xx} + \sigma^{\prime 2}_{yy} + \sigma^{\prime 2}_{zz} + \sigma^{\prime 2}_{xy} + \sigma^{\prime 2}_{yx} + \sigma^{\prime 2}_{xz} + \sigma^{\prime 2}_{zx} + \sigma^{\prime 2}_{yz} + \sigma^{\prime 2}_{zy})} \\
    \sigma_{\text{II}} &= \sqrt{\frac{1}{2} (\sigma^{\prime 2}_{xx} + \sigma^{\prime 2}_{yy} + \sigma^{\prime 2}_{zz}) + \sigma^{\prime 2}_{xy} + \sigma^{\prime 2}_{xz} + \sigma^{\prime 2}_{yz}}
  \end{aligned}
\end{equation}

Note also that many engineering applications use the von Mises stress:

\begin{equation}
  \sigma_{\text{vm}} = \sqrt{\frac{3}{2} \sum_{i, j}\sigma^{\prime 2}_{ij}}
\end{equation}

which is proportional to the second invariant of the deviatoric stress
tensor by a factor of \(\sqrt{3}\):

\begin{equation}
  \sigma_{\text{vm}} = \sqrt{3} \, \sigma_{\text{II}}
\end{equation}

\cleardoublepage

\subsection*{Simulation Snapshots: Slabs and Plumes
Continued}\label{sec:simulation-snapshots-continued}
\addcontentsline{toc}{subsection}{Simulation Snapshots: Slabs and Plumes
Continued}

\begin{figure}
\centering
\pandocbounded{\includegraphics[keepaspectratio,alt={Slab model with sluggish kinetics (slab-lnk18-Ha300-d5mm-060ppm) after 100 Ma evolution. Panels show dynamic temperature \textbackslash hat\{T\} (a), dynamic pressure \textbackslash hat\{P\} (b), dynamic density \textbackslash hat\{\textbackslash rho\} (c), thermodynamic term \textbackslash left(1 - \textbackslash left{[}\textbackslash Delta G/R\textbackslash,T\textbackslash right{]}\textbackslash right) (d), growth rate \textbackslash dot\{x\} (e), phase transition rate \textbackslash dot\{X\} (f), volume fraction of wadsleyite X (g), pressure-wave velocity V\_p (h), and shear-wave velocity V\_s (i).}]{../figs/simulation/2d_box/compositions/slab-lnk18-Ha300-d5mm-060ppm-full-set-composition-0100.png}}
\caption{Slab model with sluggish kinetics
(slab-lnk18-Ha300-d5mm-060ppm) after 100 Ma evolution. Panels show
dynamic temperature \(\hat{T}\) (a), dynamic pressure \(\hat{P}\) (b),
dynamic density \(\hat{\rho}\) (c), thermodynamic term
\(\left(1 - \left[\Delta G/R\,T\right]\right)\) (d), growth rate
\(\dot{x}\) (e), phase transition rate \(\dot{X}\) (f), volume fraction
of wadsleyite \(X\) (g), pressure-wave velocity \(V_p\) (h), and
shear-wave velocity \(V_s\) (i).}\label{fig:slab-comp-slow}
\end{figure}

\begin{figure}
\centering
\pandocbounded{\includegraphics[keepaspectratio,alt={Slab model with moderately-sluggish kinetics (slab-lnk18-Ha274-d5mm-060ppm) after 100 Ma evolution. Panels show dynamic temperature \textbackslash hat\{T\} (a), dynamic pressure \textbackslash hat\{P\} (b), dynamic density \textbackslash hat\{\textbackslash rho\} (c), thermodynamic term \textbackslash left(1 - \textbackslash left{[}\textbackslash Delta G/R\textbackslash,T\textbackslash right{]}\textbackslash right) (d), growth rate \textbackslash dot\{x\} (e), phase transition rate \textbackslash dot\{X\} (f), volume fraction of wadsleyite X (g), pressure-wave velocity V\_p (h), and shear-wave velocity V\_s (i).}]{../figs/simulation/2d_box/compositions/slab-lnk18-Ha274-d5mm-060ppm-full-set-composition-0100.png}}
\caption{Slab model with moderately-sluggish kinetics
(slab-lnk18-Ha274-d5mm-060ppm) after 100 Ma evolution. Panels show
dynamic temperature \(\hat{T}\) (a), dynamic pressure \(\hat{P}\) (b),
dynamic density \(\hat{\rho}\) (c), thermodynamic term
\(\left(1 - \left[\Delta G/R\,T\right]\right)\) (d), growth rate
\(\dot{x}\) (e), phase transition rate \(\dot{X}\) (f), volume fraction
of wadsleyite \(X\) (g), pressure-wave velocity \(V_p\) (h), and
shear-wave velocity \(V_s\) (i).}\label{fig:slab-comp-moderately-slow}
\end{figure}

\begin{figure}
\centering
\pandocbounded{\includegraphics[keepaspectratio,alt={Slab model with fast kinetics (slab-lnk18-Ha274-D1e12-060ppm) after 100 Ma evolution. Panels show dynamic temperature \textbackslash hat\{T\} (a), dynamic pressure \textbackslash hat\{P\} (b), dynamic density \textbackslash hat\{\textbackslash rho\} (c), thermodynamic term \textbackslash left(1 - \textbackslash left{[}\textbackslash Delta G/R\textbackslash,T\textbackslash right{]}\textbackslash right) (d), growth rate \textbackslash dot\{x\} (e), phase transition rate \textbackslash dot\{X\} (f), volume fraction of wadsleyite X (g), pressure-wave velocity V\_p (h), and shear-wave velocity V\_s (i).}]{../figs/simulation/2d_box/compositions/slab-lnk18-Ha274-D1e12-060ppm-full-set-composition-0100.png}}
\caption{Slab model with fast kinetics (slab-lnk18-Ha274-D1e12-060ppm)
after 100 Ma evolution. Panels show dynamic temperature \(\hat{T}\) (a),
dynamic pressure \(\hat{P}\) (b), dynamic density \(\hat{\rho}\) (c),
thermodynamic term \(\left(1 - \left[\Delta G/R\,T\right]\right)\) (d),
growth rate \(\dot{x}\) (e), phase transition rate \(\dot{X}\) (f),
volume fraction of wadsleyite \(X\) (g), pressure-wave velocity \(V_p\)
(h), and shear-wave velocity \(V_s\) (i).}\label{fig:slab-comp-fast}
\end{figure}

\begin{figure}
\centering
\pandocbounded{\includegraphics[keepaspectratio,alt={Plume model with sluggish kinetics (plume-lnk18-Ha300-d5mm-060ppm) after 100 Ma evolution. Panels show dynamic temperature \textbackslash hat\{T\} (a), dynamic pressure \textbackslash hat\{P\} (b), dynamic density \textbackslash hat\{\textbackslash rho\} (c), thermodynamic term \textbackslash left(1 - \textbackslash left{[}\textbackslash Delta G/R\textbackslash,T\textbackslash right{]}\textbackslash right) (d), growth rate \textbackslash dot\{x\} (e), phase transition rate \textbackslash dot\{X\} (f), volume fraction of olivine X (g), pressure-wave velocity V\_p (h), and shear-wave velocity V\_s (i).}]{../figs/simulation/2d_box/compositions/plume-lnk18-Ha300-d5mm-060ppm-full-set-composition-0100.png}}
\caption{Plume model with sluggish kinetics
(plume-lnk18-Ha300-d5mm-060ppm) after 100 Ma evolution. Panels show
dynamic temperature \(\hat{T}\) (a), dynamic pressure \(\hat{P}\) (b),
dynamic density \(\hat{\rho}\) (c), thermodynamic term
\(\left(1 - \left[\Delta G/R\,T\right]\right)\) (d), growth rate
\(\dot{x}\) (e), phase transition rate \(\dot{X}\) (f), volume fraction
of olivine \(X\) (g), pressure-wave velocity \(V_p\) (h), and shear-wave
velocity \(V_s\) (i).}\label{fig:plume-comp-slow}
\end{figure}

\begin{figure}
\centering
\pandocbounded{\includegraphics[keepaspectratio,alt={Plume model with moderately-sluggish kinetics (plume-lnk18-Ha274-d5mm-060ppm) after 100 Ma evolution. Panels show dynamic temperature \textbackslash hat\{T\} (a), dynamic pressure \textbackslash hat\{P\} (b), dynamic density \textbackslash hat\{\textbackslash rho\} (c), thermodynamic term \textbackslash left(1 - \textbackslash left{[}\textbackslash Delta G/R\textbackslash,T\textbackslash right{]}\textbackslash right) (d), growth rate \textbackslash dot\{x\} (e), phase transition rate \textbackslash dot\{X\} (f), volume fraction of olivine X (g), pressure-wave velocity V\_p (h), and shear-wave velocity V\_s (i).}]{../figs/simulation/2d_box/compositions/plume-lnk18-Ha274-d5mm-060ppm-full-set-composition-0100.png}}
\caption{Plume model with moderately-sluggish kinetics
(plume-lnk18-Ha274-d5mm-060ppm) after 100 Ma evolution. Panels show
dynamic temperature \(\hat{T}\) (a), dynamic pressure \(\hat{P}\) (b),
dynamic density \(\hat{\rho}\) (c), thermodynamic term
\(\left(1 - \left[\Delta G/R\,T\right]\right)\) (d), growth rate
\(\dot{x}\) (e), phase transition rate \(\dot{X}\) (f), volume fraction
of olivine \(X\) (g), pressure-wave velocity \(V_p\) (h), and shear-wave
velocity \(V_s\) (i).}\label{fig:plume-comp-moderately-slow}
\end{figure}

\begin{figure}
\centering
\pandocbounded{\includegraphics[keepaspectratio,alt={Plume model with fast kinetics (plume-lnk18-Ha274-D1e12-060ppm) after 100 Ma evolution. Panels show dynamic temperature \textbackslash hat\{T\} (a), dynamic pressure \textbackslash hat\{P\} (b), dynamic density \textbackslash hat\{\textbackslash rho\} (c), thermodynamic term \textbackslash left(1 - \textbackslash left{[}\textbackslash Delta G/R\textbackslash,T\textbackslash right{]}\textbackslash right) (d), growth rate \textbackslash dot\{x\} (e), phase transition rate \textbackslash dot\{X\} (f), volume fraction of olivine X (g), pressure-wave velocity V\_p (h), and shear-wave velocity V\_s (i).}]{../figs/simulation/2d_box/compositions/plume-lnk18-Ha274-D1e12-060ppm-full-set-composition-0100.png}}
\caption{Plume model with fast kinetics (plume-lnk18-Ha274-D1e12-060ppm)
after 100 Ma evolution. Panels show dynamic temperature \(\hat{T}\) (a),
dynamic pressure \(\hat{P}\) (b), dynamic density \(\hat{\rho}\) (c),
thermodynamic term \(\left(1 - \left[\Delta G/R\,T\right]\right)\) (d),
growth rate \(\dot{x}\) (e), phase transition rate \(\dot{X}\) (f),
volume fraction of olivine \(X\) (g), pressure-wave velocity \(V_p\)
(h), and shear-wave velocity \(V_s\) (i).}\label{fig:plume-comp-fast}
\end{figure}

\cleardoublepage

\subsection*{Phase Transition Zone: Displacement and Width
Continued}\label{sec:ptz-displacement-width-continued}
\addcontentsline{toc}{subsection}{Phase Transition Zone: Displacement
and Width Continued}

\begin{longtable}[]{@{}lrrr@{}}
\caption{\label{tbl:centerline-profile-results}PTZ displacement, width,
and maximum phase transition rate \(\dot{X}_{\mathrm{max}}\) evaluated
in plume and slab simulations after 100 Ma of evolution. Units are PTZ
displacement: km, PTZ width: km, \(\dot{X}_{\mathrm{max}}\):
Ma\(^{-1}\).}\label{tbl:centerline-profile-results}\tabularnewline
\toprule\noalign{}
Model ID & PTZ Displacement & PTZ Width & \(\dot{X}_{\mathrm{max}}\) \\
\midrule\noalign{}
\endfirsthead
\toprule\noalign{}
Model ID & PTZ Displacement & PTZ Width & \(\dot{X}_{\mathrm{max}}\) \\
\midrule\noalign{}
\endhead
\bottomrule\noalign{}
\endlastfoot
plume-lnk21-Ha274-d5mm-060ppm & 75.86 & -73.69 & 0.231 \\
plume-lnk21-Ha274-d5mm-060ppm & 75.86 & -73.69 & 0.231 \\
plume-lnk20-Ha274-d5mm-060ppm & 41.25 & -46.17 & 0.424 \\
plume-lnk20-Ha274-d5mm-060ppm & 41.25 & -46.17 & 0.424 \\
plume-lnk18-Ha300-d5mm-060ppm & 30.94 & -38.01 & 0.538 \\
plume-lnk18-Ha300-d5mm-060ppm & 30.94 & -38.01 & 0.538 \\
plume-lnk19-Ha274-d5mm-060ppm & 18.07 & -27.67 & 0.751 \\
plume-lnk19-Ha274-d5mm-060ppm & 18.07 & -27.67 & 0.751 \\
plume-lnk18-Ha274-d1cm-060ppm & 12.95 & -23.73 & 0.889 \\
plume-lnk18-Ha274-d1cm-060ppm & 12.95 & -23.73 & 0.889 \\
plume-lnk18-Ha274-d5mm-060ppm & 4.12 & -16.50 & 1.292 \\
plume-lnk18-Ha274-d5mm-060ppm & 4.12 & -16.50 & 1.292 \\
plume-lnk18-Ha274-D1e06-060ppm & 0.65 & -14.05 & 1.604 \\
plume-lnk18-Ha274-D1e06-060ppm & 0.65 & -14.05 & 1.604 \\
plume-lnk18-Ha274-d5mm-080ppm & -3.09 & -11.16 & 2.095 \\
plume-lnk18-Ha274-d5mm-080ppm & -3.09 & -11.16 & 2.095 \\
plume-lnk17-Ha274-d5mm-060ppm & -4.12 & -10.25 & 2.188 \\
plume-lnk17-Ha274-d5mm-060ppm & -4.12 & -10.25 & 2.188 \\
plume-lnk18-Ha274-d1mm-060ppm & -7.22 & -7.83 & 2.973 \\
plume-lnk18-Ha274-d1mm-060ppm & -7.22 & -7.83 & 2.973 \\
plume-lnk18-Ha274-d5mm-100ppm & -7.22 & -7.84 & 3.017 \\
plume-lnk18-Ha274-d5mm-100ppm & -7.22 & -7.84 & 3.017 \\
plume-lnk16-Ha274-d5mm-060ppm & -8.46 & -6.83 & 3.623 \\
plume-lnk16-Ha274-d5mm-060ppm & -8.46 & -6.83 & 3.623 \\
plume-lnk18-Ha274-d5mm-120ppm & -9.39 & -6.08 & 4.006 \\
plume-lnk18-Ha274-d5mm-120ppm & -9.39 & -6.08 & 4.006 \\
plume-lnk18-Ha274-D1e08-060ppm & -11.17 & -4.52 & 5.146 \\
plume-lnk18-Ha274-D1e08-060ppm & -11.17 & -4.52 & 5.146 \\
plume-lnk18-Ha274-d5mm-140ppm & -11.19 & -4.51 & 5.176 \\
plume-lnk18-Ha274-d5mm-140ppm & -11.19 & -4.51 & 5.176 \\
plume-lnk15-Ha274-d5mm-060ppm & -11.34 & -4.12 & 5.859 \\
plume-lnk15-Ha274-d5mm-060ppm & -11.34 & -4.12 & 5.859 \\
plume-lnk18-Ha274-D1e10-060ppm & -14.44 & -2.06 & 14.335 \\
plume-lnk18-Ha274-D1e10-060ppm & -14.44 & -2.06 & 14.335 \\
plume-lnk18-Ha274-D1e12-060ppm & -15.69 & -2.87 & 31.381 \\
plume-lnk18-Ha274-D1e12-060ppm & -15.69 & -2.87 & 31.381 \\
slab-lnk21-Ha274-d5mm-060ppm & -109.82 & 37.55 & 0.012 \\
slab-lnk21-Ha274-d5mm-060ppm & -109.82 & 37.55 & 0.012 \\
slab-lnk20-Ha274-d5mm-060ppm & -86.41 & 32.46 & 0.019 \\
slab-lnk20-Ha274-d5mm-060ppm & -86.41 & 32.46 & 0.019 \\
slab-lnk18-Ha300-d5mm-060ppm & -81.30 & 29.59 & 0.020 \\
slab-lnk18-Ha300-d5mm-060ppm & -81.30 & 29.59 & 0.020 \\
slab-lnk19-Ha274-d5mm-060ppm & -66.26 & 35.32 & 0.029 \\
slab-lnk19-Ha274-d5mm-060ppm & -66.26 & 35.32 & 0.029 \\
slab-lnk18-Ha274-d1cm-060ppm & -61.03 & 37.31 & 0.033 \\
slab-lnk18-Ha274-d1cm-060ppm & -61.03 & 37.31 & 0.033 \\
slab-lnk18-Ha274-d5mm-060ppm & -51.10 & 42.85 & 0.045 \\
slab-lnk18-Ha274-d5mm-060ppm & -51.10 & 42.85 & 0.045 \\
slab-lnk18-Ha274-D1e06-060ppm & -46.44 & 44.38 & 0.055 \\
slab-lnk18-Ha274-D1e06-060ppm & -46.44 & 44.38 & 0.055 \\
slab-lnk18-Ha274-d5mm-080ppm & -41.25 & 44.34 & 0.072 \\
slab-lnk18-Ha274-d5mm-080ppm & -41.25 & 44.34 & 0.072 \\
slab-lnk17-Ha274-d5mm-060ppm & -40.57 & 44.25 & 0.075 \\
slab-lnk17-Ha274-d5mm-060ppm & -40.57 & 44.25 & 0.075 \\
slab-lnk18-Ha274-d1mm-060ppm & -34.69 & 42.11 & 0.108 \\
slab-lnk18-Ha274-d1mm-060ppm & -34.69 & 42.11 & 0.108 \\
slab-lnk18-Ha274-d5mm-100ppm & -34.45 & 41.99 & 0.109 \\
slab-lnk18-Ha274-d5mm-100ppm & -34.45 & 41.99 & 0.109 \\
slab-lnk16-Ha274-d5mm-060ppm & -30.94 & 40.43 & 0.137 \\
slab-lnk16-Ha274-d5mm-060ppm & -30.94 & 40.43 & 0.137 \\
slab-lnk18-Ha274-d5mm-120ppm & -28.98 & 39.41 & 0.159 \\
slab-lnk18-Ha274-d5mm-120ppm & -28.98 & 39.41 & 0.159 \\
slab-lnk18-Ha274-D1e08-060ppm & -23.80 & 36.18 & 0.226 \\
slab-lnk18-Ha274-D1e08-060ppm & -23.80 & 36.18 & 0.226 \\
slab-lnk18-Ha274-d5mm-140ppm & -23.73 & 36.19 & 0.227 \\
slab-lnk18-Ha274-d5mm-140ppm & -23.73 & 36.19 & 0.227 \\
slab-lnk15-Ha274-d5mm-060ppm & -20.62 & 34.03 & 0.281 \\
slab-lnk15-Ha274-d5mm-060ppm & -20.62 & 34.03 & 0.281 \\
slab-lnk18-Ha274-D1e10-060ppm & 3.09 & 14.34 & 1.144 \\
slab-lnk18-Ha274-D1e10-060ppm & 3.09 & 14.34 & 1.144 \\
slab-lnk18-Ha274-D1e12-060ppm & 14.44 & 4.12 & 4.388 \\
slab-lnk18-Ha274-D1e12-060ppm & 14.44 & 4.12 & 4.388 \\
\end{longtable}

\end{document}