%%
% Copyright (c) 2017 - 2024, Pascal Wagler;
% Copyright (c) 2014 - 2024, John MacFarlane
%
% All rights reserved.
%
% Redistribution and use in source and binary forms, with or without
% modification, are permitted provided that the following conditions
% are met:
%
% - Redistributions of source code must retain the above copyright
% notice, this list of conditions and the following disclaimer.
%
% - Redistributions in binary form must reproduce the above copyright
% notice, this list of conditions and the following disclaimer in the
% documentation and/or other materials provided with the distribution.
%
% - Neither the name of John MacFarlane nor the names of other
% contributors may be used to endorse or promote products derived
% from this software without specific prior written permission.
%
% THIS SOFTWARE IS PROVIDED BY THE COPYRIGHT HOLDERS AND CONTRIBUTORS
% "AS IS" AND ANY EXPRESS OR IMPLIED WARRANTIES, INCLUDING, BUT NOT
% LIMITED TO, THE IMPLIED WARRANTIES OF MERCHANTABILITY AND FITNESS
% FOR A PARTICULAR PURPOSE ARE DISCLAIMED. IN NO EVENT SHALL THE
% COPYRIGHT OWNER OR CONTRIBUTORS BE LIABLE FOR ANY DIRECT, INDIRECT,
% INCIDENTAL, SPECIAL, EXEMPLARY, OR CONSEQUENTIAL DAMAGES (INCLUDING,
% BUT NOT LIMITED TO, PROCUREMENT OF SUBSTITUTE GOODS OR SERVICES;
% LOSS OF USE, DATA, OR PROFITS; OR BUSINESS INTERRUPTION) HOWEVER
% CAUSED AND ON ANY THEORY OF LIABILITY, WHETHER IN CONTRACT, STRICT
% LIABILITY, OR TORT (INCLUDING NEGLIGENCE OR OTHERWISE) ARISING IN
% ANY WAY OUT OF THE USE OF THIS SOFTWARE, EVEN IF ADVISED OF THE
% POSSIBILITY OF SUCH DAMAGE.
%%

%%
% This is the Eisvogel pandoc LaTeX template.
%
% For usage information and examples visit the official GitHub page:
% https://github.com/Wandmalfarbe/pandoc-latex-template
%%

% Options for packages loaded elsewhere
\PassOptionsToPackage{unicode}{hyperref}
\PassOptionsToPackage{hyphens}{url}
\PassOptionsToPackage{dvipsnames,svgnames,x11names,table}{xcolor}
%
\documentclass[
  paper=a4,
  ,captions=tableheading
]{scrartcl}
\usepackage{amsmath,amssymb}
% Use setspace anyway because we change the default line spacing.
% The spacing is changed early to affect the titlepage and the TOC.
\usepackage{setspace}
\setstretch{1.2}
\usepackage{iftex}
\ifPDFTeX
  \usepackage[T1]{fontenc}
  \usepackage[utf8]{inputenc}
  \usepackage{textcomp} % provide euro and other symbols
\else % if luatex or xetex
  \usepackage{unicode-math} % this also loads fontspec
  \defaultfontfeatures{Scale=MatchLowercase}
  \defaultfontfeatures[\rmfamily]{Ligatures=TeX,Scale=1}
\fi
\usepackage{lmodern}
\ifPDFTeX\else
  % xetex/luatex font selection
\fi
% Use upquote if available, for straight quotes in verbatim environments
\IfFileExists{upquote.sty}{\usepackage{upquote}}{}
\IfFileExists{microtype.sty}{% use microtype if available
  \usepackage[]{microtype}
  \UseMicrotypeSet[protrusion]{basicmath} % disable protrusion for tt fonts
}{}
\makeatletter
\@ifundefined{KOMAClassName}{% if non-KOMA class
  \IfFileExists{parskip.sty}{%
    \usepackage{parskip}
  }{% else
    \setlength{\parindent}{0pt}
    \setlength{\parskip}{6pt plus 2pt minus 1pt}}
}{% if KOMA class
  \KOMAoptions{parskip=half}}
\makeatother
\usepackage{xcolor}
\definecolor{default-linkcolor}{HTML}{A50000}
\definecolor{default-filecolor}{HTML}{A50000}
\definecolor{default-citecolor}{HTML}{4077C0}
\definecolor{default-urlcolor}{HTML}{4077C0}
\usepackage[margin=2.5cm,includehead=true,includefoot=true,centering,]{geometry}
\usepackage{longtable,booktabs,array}
\usepackage{calc} % for calculating minipage widths
% Correct order of tables after \paragraph or \subparagraph
\usepackage{etoolbox}
\makeatletter
\patchcmd\longtable{\par}{\if@noskipsec\mbox{}\fi\par}{}{}
\makeatother
% Allow footnotes in longtable head/foot
\IfFileExists{footnotehyper.sty}{\usepackage{footnotehyper}}{\usepackage{footnote}}
\makesavenoteenv{longtable}
% add backlinks to footnote references, cf. https://tex.stackexchange.com/questions/302266/make-footnote-clickable-both-ways
\usepackage{footnotebackref}
\usepackage{graphicx}
\makeatletter
\newsavebox\pandoc@box
\newcommand*\pandocbounded[1]{% scales image to fit in text height/width
  \sbox\pandoc@box{#1}%
  \Gscale@div\@tempa{\textheight}{\dimexpr\ht\pandoc@box+\dp\pandoc@box\relax}%
  \Gscale@div\@tempb{\linewidth}{\wd\pandoc@box}%
  \ifdim\@tempb\p@<\@tempa\p@\let\@tempa\@tempb\fi% select the smaller of both
  \ifdim\@tempa\p@<\p@\scalebox{\@tempa}{\usebox\pandoc@box}%
  \else\usebox{\pandoc@box}%
  \fi%
}
% Set default figure placement to htbp
% Make use of float-package and set default placement for figures to H.
% The option H means 'PUT IT HERE' (as  opposed to the standard h option which means 'You may put it here if you like').
\usepackage{float}
\floatplacement{figure}{H}
\makeatother
\setlength{\emergencystretch}{3em} % prevent overfull lines
\providecommand{\tightlist}{%
  \setlength{\itemsep}{0pt}\setlength{\parskip}{0pt}}
\setcounter{secnumdepth}{5}
% definitions for citeproc citations
\NewDocumentCommand\citeproctext{}{}
\NewDocumentCommand\citeproc{mm}{%
  \begingroup\def\citeproctext{#2}\cite{#1}\endgroup}
\makeatletter
 % allow citations to break across lines
 \let\@cite@ofmt\@firstofone
 % avoid brackets around text for \cite:
 \def\@biblabel#1{}
 \def\@cite#1#2{{#1\if@tempswa , #2\fi}}
\makeatother
\newlength{\cslhangindent}
\setlength{\cslhangindent}{1.5em}
\newlength{\csllabelwidth}
\setlength{\csllabelwidth}{3em}
\newenvironment{CSLReferences}[2] % #1 hanging-indent, #2 entry-spacing
  {\begin{list}{}{%
   \setlength{\itemindent}{0pt}
   \setlength{\leftmargin}{0pt}
   \setlength{\parsep}{0pt}
   % turn on hanging indent if param 1 is 1
   \ifodd #1
    \setlength{\leftmargin}{\cslhangindent}
    \setlength{\itemindent}{-1\cslhangindent}
   \fi
   % set entry spacing
   \setlength{\itemsep}{#2\baselineskip}}}
  {\end{list}}
\usepackage{calc}
\newcommand{\CSLBlock}[1]{\hfill\break\parbox[t]{\linewidth}{\strut\ignorespaces#1\strut}}
\newcommand{\CSLLeftMargin}[1]{\parbox[t]{\csllabelwidth}{\strut#1\strut}}
\newcommand{\CSLRightInline}[1]{\parbox[t]{\linewidth - \csllabelwidth}{\strut#1\strut}}
\newcommand{\CSLIndent}[1]{\hspace{\cslhangindent}#1}
\ifLuaTeX
\usepackage[bidi=basic]{babel}
\else
\usepackage[bidi=default]{babel}
\fi
\babelprovide[main,import]{american}
% get rid of language-specific shorthands (see #6817):
\let\LanguageShortHands\languageshorthands
\def\languageshorthands#1{}
\makeatletter
\newcounter{none}
\renewcommand{\thenone}{}
\makeatother
\makeatletter
\@ifpackageloaded{subcaption}{}{\usepackage{subcaption}}
\@ifpackageloaded{caption}{}{\usepackage{caption}}
\captionsetup[subfigure]{margin=0.5em}
\AtBeginDocument{%
\renewcommand*\figurename{Figure}
\renewcommand*\tablename{Table}
}
\AtBeginDocument{%
\renewcommand*\listfigurename{List of Figures}
\renewcommand*\listtablename{List of Tables}
}
\newcounter{pandoccrossref@subfigures@footnote@counter}
\newenvironment{pandoccrossrefsubfigures}{%
\setcounter{pandoccrossref@subfigures@footnote@counter}{0}
\begin{figure}\centering%
\gdef\global@pandoccrossref@subfigures@footnotes{}%
\DeclareRobustCommand{\footnote}[1]{\footnotemark%
\stepcounter{pandoccrossref@subfigures@footnote@counter}%
\ifx\global@pandoccrossref@subfigures@footnotes\empty%
\gdef\global@pandoccrossref@subfigures@footnotes{{##1}}%
\else%
\g@addto@macro\global@pandoccrossref@subfigures@footnotes{, {##1}}%
\fi}}%
{\end{figure}%
\addtocounter{footnote}{-\value{pandoccrossref@subfigures@footnote@counter}}
\@for\f:=\global@pandoccrossref@subfigures@footnotes\do{\stepcounter{footnote}\footnotetext{\f}}%
\gdef\global@pandoccrossref@subfigures@footnotes{}}
\@ifpackageloaded{float}{}{\usepackage{float}}
\floatstyle{ruled}
\@ifundefined{c@chapter}{\newfloat{codelisting}{h}{lop}}{\newfloat{codelisting}{h}{lop}[chapter]}
\floatname{codelisting}{Listing}
\newcommand*\listoflistings{\listof{codelisting}{List of Listings}}
\makeatother
\usepackage{bookmark}
\IfFileExists{xurl.sty}{\usepackage{xurl}}{} % add URL line breaks if available
\urlstyle{same}
\hypersetup{
  pdftitle={Beyond Equilibrium},
  pdfauthor={Kerswell B.; Wheeler J.; Gassmöller R.; Davies, J.H.},
  pdflang={en-US},
  pdfsubject={Mantle Convection},
  pdfkeywords={mantle convection, phase transition, reaction
kinetics, geodynamics, numerical modeling},
  colorlinks=true,
  linkcolor={default-linkcolor},
  filecolor={default-filecolor},
  citecolor={default-citecolor},
  urlcolor={default-urlcolor},
  breaklinks=true,
  pdfcreator={LaTeX via pandoc with the Eisvogel template}}
\title{Beyond Equilibrium}
\usepackage{etoolbox}
\makeatletter
\providecommand{\subtitle}[1]{% add subtitle to \maketitle
  \apptocmd{\@title}{\par {\large #1 \par}}{}{}
}
\makeatother
\subtitle{Kinetic Thresholds and Rheological Feedbacks Control 410
Topography}
\author{Kerswell B. \and Wheeler J. \and Gassmöller R. \and Davies,
J.H.}
\date{24 November 2025}



%%
%% added
%%


%
% for the background color of the title page
%
\usepackage{pagecolor}
\usepackage{afterpage}
\usepackage[margin=2.5cm,includehead=true,includefoot=true,centering]{geometry}

%
% break urls
%
\PassOptionsToPackage{hyphens}{url}

%
% When using babel or polyglossia with biblatex, loading csquotes is recommended
% to ensure that quoted texts are typeset according to the rules of your main language.
%
\usepackage{csquotes}

%
% captions
%
\definecolor{caption-color}{HTML}{777777}
\usepackage[font={stretch=1.2}, textfont={color=caption-color}, position=top, skip=4mm, labelfont=bf, singlelinecheck=false, justification=justified]{caption}
\setcapindent{0em}

%
% blockquote
%
\definecolor{blockquote-border}{RGB}{221,221,221}
\definecolor{blockquote-text}{RGB}{119,119,119}
\usepackage{mdframed}
\newmdenv[rightline=false,bottomline=false,topline=false,linewidth=3pt,linecolor=blockquote-border,skipabove=\parskip]{customblockquote}
\renewenvironment{quote}{\begin{customblockquote}\list{}{\rightmargin=0em\leftmargin=0em}%
\item\relax\color{blockquote-text}\ignorespaces}{\unskip\unskip\endlist\end{customblockquote}}

%
% Source Sans Pro as the default font family
% Source Code Pro for monospace text
%
% 'default' option sets the default
% font family to Source Sans Pro, not \sfdefault.
%
\ifnum 0\ifxetex 1\fi\ifluatex 1\fi=0 % if pdftex
    \usepackage[default]{sourcesanspro}
  \usepackage{sourcecodepro}
  \else % if not pdftex
    \usepackage[default]{sourcesanspro}
  \usepackage{sourcecodepro}

  % XeLaTeX specific adjustments for straight quotes: https://tex.stackexchange.com/a/354887
  % This issue is already fixed (see https://github.com/silkeh/latex-sourcecodepro/pull/5) but the
  % fix is still unreleased.
  % TODO: Remove this workaround when the new version of sourcecodepro is released on CTAN.
  \ifxetex
    \makeatletter
    \defaultfontfeatures[\ttfamily]
      { Numbers   = \sourcecodepro@figurestyle,
        Scale     = \SourceCodePro@scale,
        Extension = .otf }
    \setmonofont
      [ UprightFont    = *-\sourcecodepro@regstyle,
        ItalicFont     = *-\sourcecodepro@regstyle It,
        BoldFont       = *-\sourcecodepro@boldstyle,
        BoldItalicFont = *-\sourcecodepro@boldstyle It ]
      {SourceCodePro}
    \makeatother
  \fi
  \fi

%
% heading color
%
\definecolor{heading-color}{RGB}{40,40,40}
\addtokomafont{section}{\color{heading-color}}
% When using the classes report, scrreprt, book,
% scrbook or memoir, uncomment the following line.
%\addtokomafont{chapter}{\color{heading-color}}

%
% variables for title, author and date
%
\usepackage{titling}
\title{Beyond Equilibrium}
\author{Kerswell B., Wheeler J., Gassmöller R., Davies, J.H.}
\date{24 November 2025}

%
% tables
%

\definecolor{table-row-color}{HTML}{F5F5F5}
\definecolor{table-rule-color}{HTML}{999999}

%\arrayrulecolor{black!40}
\arrayrulecolor{table-rule-color}     % color of \toprule, \midrule, \bottomrule
\setlength\heavyrulewidth{0.3ex}      % thickness of \toprule, \bottomrule
\renewcommand{\arraystretch}{1.3}     % spacing (padding)


%
% remove paragraph indentation
%
\setlength{\parindent}{0pt}
\setlength{\parskip}{6pt plus 2pt minus 1pt}
\setlength{\emergencystretch}{3em}  % prevent overfull lines

%
%
% Listings
%
%


%
% header and footer
%
\usepackage[headsepline,footsepline]{scrlayer-scrpage}

\newpairofpagestyles{eisvogel-header-footer}{
  \clearpairofpagestyles
  \ihead*{Beyond Equilibrium}
  \chead*{}
  \ohead*{24 November 2025}
  \ifoot*{Kerswell B., Wheeler J., Gassmöller R., Davies, J.H.}
  \cfoot*{}
  \ofoot*{\thepage}
  \addtokomafont{pageheadfoot}{\upshape}
}
\pagestyle{eisvogel-header-footer}



%%
%% end added
%%

\begin{document}

%%
%% begin titlepage
%%
\begin{titlepage}
\newgeometry{left=6cm}
\definecolor{titlepage-color}{HTML}{2E7A40}
\newpagecolor{titlepage-color}\afterpage{\restorepagecolor}
\newcommand{\colorRule}[3][black]{\textcolor[HTML]{#1}{\rule{#2}{#3}}}
\begin{flushleft}
\noindent
\\[-1em]
\color[HTML]{FFFFFF}
\makebox[0pt][l]{\colorRule[FFFFFF]{1.3\textwidth}{2pt}}
\par
\noindent

{
  \setstretch{1.4}
  \vfill
  \noindent {\huge \textbf{\textsf{Beyond Equilibrium}}}
    \vskip 1em
  {\Large \textsf{Kinetic Thresholds and Rheological Feedbacks Control
410 Topography}}
    \vskip 2em
  \noindent {\Large \textsf{Kerswell B., Wheeler J., Gassmöller
R., Davies, J.H.}}
  \vfill
}


\textsf{24 November 2025}
\end{flushleft}
\end{titlepage}
\restoregeometry
\pagenumbering{arabic}

%%
%% end titlepage
%%

% \maketitle


{
\hypersetup{linkcolor=}
\setcounter{tocdepth}{3}
\tableofcontents
\newpage
}
\section*{Abstract}\label{sec:abstract}
\addcontentsline{toc}{section}{Abstract}

The seismic expression of Earth's 410 km discontinuity varies
substantially across tectonic settings, from sharp, high-amplitude
interfaces to broad, diffuse transitions---patterns that cannot be
explained by equilibrium thermodynamics without invoking large-scale
thermal and/or compositional heterogeneities in the upper mantle.
Laboratory experiments demonstrate that the olivine \(\Leftrightarrow\)
wadsleyite phase transition responsible for the 410 is rate-limited, yet
previous numerical studies investigating olivine metastability have not
directly evaluated the sensitivity of 410 topography to kinetic and
rheological factors. Here we systematically investigate these
relationships by coupling an interface-controlled growth model to
compressible simulations of mantle plumes and subducting slabs. We vary
kinetic parameters across seven orders of magnitude and quantify the
resulting 410 displacements and widths. Our results reveal a fundamental
asymmetry between hot and cold mantle environments. In plumes, high
temperatures produce consistently sharp 410s (2--3 km wide) regardless
of kinetic or rheological parameters. In slabs, kinetics exert
first-order control on 410 structure and flow dynamics through three
distinct regimes: (1) quasi-equilibrium behavior at high reaction rates
producing narrow, uplifted 410s and continuous slab penetration; (2)
intermediate reaction rates generating progressively broader, deeper
410s with metastable olivine wedges that resist but do not prevent slab
descent; and (3) ultra-sluggish reaction rates causing complete slab
stagnation with re-sharpened but deeply displaced 410s (\(\lesssim\) 100
km). Strength contrasts further modulate these kinetic effects by
controlling slab geometry and residence time in the phase transition
zone. These findings demonstrate that reaction rates strongly influence
410 structure in subduction zones and establish the 410 as a potential
seismological constraint on kinetic processes operating in Earth's upper
mantle, particularly in cold environments where disequilibrium effects
are amplified.

\section*{Plain Language Summary}\label{sec:plain-language-summary}
\addcontentsline{toc}{section}{Plain Language Summary}

Seismic imaging reveals that Earth's 410 km discontinuity---a boundary
in the mantle marking where the mineral olivine transforms into a denser
form called wadsleyite---looks very different in different parts of the
planet, sometimes appearing as a sharp boundary and other times as a
broad, fuzzy zone. While scientists have traditionally explained these
variations through temperature and compositional differences, laboratory
experiments show that this mineral transformation takes some time to
complete, raising the possibility that reaction speed (kinetics) plays
an important role. We used computer simulations of rising mantle plumes
and descending tectonic slabs to explore how kinetics and rock strength
affect the structure of the 410 km discontinuity. Our models reveal a
striking difference between hot and cold regions: in hot mantle plumes,
the discontinuity remains consistently sharp (2-3 km thick) because high
temperatures allow reactions to proceed quickly, but in cold subducting
slabs, kinetics becomes critical. Fast reactions produce a narrow,
uplifted discontinuity as slabs sink smoothly, moderate reaction rates
create broader, deeper discontinuities with wedges of unreacted olivine
that slow slab descent, and extremely slow reactions cause slabs to
stagnate completely with sharp but deeply depressed discontinuities up
to 100 km below their normal depth. These findings suggest that
observations of 410 km discontinuity structure, particularly in
subduction zones, could provide valuable constraints on how fast mineral
reactions occur deep within Earth's mantle.

\section*{Keypoints}\label{sec:key-points}
\addcontentsline{toc}{section}{Keypoints}

\begin{itemize}
\tightlist
\item
  Plumes produce sharp 410s regardless of reaction rates; slabs show
  three distinct kinetic regimes controlling discontinuity structure
\item
  Rheology modulates kinetic effects: strong slabs descend slowly
  allowing complete reaction; weak slabs descend fast amplifying
  metastability
\item
  Seismic observations of 410 structure in subduction zones can
  constrain reaction rates but require independent rheological
  constraints
\end{itemize}

\clearpage

\section*{Definition of Symbols}\label{sec:symbols}
\addcontentsline{toc}{section}{Definition of Symbols}

{\def\LTcaptype{none} % do not increment counter
\begin{longtable}[]{@{}
  >{\raggedright\arraybackslash}p{(\linewidth - 6\tabcolsep) * \real{0.4688}}
  >{\raggedright\arraybackslash}p{(\linewidth - 6\tabcolsep) * \real{0.2188}}
  >{\raggedright\arraybackslash}p{(\linewidth - 6\tabcolsep) * \real{0.1562}}
  >{\raggedright\arraybackslash}p{(\linewidth - 6\tabcolsep) * \real{0.1562}}@{}}
\toprule\noalign{}
\begin{minipage}[b]{\linewidth}\raggedright
Parameter
\end{minipage} & \begin{minipage}[b]{\linewidth}\raggedright
Symbol
\end{minipage} & \begin{minipage}[b]{\linewidth}\raggedright
Unit
\end{minipage} & \begin{minipage}[b]{\linewidth}\raggedright
Equations
\end{minipage} \\
\midrule\noalign{}
\endhead
\bottomrule\noalign{}
\endlastfoot
Activation energy & \(E^{\ast}\) & J mol\(^{-1}\) &
\ref{eq:arrhenius-viscosity} \\
Activation enthalpy & \(H^{\ast}\) & J mol\(^{-1}\) &
\ref{eq:growth-rate}, \ref{eq:reaction-rate} \\
Activation factor (rheology) & \(B\) & - & \ref{eq:rheological-model} \\
Activation volume & \(V^{\ast}\) & m\(^3\) mol\(^{-1}\) &
\ref{eq:growth-rate}, \ref{eq:reaction-rate} \\
Compressibility (reference) & \(\bar{\beta}\) & Pa\(^{-1}\) &
\ref{eq:density-ala} \\
Density & \(\rho\) & kg m\(^{-3}\) &
\ref{eq:navier-stokes-no-inertia}--\ref{eq:continuity-expanded},
\ref{eq:density-ala-expansion}--\ref{eq:density-ala} \\
Density (reference) & \(\bar{\rho}\) & kg m\(^{-3}\) &
\ref{eq:adiabatic-pressure}--\ref{eq:density-ala} \\
Density (dynamic) & \(\hat{\rho}\) & kg m\(^{-3}\) & - \\
Deviatoric stress tensor & \(\sigma^{\prime}\) & Pa &
\ref{eq:navier-stokes-no-inertia}, \ref{eq:energy} \\
Deviatoric strain rate tensor & \(\dot{\epsilon}^{\prime}\) & s\(^{-1}\)
& \ref{eq:energy} \\
Gas constant & \(R\) & J mol\(^{-1}\) K\(^{-1}\) & \ref{eq:growth-rate},
\ref{eq:reaction-rate},
\ref{eq:arrhenius-viscosity}--\ref{eq:rheological-model} \\
Grain size & \(d\) & m & - \\
Gravitational acceleration & \(g\) & m s\(^{-2}\) &
\ref{eq:navier-stokes-no-inertia},
\ref{eq:adiabatic-temperature}--\ref{eq:adiabatic-pressure} \\
Growth rate & \(\dot{x}\) & m s\(^{-1}\) &
\ref{eq:volume-fraction}--\ref{eq:growth-rate},
\ref{eq:reaction-rate-short} \\
Latent heat & \(Q_L\) & J kg\(^{-1}\) & \ref{eq:energy} \\
Molar entropy & \(\bar{S}\) & J mol\(^{-1}\) K\(^{-1}\) &
\ref{eq:gibbs} \\
Molar Gibbs free energy & \(\bar{G}\) & J mol\(^{-1}\) &
\ref{eq:gibbs} \\
Molar volume & \(\bar{V}\) & m\(^{3}\) mol\(^{-1}\) & \ref{eq:gibbs} \\
Nucleation site factor & \(N\) & m\(^{-1}\) & \ref{eq:volume-fraction},
\ref{eq:reaction-rate-short} \\
Prefactor (growth rate) & \(\Gamma\) & m s\(^{-1}\) K\(^{-1}\)
ppm\(_\mathrm{OH}^{-n}\) & \ref{eq:growth-rate} \\
Prefactor (kinetic) & \(Z\) & K s\(^{-1}\) & \ref{eq:reaction-rate} \\
Prefactor (viscosity) & \(A\) & Pa s &
\ref{eq:arrhenius-viscosity}--\ref{eq:background-viscosity} \\
Pressure & \(P\) & Pa & \ref{eq:navier-stokes-no-inertia},
\ref{eq:energy}, \ref{eq:growth-rate}, \ref{eq:reaction-rate} \\
Pressure (reference) & \(\bar{P}\) & Pa & \ref{eq:adiabatic-pressure} \\
Pressure (dynamic) & \(\hat{P}\) & Pa & \ref{eq:density-ala},
\ref{eq:gibbs} \\
Reaction rate & \(\frac{dX}{dt}\), \(\frac{\partial X}{\partial t}\),
\(\dot{X}\) & s\(^{-1}\) &
\ref{eq:reaction-rate-short}--\ref{eq:composition} \\
Specific heat capacity (reference) & \(\bar{C}_p\) & J kg\(^{-1}\)
K\(^{-1}\) & \ref{eq:energy}, \ref{eq:adiabatic-temperature} \\
Temperature & \(T\) & K & \ref{eq:energy}, \ref{eq:growth-rate},
\ref{eq:reaction-rate}, \ref{eq:arrhenius-viscosity},
\ref{eq:rheological-model} \\
Temperature (reference) & \(\bar{T}\) & K &
\ref{eq:adiabatic-temperature},
\ref{eq:arrhenius-viscosity-expanded}--\ref{eq:rheological-model} \\
Temperature (dynamic) & \(\hat{T}\) & K & \ref{eq:density-ala},
\ref{eq:gibbs}, \ref{eq:arrhenius-viscosity-expanded},
\ref{eq:rheological-model} \\
Thermal conductivity (reference) & \(\bar{k}\) & W m\(^{-1}\) K\(^{-1}\)
& \ref{eq:energy} \\
Thermal expansivity (reference) & \(\bar{\alpha}\) & K\(^{-1}\) &
\ref{eq:energy}, \ref{eq:adiabatic-temperature}, \ref{eq:density-ala} \\
Time & \(t\) & s &
\ref{eq:continuity-compressible}--\ref{eq:continuity-expanded},
\ref{eq:volume-fraction},
\ref{eq:reaction-rate-short}--\ref{eq:composition} \\
Velocity & \(\vec{u}\) & m s\(^{-1}\) &
\ref{eq:continuity-compressible}--\ref{eq:continuity-expanded},
\ref{eq:composition} \\
Viscosity & \(\eta\) & Pa s &
\ref{eq:arrhenius-viscosity}--\ref{eq:arrhenius-viscosity-expanded},
\ref{eq:rheological-model} \\
Viscosity (reference) & \(\bar{\eta}\) & Pa s &
\ref{eq:background-viscosity}--\ref{eq:rheological-model} \\
Volume fraction & \(X\) & - & \ref{eq:volume-fraction},
\ref{eq:reaction-rate-short}--\ref{eq:composition} \\
Water content & \(C_\mathrm{OH}\) & ppm & \ref{eq:growth-rate} \\
Water content exponent & \(n\) & - & \ref{eq:growth-rate} \\
\end{longtable}
}

\clearpage

\section{Introduction}\label{sec:introduction}

Earth's mantle transition zone hosts two prominent seismic
discontinuities near 410 and 660 km depth, attributed to polymorphic
phase transitions of olivine (\citeproc{ref-katsura2004}{{Katsura et
al.}, 2004}; \citeproc{ref-ringwood1975}{Ringwood, 1975}). While these
discontinuities are observed globally, their detailed seismic
characteristics---depth, sharpness, amplitude, and lateral
continuity---vary substantially between tectonic settings
(\citeproc{ref-deuss2009}{Deuss, 2009}; \citeproc{ref-fukao2013}{Fukao
\& Obayashi, 2013}; \citeproc{ref-lawrence2008}{Lawrence \& Shearer,
2008}; \citeproc{ref-schmerr2007}{Schmerr \& Garnero, 2007}). Some
regions display sharp, high-amplitude reflectors consistent with abrupt
mineralogical boundaries, while others exhibit broad, weakened, or
laterally variable signals. Such heterogeneity cannot be explained by
equilibrium thermodynamics alone, which relates discontinuity topography
mainly to temperature--dependent phase boundaries (e.g.,
\citeproc{ref-cottaar2016}{Cottaar \& Deuss, 2016};
\citeproc{ref-jenkins2016}{Jenkins et al., 2016}). Additional physical
processes, including reaction kinetics and dynamic pressure effects
(\citeproc{ref-faccenda2017}{Faccenda \& Dal Zilio, 2017};
\citeproc{ref-rubie1994}{Rubie \& Ross II, 1994}), or compositional
heterogeneities, including variations in water content
(\citeproc{ref-karato2011}{Karato, 2011};
\citeproc{ref-smyth1987}{Smyth, 1987}; \citeproc{ref-smyth2002}{Smyth \&
Frost, 2002}) or bulk composition (\citeproc{ref-glasgow2024}{Glasgow et
al., 2024}; \citeproc{ref-goes2022}{Goes et al., 2022};
\citeproc{ref-saikia2008}{Saikia et al., 2008};
\citeproc{ref-schmerr2007}{Schmerr \& Garnero, 2007};
\citeproc{ref-tauzin2017}{Tauzin et al., 2017}) likely contribute to
observed variability.

Laboratory studies demonstrate that the olivine \(\Leftrightarrow\)
wadsleyite phase transition at 410 km depth (the ``410'') is
rate-limited, with kinetics governed by temperature, pressure, water
content, bulk chemical composition, grain size, and microstructural
evolution (\citeproc{ref-hosoya2005}{Hosoya et al., 2005};
\citeproc{ref-kubo2004}{Kubo et al., 2004};
\citeproc{ref-ledoux2023}{{Ledoux et al.}, 2023};
\citeproc{ref-liu1998}{Liu et al., 1998};
\citeproc{ref-perrillat2013}{Perrillat et al., 2013};
\citeproc{ref-rubie1994}{Rubie \& Ross II, 1994}). In cold subducting
slabs, sluggish reaction rates can allow metastable olivine to persist
tens of kilometers below its thermodynamic stability limit, promoting
slab stagnation and potentially triggering deep earthquakes via
transformational faulting (\citeproc{ref-green1995}{Green \& Houston,
1995}; \citeproc{ref-ishii2021}{Ishii \& Ohtani, 2021};
\citeproc{ref-kirby1996}{Kirby et al., 1996};
\citeproc{ref-ohuchi2022}{Ohuchi et al., 2022};
\citeproc{ref-rubie1994}{Rubie \& Ross II, 1994};
\citeproc{ref-sindhusuta2025}{Sindhusuta et al., 2025}). In hot
upwellings, slow kinetics may broaden and uplift the discontinuity,
possibly explaining reduced seismic amplitudes beneath some hotspots
(\citeproc{ref-chambers2005}{Chambers et al., 2005}). However, published
kinetic models remain poorly constrained, with parameters spanning
several orders of magnitude (e.g., \citeproc{ref-hosoya2005}{Hosoya et
al., 2005}), leaving the effects of micro-scale kinetic processes on
flow dynamics and seismic observables ambiguous.

Bridging the gap between laboratory-derived kinetic rate laws and
mantle-scale seismic observations requires numerical models that couple
reaction kinetics to realistic treatments of mantle convection. Previous
modeling efforts have demonstrated that kinetics can strongly influence
mantle flow (\citeproc{ref-agrusta2017}{Agrusta et al., 2017};
\citeproc{ref-dassler1996b}{Däßler et al., 1996};
\citeproc{ref-dassler1996a}{Däßler \& Yuen, 1996};
\citeproc{ref-faccenda2017}{Faccenda \& Dal Zilio, 2017};
\citeproc{ref-guest2004}{Guest et al., 2004};
\citeproc{ref-schmeling1999}{Schmeling et al., 1999}), but
investigations quantifying the sensitivity of 410 structure to kinetic
parameters remain limited. Moreover, most prior studies impose kinematic
restrictions and/or employ simplified treatments of compressibility or
kinetic rate laws that may inadequately capture feedbacks between
kinetically controlled phase transitions and flow dynamics.
Additionally, the role of rheology in modulating kinetic effects through
its control on slab geometry and descent rate has not been thoroughly
explored.

This study aims to clarify these issues by implementing an
interface-controlled growth model (after
\citeproc{ref-hosoya2005}{Hosoya et al., 2005}) within compressible
mantle flow simulations using the open-source geodynamic modeling
software ASPECT. We systematically explore how reaction kinetics and
rheological strength jointly influence 410 structure and address three
specific questions:

\begin{enumerate}
\def\labelenumi{\arabic{enumi}.}
\tightlist
\item
  How do kinetic factors impact flow dynamics and shape the 410 in hot
  versus cold environments?
\item
  How do viscosity contrasts modulate these kinetic effects?
\item
  Can seismic observations of 410 structure constrain effective kinetic
  and rheological parameters?
\end{enumerate}

To investigate these questions, we analyze a suite of numerical
experiments varying kinetic parameters across seven orders of magnitude.
For each experiment, we test a range of viscosity contrasts and quantify
410 displacement and width, enabling direct comparisons with
seismological observations. Our results establish quantitative
relationships between reaction rates, rheological strength, flow
dynamics, and 410 structure, demonstrating that realistic treatment of
reaction kinetics is essential for accurately modeling subduction
dynamics and interpreting seismic structures.

\section{Methods}\label{sec:methods}

\subsection{Governing Equations for Compressible Mantle
Flow}\label{sec:governing-equations}

Mantle flow is simulated using the finite-element geodynamic code ASPECT
(v3.0.0, \citeproc{ref-aspect-doi-v3.0.0}{Bangerth et al., 2024a},
\citeproc{ref-aspectmanual}{2024b};
\citeproc{ref-clevenger2021}{Clevenger \& Heister, 2021};
\citeproc{ref-fraters2019}{{Fraters et al.}, 2019};
\citeproc{ref-fraters2020}{Fraters, 2020};
\citeproc{ref-gassmoller2018}{Gassmöller et al., 2018};
\citeproc{ref-heister2017}{Heister et al., 2017};
\citeproc{ref-kronbichler2012}{Kronbichler et al., 2012}) to find the
velocity \(\vec{u}\), pressure \(P\), and temperature \(T\) fields that
satisfy the following equations:

\begin{equation}
  \nabla P - \nabla \cdot \sigma^{\prime} = \rho\, g
  \label{eq:navier-stokes-no-inertia}
\end{equation}

\begin{equation}
  \frac{\partial \rho}{\partial t} + \nabla \cdot (\rho\, \vec{u}) = 0
  \label{eq:continuity-compressible}
\end{equation}

\begin{equation}
  \rho\, \bar{C}_p \left(\frac{\partial T}{\partial t} + \vec{u} \cdot \nabla T \right) - \nabla \cdot \left(\bar{k}\, \nabla T \right) = \sigma^{\prime} : \dot{\epsilon}^{\prime} + \bar{\alpha}\, T \left(\vec{u} \cdot \nabla P \right) + Q_L
  \label{eq:energy}
\end{equation}

where \(\sigma^{\prime}\) is the deviatoric stress tensor, \(\rho\) is
density, \(g\) is gravitational acceleration, \(t\) is time,
\(\bar{C}_p\), \(\bar{k}\), \(\bar{\alpha}\) are the reference specific
heat capacity, thermal conductivity, and thermal expansivity,
respectively (see Section \ref{sec:adiabatic-reference-conditions}), and
\(Q_L\) is the latent heat released or absorbed during phase
transitions. Equations \ref{eq:navier-stokes-no-inertia} and
\ref{eq:continuity-compressible} together describe the buoyancy-driven
flow of an isotropic fluid with negligible inertia and Equation
\ref{eq:energy} describes the conduction, advection, and production (or
consumption) of thermal energy (\citeproc{ref-schubert2001}{Schubert et
al., 2001}). Note that the pressure \(P\) in this context is equal to
the mean normal stress and is positive under compression:
\(P = - \frac{\sigma_{xx} + \sigma_{yy}}{2}\).

The compressible form of the continuity equation (Equation
\ref{eq:continuity-compressible}) is essential for capturing the full
coupling between density changes, pressure and temperature (PT)
variations, and kinetically-controlled phase transitions. This is in
contrast to simplified formulations such as the Boussinesq approximation
or anelastic liquid approximation, which either neglect the time
derivative of density (\(\frac{\partial \rho}{\partial t}\)) entirely or
impose restrictions on where density variations can appear in the
governing equations (\citeproc{ref-gassmoller2020}{Gassmöller et al.,
2020}). By retaining \(\frac{\partial \rho}{\partial t}\), the
compressible continuity equation enables density changes from
kinetically-controlled phase transitions to influence the flow field
anywhere within the model domain---not just through equilibrium
thermodynamics, but through time-dependent reaction progress. This
bidirectional coupling between flow dynamics and reaction kinetics is
critical for accurately simulating systems where phase transitions occur
over finite timescales comparable to advective timescales.

To solve Equation \ref{eq:continuity-compressible} numerically we adopt
the \emph{projected density approximation} (PDA,
\citeproc{ref-gassmoller2020}{Gassmöller et al., 2020}), which
reformulates the continuity equation by applying the product rule to
\(\nabla \cdot (\rho\, \vec{u})\) and multiplying both sides by
\(\frac{1}{\rho}\):

\begin{equation}
  \frac{1}{\rho} \frac{\partial \rho}{\partial t} + \nabla \cdot \vec{u} + \left(\frac{1}{\rho} \nabla \rho \right) \cdot \vec{u} = 0
  \label{eq:continuity-expanded}
\end{equation}

The projected density field \(\rho(T, P, X)\) varies with temperature,
pressure, and reaction progress, ensuring that buoyancy-driven flow
responds dynamically to local changes in density arising from both PT
variations and phase transitions. The phase transitions themselves are
modeled using a separate kinetic rate law (described in Section
\ref{sec:reaction-kinetics}), which determines the reaction progress
\(X\) based on local thermodynamic and kinetic conditions. This makes
the PDA ideally suited for our numerical experiments, which incorporate
density changes due to the olivine \(\Leftrightarrow\) wadsleyite phase
transition.

\subsection{Numerical Setup}\label{sec:numerical-setup}

\subsubsection{Adiabatic Reference
Conditions}\label{sec:adiabatic-reference-conditions}

To ensure numerical convergence, we initialized our ASPECT simulations
with reasonable estimates of the pressure-temperature (PT) fields and
material properties in Earth's upper mantle. We began by evaluating
entropy changes over a PT range of 1573--1973 K and 0.001--25 GPa
(Figure \ref{fig:isentrope}) using the Gibbs free energy minimization
software Perple\_X (v.7.0.9, \citeproc{ref-connolly2009}{Connolly,
2009}). We assumed a dry pyrolitic bulk composition after Green et al.
(\citeproc{ref-green1979}{1979}) and phase equilibria were evaluated in
the Na\(_2\)O-CaO-FeO-MgO-Al\(_2\)O\(_3\)-SiO\(_2\) (NCFMAS) chemical
system with thermodynamic data and solution models of Stixrude \&
Lithgow-Bertelloni (\citeproc{ref-stixrude2022}{2022}). Equations of
state were included for solid solution phases: olivine, plagioclase,
spinel, clinopyroxene, wadsleyite, ringwoodite, perovskite,
ferropericlase, high‐pressure C2/c pyroxene, orthopyroxene, akimotoite,
post‐perovskite, Ca‐ferrite, garnet, and Na-Al phase.

\begin{figure}
\centering
\includegraphics[width=0.65\linewidth,height=\textheight,keepaspectratio,alt={Entropy (left) and density (right) changes in Earth's upper mantle under thermodynamic equilibrium and hydrostatic stress conditions. Material properties were computed with Perple\_X using the equations of state and thermodynamic data of Stixrude \& Lithgow-Bertelloni (2022). The black box indicates the approximate PT range of our ASPECT simulations, while the white line indicates the isentropic adiabat used to calculate reference material properties.}]{../figs/PYR-material-table.png}
\caption{Entropy (left) and density (right) changes in Earth's upper
mantle under thermodynamic equilibrium and hydrostatic stress
conditions. Material properties were computed with Perple\_X using the
equations of state and thermodynamic data of Stixrude \&
Lithgow-Bertelloni (\citeproc{ref-stixrude2022}{2022}). The black box
indicates the approximate PT range of our ASPECT simulations, while the
white line indicates the isentropic adiabat used to calculate reference
material properties.}\label{fig:isentrope}
\end{figure}

We then determined the mantle adiabat by applying the Newton--Raphson
algorithm to find temperatures corresponding to each pressure that
maintain constant entropy (white line in Figure \ref{fig:isentrope}).
Material properties were evaluated at each PT point along the isentrope
to construct the adiabatic reference conditions shown in Figure
\ref{fig:material-property-profile}. These reference conditions serve
three main purposes: 1) initializing the PT fields and material
properties in our ASPECT simulations (see Section
\ref{sec:initialization-and-boundary-conditions}), 2) updating the
material model during the simulations (see Section
\ref{sec:material-model}), and 3) serving as a basis for computing
``dynamic'' quantities, such as the dynamic temperature
\(\hat{T} = T - \bar{T}\), dynamic pressure \(\hat{P} = P - \bar{P}\),
and dynamic density \(\hat{\rho} = \rho - \bar{\rho}\), that quantify
how much the approximate numerical solution deviates from the adiabatic
reference conditions.

\begin{figure}
\centering
\includegraphics[width=1\linewidth,height=\textheight,keepaspectratio,alt={Reference material properties used in our ASPECT simulations. Profiles were computed using the BurnMan software (Cottaar et al., 2014; Myhill et al., 2023) and were based on the equations of state and thermodynamic data of Stixrude \& Lithgow-Bertelloni (2022) for pure Mg olivine (ol) and wadsleyite (wd).}]{../figs/material-property-profile.png}
\caption{Reference material properties used in our ASPECT simulations.
Profiles were computed using the BurnMan software
(\citeproc{ref-cottaar2014}{Cottaar et al., 2014};
\citeproc{ref-myhill2023}{Myhill et al., 2023}) and were based on the
equations of state and thermodynamic data of Stixrude \&
Lithgow-Bertelloni (\citeproc{ref-stixrude2022}{2022}) for pure Mg
olivine (ol) and wadsleyite (wd).}\label{fig:material-property-profile}
\end{figure}

\subsubsection{Initialization and Boundary
Conditions}\label{sec:initialization-and-boundary-conditions}

We setup our ASPECT simulations within a 900 \(\times\) 600 km
rectangular model domain, initialized with pure Mg olivine and
wadsleyite (Figure \ref{fig:initial-setup}). ``Surface'' PT conditions
of 10 GPa and 1706 K were applied at the top boundary such that the
olivine \(\Leftrightarrow\) wadsleyite transition occurs at
approximately 130--140 km from the top boundary. The initial adiabatic
PT profiles were computed by numerically integrating the following
equations:

\begin{equation}
  \frac{d\bar{T}}{dy} = \frac{\bar{\alpha}\, \bar{T}\, g}{\bar{C}_p}
  \label{eq:adiabatic-temperature}
\end{equation}

\begin{equation}
  \frac{d\bar{P}}{dy} = \bar{\rho}\, g
  \label{eq:adiabatic-pressure}
\end{equation}

where the material properties \(\bar{\rho}\), \(\bar{\alpha}\), and
\(\bar{C}_p\) were determined from the adiabatic reference conditions
shown in Figure \ref{fig:material-property-profile}.

\begin{figure}
\centering
\includegraphics[width=0.45\linewidth,height=\textheight,keepaspectratio,alt={Initial setup for slab (top) and plume (bottom) simulations. The top boundary has a constant ``surface'' PT of 10 GPa and 1706 K such that the olivine \textbackslash Leftrightarrow wadsleyite phase transition (dashed line) occurs at 130--140 km from the top boundary. Thermal anomalies with Gaussian profiles were superimposed on top of the initial adiabatic temperature profile and remained fixed at the inflow boundary. Constant inflow velocities of 5 cm/yr were prescribed parallel to the thermal anomalies. Normal tractions equal to the initial lithostatic pressure profile were enforced on the left, right, and open boundaries to ensure that outflows are driven only by dynamic pressures.}]{../figs/initial-setup.png}
\caption{Initial setup for slab (top) and plume (bottom) simulations.
The top boundary has a constant ``surface'' PT of 10 GPa and 1706 K such
that the olivine \(\Leftrightarrow\) wadsleyite phase transition (dashed
line) occurs at 130--140 km from the top boundary. Thermal anomalies
with Gaussian profiles were superimposed on top of the initial adiabatic
temperature profile and remained fixed at the inflow boundary. Constant
inflow velocities of 5 cm/yr were prescribed parallel to the thermal
anomalies. Normal tractions equal to the initial lithostatic pressure
profile were enforced on the left, right, and open boundaries to ensure
that outflows are driven only by dynamic
pressures.}\label{fig:initial-setup}
\end{figure}

Thermal anomalies with amplitudes of \(\pm\) 500 K were superimposed on
the adiabatic temperature profile. These anomalies were defined as
smooth linear features with Gaussian cross-sections (15 km half-width)
and tanh tapered ends (5 km taper length) to avoid sharp
discontinuities. Slabs extended 100 km horizontally and 100 km downward
from the top boundary; plumes extended 450 km upward from the bottom
boundary. Velocity boundary conditions prescribed constant inflow of 5
cm/yr parallel to the thermal anomalies, tapering smoothly to zero at
the thermal anomaly edges with the same Gaussian profile. Zero
horizontal velocities were imposed at the side boundaries (\(\vec{u}_x\)
= 0). The full functional form of the Gaussian-tanh thermal anomalies
and velocity boundary conditions are available within the accompanying
data repository (see Data Availability statement for details).

Stress boundary conditions on the left and right boundaries enforced a
normal traction equal to the lithostatic pressure profile computed in
Equation \ref{eq:adiabatic-pressure}. No tangential (shear) stresses
were applied to the side boundaries, so they approximated impermeable,
free-slip surfaces under hydrostatic confinement. Open boundaries
(bottom for slabs, top for plumes) were assigned a constant normal
traction equal to the initial lithostatic pressure at the respective
boundary \(\sigma_{yy}\) = \(\bar{P}(y)\). These stress conditions allow
outflow to occur freely at the top or bottom boundaries (for plumes
versus slabs), driven only by dynamic pressure variations associated
with convection and/or volume changes during the olivine
\(\Leftrightarrow\) wadsleyite phase transition.

\subsubsection{Material Model}\label{sec:material-model}

\paragraph{Material Properties}\label{sec:material-properties}

Material properties were updated during our ASPECT simulations by
referencing the adiabatic reference conditions shown in Figure
\ref{fig:material-property-profile}. Except for density, material
properties received no PT corrections, effectively assuming that
deviations from the adiabatic reference conditions were negligible. For
density, however, we applied a dynamic PT correction through a
first-order Taylor expansion (\citeproc{ref-gassmoller2020}{Gassmöller
et al., 2020}; \citeproc{ref-jarvis1980}{Jarvis \& Mckenzie, 1980}):

\begin{equation}
  \rho \approx \bar{\rho} + \left(\frac{\partial \bar{\rho}}{\partial P} \right)_T \Delta P + \left(\frac{\partial \bar{\rho}}{\partial T} \right)_P \Delta T
  \label{eq:density-ala-expansion}
\end{equation}

Equation \ref{eq:density-ala-expansion} is rewritten using standard
thermodynamic relations
\(\beta = \frac{1}{\rho} \left(\frac{\partial \rho}{\partial P}\right)_T\)
and
\(\alpha = -\frac{1}{\rho} \left(\frac{\partial \rho}{\partial T}\right)_P\)
to obtain the expression:

\begin{equation}
  \rho \approx \bar{\rho} \left(1 + \bar{\beta}\, \hat{P} - \bar{\alpha}\, \hat{T} \right)
  \label{eq:density-ala}
\end{equation}

where \(\bar{\rho}\), \(\, \bar{\beta}\), \(\, \bar{\alpha}\), are the
adiabatic reference density, compressibility, and thermal expansivity,
respectively, and \(\Delta P = \hat{P} = P - \bar{P}\) and
\(\Delta T = \hat{T} = T - \bar{T}\) are the dynamic PT. Note that the
reference thermal conductivity \(\bar{k}\) = 4.0 W m\(^{-1}\)K\(^{-1}\)
is constant in all our numerical experiments.

\paragraph{Reaction Kinetics}\label{sec:reaction-kinetics}

The kinetics of the olivine \(\Leftrightarrow\) wadsleyite phase
transition were governed entirely by interface-controlled growth, as
nucleation was assumed to saturate rapidly and did not limit the
reaction (\citeproc{ref-cahn1956}{Cahn, 1956}). Following Faccenda \&
Dal Zilio (\citeproc{ref-faccenda2017}{2017}), the transformed volume
fraction is given by:

\begin{equation}
  X = 1 - \exp\!\left(-N\, \dot{x}\, t \right)
  \label{eq:volume-fraction}
\end{equation}

where \(X\) is the volume fraction of the product phase (olivine or
wadsleyite), \(N\) is a geometric factor that accounts for nucleation
sites, \(\dot{x}\) is the growth rate, and \(t\) is the elapsed time
after site saturation. For inter-crystalline grain-boundary controlled
growth, \(N = 6.67/d\), where \(d\) is grain size.

Since we assumed interface-controlled growth kinetics, the following
expression determined the overall reaction rate
(\citeproc{ref-hosoya2005}{Hosoya et al., 2005}):

\begin{equation}
  \dot{x} = \Gamma\, T\, C_\mathrm{OH}^n\, \exp\!\left(-\frac{H^{\ast} + P V^{\ast}}{R\, T}\right) \left(1 - \exp\!\left[-\frac{\Delta G}{R\, T}\right] \right)
  \label{eq:growth-rate}
\end{equation}

where \(\Gamma\) is the growth rate prefactor, \(C_\mathrm{OH}\) is the
concentration of water in the reactant phase, \(n\) is the water content
exponent, \(H^{\ast}\) is activation enthalpy, \(V^{\ast}\) is
activation volume, \(P\) is pressure, \(T\) is temperature, \(R\) is the
gas constant, and \(\Delta G\) is the Gibbs free energy difference
between olivine and wadsleyite, which is approximated by:

\begin{equation}
  \Delta G \approx \Delta \bar{G} + \hat{P}\, \Delta \bar{V} - \hat{T}\, \Delta \bar{S}
  \label{eq:gibbs}
\end{equation}

where \(\Delta \bar{G}\), \(\Delta \bar{V}\), and \(\Delta \bar{S}\) are
the molar Gibbs free energy, volume, and entropy differences between
olivine and wadsleyite along the adiabatic reference profile (Figure
\ref{fig:thermodynamic-property-profile}), respectively, and \(\hat{P}\)
and \(\hat{T}\) are the dynamic PT.

In this formulation, the time evolution of the olivine
\(\Leftrightarrow\) wadsleyite phase transition is fully described by
the interplay of pressure, temperature, and kinetic parameters applied
to the interface-controlled growth model, without explicit consideration
of nucleation kinetics (\citeproc{ref-faccenda2017}{Faccenda \& Dal
Zilio, 2017}; \citeproc{ref-hosoya2005}{Hosoya et al., 2005}). The
macro-scale olivine \(\Leftrightarrow\) wadsleyite reaction rate was
therefore computed by taking the time derivative of Equation
\ref{eq:volume-fraction}:

\begin{equation}
  \frac{dX}{dt} = \dot{X} = N\, \dot{x}\, \left(1 - X \right)
  \label{eq:reaction-rate-short}
\end{equation}

Since all of the kinetic parameters \(N\), \(\Gamma\), and
\(C_\mathrm{OH}^n\) ultimately scale the reaction rate (Equations
\ref{eq:growth-rate}--\ref{eq:reaction-rate-short}), varying them
independently adds little scientific value. Instead, we simplified our
numerical implementation of Equation \ref{eq:reaction-rate-short} by
combining the parameters \(N\), \(\Gamma\), and \(C_\mathrm{OH}^n\) into
a single kinetic prefactor
\(Z = \frac{6.67}{d}\, \Gamma\, C_\mathrm{OH}^n\). Thus, the full
expression for the reaction rate became:

\begin{equation}
  \dot{X} = Z\, T\, \exp\!\left(-\frac{H^{\ast} + P V^{\ast}}{R\, T}\right) \left(1 - \exp\!\left[-\frac{\Delta G}{R\, T}\right] \right)\, \left(1 - X \right)
  \label{eq:reaction-rate}
\end{equation}

The range of kinetic prefactors \(Z\) used in our numerical experiments
(3.0e0--7.0e7 K s\(^{-1}\)) was determined by holding \(\Gamma\) =
\(\exp\!\left(-18\right)\) m s\(^{-1}\) K\(^{-1}\)
ppm\(_\mathrm{OH}^{-n}\), \(H^{\ast}\) = 274 kJ mol\(^{-1}\),
\(V^{\ast}\) = 3.0e-6 m\(^3\) mol\(^{-1}\), and \(n\) = 3.2 constant,
while varying water content \(C_\mathrm{OH}\) from 50--5000 ppm and
grain size \(d\) from 1--10 mm. These water contents and grain sizes are
consistent with the experimental conditions of Hosoya et al.
(\citeproc{ref-hosoya2005}{2005}), previous numerical studies of
metastable olivine wedges (\citeproc{ref-rubie1994}{Rubie \& Ross II,
1994}), and typical grain sizes of upper mantle xenoliths
(\textasciitilde3--10 mm, \citeproc{ref-karato1984}{Karato, 1984};
\citeproc{ref-karato2008}{Karato, 2008}). Thus, our experiments
approximate kinetic conditions ranging from slow kinetics in dry rocks
with large grain sizes (50 ppm OH; 10 mm; \(Z\) = 3.0e0 K s\(^{-1}\)) to
fast kinetics in hydrated rocks with small grain sizes (5000 ppm OH; 1
mm; \(Z\) = 7.0e7 K s\(^{-1}\)).

\begin{figure}
\centering
\includegraphics[width=0.8\linewidth,height=\textheight,keepaspectratio,alt={Reference thermodynamic properties used in our ASPECT simulations. Profiles were computed using the same methods as described in Figure  (see Section ).}]{../figs/thermodynamic-property-profile.png}
\caption{Reference thermodynamic properties used in our ASPECT
simulations. Profiles were computed using the same methods as described
in Figure \ref{fig:material-property-profile} (see Section
\ref{sec:adiabatic-reference-conditions}).}\label{fig:thermodynamic-property-profile}
\end{figure}

\paragraph{Operator Splitting}\label{sec:operator-splitting}

Since the reaction rate \(\dot{X}\) was faster than the advection
timescale in our ASPECT simulations, we employed a first-order operator
splitting scheme to decouple advection from interface-controlled growth
kinetics. In this approach, the transformed volume fraction \(X\) was
updated in two sequential steps within each overall time step
\(\Delta t\):

\begin{equation}
  \frac{\partial X}{\partial t} + \vec{u} \cdot \nabla X = 0
  \label{eq:composition}
\end{equation}

\begin{enumerate}
\def\labelenumi{\arabic{enumi}.}
\tightlist
\item
  \textbf{Reaction step:} Starting from \(X^{t}\), integrate Equation
  \ref{eq:reaction-rate} over the time interval \(\Delta t\) using a
  smaller sub-step \(\delta t \le \Delta t\) to obtain an intermediate
  composition \(X^\ast\)
\item
  \textbf{Advection step:} Starting from \(X^\ast\), solve the transport
  of material \(\left(\vec{u} \cdot \nabla X \right)\) without phase
  changes over the same time interval to yield the updated composition
  \(X^{t+1}\)
\end{enumerate}

In our simulations, the reaction step was solved using ASPECT's ARKode
solver, which employs an adaptive-step additive Runge--Kutta method
(\citeproc{ref-aspectmanual}{Bangerth et al., 2024b}). In this scheme,
the reaction substep size \(\delta t\) was not prescribed explicitly.
Instead, it was determined dynamically by ARKode to satisfy a specified
relative tolerance during the reaction step (set to \(10^{-6}\) in this
study). This adaptive integration ensures that fast solid-state reaction
kinetics were accurately captured without imposing overly restrictive
global timesteps.

\subsubsection{Rheological Model}\label{sec:rheological-model}

We use a temperature-dependent viscosity following an Arrhenius law:

\begin{equation}
  \eta = A \exp\!\left(\frac{E^{\ast}}{R\,T}\right)
  \label{eq:arrhenius-viscosity}
\end{equation}

with viscosity prefactor \(A\), activation energy \(E^{\ast}\), gas
constant \(R\), and absolute temperature \(T\), which is decomposed into
an adiabatic reference temperature \(\bar{T}\) and a perturbation
\(\hat{T}\), \(T=\bar{T}+\hat{T}\). Linearizing the Arrhenius relation
through a first-order Taylor expansion about \(\bar{T}\) yields:

\begin{equation}
  \eta \approx A \exp\!\left(\frac{E^{\ast}}{R\,\bar{T}}\right) \exp\!\left(-\frac{E^{\ast}}{R\,\bar{T}}\frac{\hat{T}}{\bar{T}}\right)
  \label{eq:arrhenius-viscosity-expanded}
\end{equation}

By defining a reference background viscosity as:

\begin{equation}
  \bar{\eta} = A \exp\!\left(\frac{E^{\ast}}{R\,\bar{T}}\right)
  \label{eq:background-viscosity}
\end{equation}

we arrive at a rheological model where viscosity varies exponentially
with thermal perturbations about an adiabatic reference profile:

\begin{equation}
  \eta \approx \bar{\eta}\,\exp\!\left(-B\,\frac{\hat{T}}{\bar{T}}\right) \qquad B=\frac{E^{\ast}}{R\,\bar{T}}
  \label{eq:rheological-model}
\end{equation}

In our simulations, we prescribe a uniform background viscosity
\(\bar{\eta}\) = 10\(^{21}\) Pa s throughout the upper mantle (e.g.,
\citeproc{ref-karato2008}{Karato, 2008};
\citeproc{ref-ranalli1995}{Ranalli, 1995}) and vary the rheological
activation factor \(B\) between 2 (low thermal sensitivity) and 10 (high
thermal sensitivity). Thus, our numerical experiments explore a range of
viscosity contrasts between thermal anomalies (slabs and plumes) and
background adiabatic reference conditions.

\subsubsection{Numerical Stabilization of Dynamic Pressure
Oscillations}\label{sec:numerical-stability}

A significant numerical limitation emerges from the coupled feedback
between our kinetic rate law (Equations
\ref{eq:volume-fraction}--\ref{eq:reaction-rate}) and the buildup of
dynamic pressure in the fully compressible continuity equation (Equation
\ref{eq:continuity-compressible}). At sufficiently high rheological
contrasts (\(B\) \(\gtrsim\) 4), cold slabs develop internal dynamic
pressures that can exceed several hundred MPa. These dynamic pressure
perturbations accelerate the forward reaction (through the Gibbs free
energy term in Equation \ref{eq:reaction-rate}), causing rapid local
density changes. These density changes then alter the pressure field
through the coupled continuity and momentum equations (Equations
\ref{eq:navier-stokes-no-inertia}--\ref{eq:continuity-compressible}),
which in turn drives the reverse reaction. This positive feedback loop
manifests as spurious pressure waves propagating through the slab
interior.

To address this issue, we adopt an approach similar to Gassmöller et al.
(\citeproc{ref-gassmoller2020}{2020}) and exclude the dynamic pressure
contribution from the Gibbs free energy calculation (Equation
\ref{eq:gibbs}) while retaining it in the Arrhenius term in Equation
\ref{eq:reaction-rate} and density formulation (Equation
\ref{eq:density-ala}). In practice, this means replacing
\(\Delta G \approx \Delta \bar{G} + \hat{P}\, \Delta \bar{V} - \hat{T}\, \Delta \bar{S}\)
with \(\Delta G \approx \Delta \bar{G} - \hat{T}\, \Delta \bar{S}\) in
our kinetic rate law. As discussed by Gassmöller et al.
(\citeproc{ref-gassmoller2020}{2020}), this approximation is justified
because density variations from dynamic pressure are typically small
compared to those from compositional heterogeneities and temperature
variations in mantle convection. However, this treatment does introduce
a limitation: in scenarios where dynamic pressure effects dominate over
thermal and compositional contributions, such as in regions with extreme
viscosity contrasts or near phase boundaries with large
\(\Delta \bar{V}\), our kinetic model may underestimate the
thermodynamic driving force for phase transitions. Nevertheless, this
compromise ensures numerical stability while preserving the essential
physics of kinetically controlled phase transitions coupled to
compressible flow.

\section{Results}\label{sec:results}

\subsection{Simulation Snapshots: Slabs and
Plumes}\label{sec:simulation-snapshots}

Figures \ref{fig:slab-composition-set2} and
\ref{fig:plume-composition-set2} illustrate how reaction rates impact
dynamic flow patterns and shape the 410 discontinuity. These snapshots,
taken after 100 Ma of evolution, provide visual context for the
quantitative analysis in Section \ref{sec:410-displacement-width}.

In slab simulations, ultra-sluggish kinetics (Figure
\ref{fig:slab-composition-set2}: top row) allow metastable olivine to
persist deep into the transition zone. This inhibition causes the slab
to stagnate and pond, depressing the 410. Within the cold, metastable
olivine region, Gibbs free energy accumulates and wadsleyite saturation
remains low until the thermodynamic driving force overcomes kinetic
barriers. Once this threshold is reached, the olivine
\(\Leftrightarrow\) wadsleyite reaction rapidly completes, producing a
sharp 410 that is displaced downwards by tens of kilometers.

At intermediate reaction rates (Figure \ref{fig:slab-composition-set2}:
middle row), the olivine \(\Leftrightarrow\) wadsleyite reaction still
lags but is fast enough to limit widespread olivine metastability and
avoid total slab stagnation. The resulting 410 is broad and diffuse, as
density and seismic velocity contrasts gradually fade with depth. This
moderately-sluggish kinetic regime produces complex 410 structures
through intermediate reaction rates, incomplete slab stagnation, and
deflected flow patterns. However, when reaction rates are sufficiently
fast to maintain quasi-equilibrium conditions (Figure
\ref{fig:slab-composition-set2}: bottom row), the 410 sharpens and
shifts upwards as expected from equilibrium thermodynamics. Under this
fast kinetic regime, rapid wadsleyite growth within the slab allows
continuous slab descent through the 410 without hesitation.

In plume simulations, thermal effects dominate mantle flow dynamics and
410 structure. Even under ultra-sluggish kinetics (Figure
\ref{fig:plume-composition-set2}: top row), the high temperatures of
upwellings prevent significant olivine metastability. The olivine
\(\Leftrightarrow\) wadsleyite transition proceeds rapidly, maintaining
thin, sharp 410 interfaces and strong density and seismic contrasts.
Although ultra-sluggish kinetics slightly broaden and uplift the 410,
reducing buoyancy contrasts, plume structures remain vertically coherent
across the full range of tested kinetic prefactors \(Z\) (Figure
\ref{fig:plume-composition-set2}).

Altogether, these simulations demonstrate that in cold environments,
kinetics strongly influence slab dynamics and control whether the 410
appears as a diffuse, low-amplitude feature or as a sharp, high-contrast
seismic boundary. In contrast, thermal effects dominate in hot plume
environments, producing stable, sharply defined 410s that are largely
independent of tested kinetic prefactors.

\begin{figure}
\centering
\includegraphics[width=1\linewidth,height=\textheight,keepaspectratio,alt={Slab simulations with intermediate strength contrasts (B = 4) showing ultra-sluggish (top row: Z = 3.0e0 K s\^{}\{-1\}), intermediate (middle row: Z = 4.7e2 K s\^{}\{-1\}), and quasi-equilibrium (bottom row: Z = 7.0e7 K s\^{}\{-1\}) kinetic regimes after 100 Ma evolution. Panels show dynamic temperature \textbackslash hat\{T\} (left column), dynamic density \textbackslash hat\{\textbackslash rho\} (middle column), and pressure-wave velocity V\_p (right column).}]{../figs/simulation/compositions/slab-Z3.0e0-B4-Z4.7e2-B4-Z7.0e7-B4-set2-composition-0010.png}
\caption{Slab simulations with intermediate strength contrasts (\(B\) =
4) showing ultra-sluggish (top row: \(Z\) = 3.0e0 K s\(^{-1}\)),
intermediate (middle row: \(Z\) = 4.7e2 K s\(^{-1}\)), and
quasi-equilibrium (bottom row: \(Z\) = 7.0e7 K s\(^{-1}\)) kinetic
regimes after 100 Ma evolution. Panels show dynamic temperature
\(\hat{T}\) (left column), dynamic density \(\hat{\rho}\) (middle
column), and pressure-wave velocity \(V_p\) (right
column).}\label{fig:slab-composition-set2}
\end{figure}

\begin{figure}
\centering
\includegraphics[width=1\linewidth,height=\textheight,keepaspectratio,alt={Plume simulations with intermediate strength contrasts (B = 4) showing ultra-sluggish (top row: Z = 3.0e0 K s\^{}\{-1\}), intermediate-sluggish (middle row: Z = 4.7e2 K s\^{}\{-1\}), and quasi-equilibrium (bottom row: Z = 7.0e7 K s\^{}\{-1\}) kinetic regimes after 100 Ma evolution. Panels show dynamic temperature \textbackslash hat\{T\} (left column), dynamic density \textbackslash hat\{\textbackslash rho\} (middle column), and pressure-wave velocity V\_p (right column).}]{../figs/simulation/compositions/plume-Z3.0e0-B4-Z4.7e2-B4-Z7.0e7-B4-set2-composition-0010.png}
\caption{Plume simulations with intermediate strength contrasts (\(B\) =
4) showing ultra-sluggish (top row: \(Z\) = 3.0e0 K s\(^{-1}\)),
intermediate-sluggish (middle row: \(Z\) = 4.7e2 K s\(^{-1}\)), and
quasi-equilibrium (bottom row: \(Z\) = 7.0e7 K s\(^{-1}\)) kinetic
regimes after 100 Ma evolution. Panels show dynamic temperature
\(\hat{T}\) (left column), dynamic density \(\hat{\rho}\) (middle
column), and pressure-wave velocity \(V_p\) (right
column).}\label{fig:plume-composition-set2}
\end{figure}

\subsection{Structure of the 410: Displacement and
Width}\label{sec:410-displacement-width}

Figure \ref{fig:410-structure} summarizes the quantitative relationships
between 410 structure and the maximum reaction rate
\(\dot{X}_{\mathrm{max}}\) evaluated in slab and plume simulations after
100 Ma of evolution. The results reveal fundamentally different
responses of plumes and slabs to reaction kinetics. See the
Supplementary Information for the full set of experimental results and
technical details describing the 410 structural measurements.

\begin{figure}
\centering
\includegraphics[width=0.65\linewidth,height=\textheight,keepaspectratio,alt={Measured 410 displacement and width versus maximum reaction rates \textbackslash dot\{X\}\_\{\textbackslash mathrm\{max\}\} in plume and slab simulations with intermediate strength contrasts (B = 4) after 100 Ma. Structure of the 410 near plumes (left column) shows minimal dependence on \textbackslash dot\{X\}\_\{\textbackslash mathrm\{max\}\}, with both displacement and width remaining nearly constant across seven orders of magnitude variation in \textbackslash dot\{X\}\_\{\textbackslash mathrm\{max\}\}. In contrast, 410 structure near slabs (right column) changes distinctly across three kinetic regimes: (1) quasi-equilibrium at high \textbackslash dot\{X\}\_\{\textbackslash mathrm\{max\}\}, where 410 widths are narrow and elevated; (2) an intermediate regime where decreasing reaction rates \textbackslash dot\{X\}\_\{\textbackslash mathrm\{max\}\} progressively widen and deepen the 410; and (3) an ultra-sluggish regime at low \textbackslash dot\{X\}\_\{\textbackslash mathrm\{max\}\}, where the 410 narrows while deepening, and slabs completely stall and pond.}]{../figs/410-structure-B4.png}
\caption{Measured 410 displacement and width versus maximum reaction
rates \(\dot{X}_{\mathrm{max}}\) in plume and slab simulations with
intermediate strength contrasts (\(B\) = 4) after 100 Ma. Structure of
the 410 near plumes (left column) shows minimal dependence on
\(\dot{X}_{\mathrm{max}}\), with both displacement and width remaining
nearly constant across seven orders of magnitude variation in
\(\dot{X}_{\mathrm{max}}\). In contrast, 410 structure near slabs (right
column) changes distinctly across three kinetic regimes: (1)
quasi-equilibrium at high \(\dot{X}_{\mathrm{max}}\), where 410 widths
are narrow and elevated; (2) an intermediate regime where decreasing
reaction rates \(\dot{X}_{\mathrm{max}}\) progressively widen and deepen
the 410; and (3) an ultra-sluggish regime at low
\(\dot{X}_{\mathrm{max}}\), where the 410 narrows while deepening, and
slabs completely stall and pond.}\label{fig:410-structure}
\end{figure}

In plume simulations, the 410 shows little dependence on
\(\dot{X}_{\mathrm{max}}\). Its structure remains nearly constant across
seven orders of magnitude variation in \(\dot{X}_{\mathrm{max}}\), with
consistent displacements of 24 km and widths between 2--9 km. The only
exception is a few ultra-sluggish kinetic models where displacements
decrease to 18--21 km and widths increase to 12--21 km. The weak
dependence of 410 structure on \(\dot{X}_{\mathrm{max}}\) reflects the
strong thermal control of the reaction front in upwellings, where high
temperatures promote rapid wadsleyite \(\Leftrightarrow\) olivine
transition---maintaining a sharp discontinuity regardless of the kinetic
prefactor \(Z\) applied to the interface-controlled growth model
(Equation \ref{eq:reaction-rate}). Similarly, variations in rheological
strength contrast (controlled by the \(B\) parameter in Equation
\ref{eq:rheological-model}) produce negligible changes to plume
morphology or 410 structure, as the thermally-dominated behavior is
largely insensitive to viscosity variations (see Supplementary
Information for examples).

In slab simulations, the 410 exhibits distinct structural changes across
three kinetic regimes. At high reaction rates (\(Z\) \(\gtrsim\) 1.8e5 K
s\(^{-1}\); \(\dot{X}_{\mathrm{max}}\) \(\gtrsim\) 2.2 Ma\(^{-1}\)), the
olivine \(\Leftrightarrow\) wadsleyite transition remains near
thermodynamic equilibrium, producing a narrow 410 (\(\lesssim\) 10 km),
displaced 27--39 km upwards within the slab's inner core. As
\(\dot{X}_{\mathrm{max}}\) decreases (2.0e2 \(\lesssim\) \(Z\)
\(\lesssim\) 1.8e5 K s\(^{-1}\); 0.07 \(\lesssim\)
\(\dot{X}_{\mathrm{max}}\) \(\lesssim\) 2.2 Ma\(^{-1}\)), the 410
deepens and widens, approximating a log-linear relationship where
reductions in \(\dot{X}_{\mathrm{max}}\) progressively broaden the
reaction front. This intermediate kinetic regime corresponds to a
partially inhibited olivine \(\Leftrightarrow\) phase transition that
proceeds slowly, hindering downward flow without complete slab
stagnation.

At the lowest reaction rates (\(Z\) \(\lesssim\) 2.0e2 K s\(^{-1}\);
\(\dot{X}_{\mathrm{max}}\) \(\lesssim\) 0.07 Ma\(^{-1}\)), a third
kinetic regime emerges. While the 410 is displaced downwards, its width
narrows with further reductions in \(\dot{X}_{\mathrm{max}}\). This
ultra-sluggish kinetic regime reflects a transition to strong
disequilibrium conditions and complete slab stagnation. As metastable
olivine ponds and is pushed deeper, increasing pressure drives growing
thermodynamic disequilibrium, eventually triggering rapid transformation
within a narrow depth range while surrounding regions remain kinetically
frozen. The apparent 410 sharpening results from this
pressure-controlled reaction front; only material pushed sufficiently
deep accumulates enough thermodynamic driving force to react, producing
a sharp seismic discontinuity where this critical pressure is reached.

Rheological strength contrasts substantially modulate slab dynamics and
410 structure through interactions with kinetic effects (Figure
\ref{fig:410-structure-comp}). Higher \(B\) values in Equation
\ref{eq:rheological-model} increase the viscosity contrast between the
cold slab and surrounding mantle, producing a more coherent slab that
penetrates the 410 with less internal deformation. Stronger slabs sink
more sub-horizontally through the 410, limiting their vertical descent
rates. Therefore, as \(B\) increases, the 410 sharpens because material
traverses the phase transition zone more slowly, allowing greater time
for reaction progress despite sluggish kinetics. In contrast, lower
\(B\) values permit greater internal deformation and faster vertical
descent, which amplifies kinetic effects and produces broader phase
transition zones (see Supplementary Information for examples).

\begin{figure}
\centering
\includegraphics[width=0.95\linewidth,height=\textheight,keepaspectratio,alt={Variation in 410 structure and slab descent across kinetic and rheological regimes. Panels show 410 displacement (left), 410 width (middle), and maximum vertical velocity (right) as functions of the kinetic prefactor Z (horizontal axis, log scale) and rheological activation factor B (vertical axis). Each colored tile represents the measured value within a simulation after 100 Ma, and black/white lines delineate transitions between regime behaviors. The precise location of these transitions depends on B, where increasing rheological contrast progressively shifts the regime boundaries towards lower Z (more sluggish kinetic conditions). Text labels highlight qualitative regimes inferred from the trends. The complete dataset is given in the Supplementary Information.}]{../figs/410-structure-comp.png}
\caption{Variation in 410 structure and slab descent across kinetic and
rheological regimes. Panels show 410 displacement (left), 410 width
(middle), and maximum vertical velocity (right) as functions of the
kinetic prefactor \(Z\) (horizontal axis, log scale) and rheological
activation factor \(B\) (vertical axis). Each colored tile represents
the measured value within a simulation after 100 Ma, and black/white
lines delineate transitions between regime behaviors. The precise
location of these transitions depends on \(B\), where increasing
rheological contrast progressively shifts the regime boundaries towards
lower \(Z\) (more sluggish kinetic conditions). Text labels highlight
qualitative regimes inferred from the trends. The complete dataset is
given in the Supplementary Information.}\label{fig:410-structure-comp}
\end{figure}

In summary, 410 structure near plumes is regulated by thermal effects
near thermodynamic equilibrium, whereas 410 structure near slabs
exhibits distinct kinetic thresholds and non-linear scaling between its
width, displacement, and the reaction rate \(\dot{X}\). These
contrasting behaviors underscore the differing roles of kinetics in hot
versus cold mantle environments and imply that the 410 beneath slabs can
transition abruptly between thermodynamically- and
kinetically-controlled regimes as reaction rates decrease.

\section{Discussion}\label{sec:discussion}

Our numerical simulations show that reaction kinetics exert a
first-order control on the structure and seismic expression of the 410
discontinuity. The results presented in Section \ref{sec:results}
demonstrate that plume and slab dynamics respond in systematically
different ways: plumes are insensitive to kinetics due to high
temperatures, whereas slabs show three distinct kinetic regimes with
thresholded behavior (Figure \ref{fig:410-structure}). Rheological
strength contrasts, controlled by the activation factor \(B\), further
modulate these kinetic effects by altering slab geometry and descent
rate (Figure \ref{fig:410-structure-comp}). The implications of such
contrasting relationships in slabs versus plumes are discussed below.

\subsection{Uncertainties and Model
Limitations}\label{sec:uncertainties-and-model-limitations}

The primary quantitative uncertainty in our analysis stems from the
kinetic prefactor \(Z\) in the interface-controlled growth model
(Equation \ref{eq:reaction-rate}), which spans several orders of
magnitude reflecting variable water contents (50--5000 ppm) and grain
sizes (1--10 mm). Laboratory studies also reveal large uncertainties in
kinetic parameters \(n\), \(\Gamma\), \(H^{\ast}\), and \(V^{\ast}\)
that depend strongly on water content, grain size, Mg-Fe composition,
and microstructural evolution (\citeproc{ref-hosoya2005}{Hosoya et al.,
2005}; \citeproc{ref-kubo2004}{Kubo et al., 2004};
\citeproc{ref-ledoux2023}{{Ledoux et al.}, 2023};
\citeproc{ref-liu1998}{Liu et al., 1998};
\citeproc{ref-perrillat2013}{Perrillat et al., 2013};
\citeproc{ref-rubie1994}{Rubie \& Ross II, 1994}). Our simulations
therefore explore only a limited subset of potential reaction rates in
Earth's upper mantle.

Our kinetic model also assumes instantaneous nucleation site saturation
followed by interface-controlled growth (Equations
\ref{eq:volume-fraction}--\ref{eq:reaction-rate}), thereby neglecting
nucleation kinetics. While often justified because nucleation rates
occur too rapidly for reliable measurement
(\citeproc{ref-faccenda2017}{Faccenda \& Dal Zilio, 2017};
\citeproc{ref-hosoya2005}{Hosoya et al., 2005};
\citeproc{ref-kubo2004}{Kubo et al., 2004};
\citeproc{ref-perrillat2016}{Perrillat et al., 2016}), recent in-situ
X-ray and acoustic studies (\citeproc{ref-ledoux2023}{{Ledoux et al.},
2023}; \citeproc{ref-ohuchi2022}{Ohuchi et al., 2022}) document complex
nucleation-growth microstructures that can limit net reaction rates
under some PT conditions. Our saturated nucleation assumption therefore
generally overestimates reaction rates and underestimates olivine
metastability and its effects on flow dynamics and 410 structure.

Compositional and rheological simplifications further limit our results.
Assuming pure Mg-rich end-members neglects Fe-partitioning and
minor-element effects that shift equilibrium depths by
\textasciitilde10--20 km and alter kinetics
(\citeproc{ref-katsura2004}{{Katsura et al.}, 2004};
\citeproc{ref-perrillat2013}{Perrillat et al., 2013},
\citeproc{ref-perrillat2016}{2016}). We also neglect non-hydrostatic
stress effects on microstructures and solid-state reactions
(\citeproc{ref-wheeler2014}{Wheeler, 2014},
\citeproc{ref-wheeler2018}{2018}, \citeproc{ref-wheeler2020}{2020}), and
our simple temperature-dependent viscosity omits grain-size evolution,
plastic deformation, and stress-dependent rheologies that modify strain
localization (\citeproc{ref-karato2001}{Karato et al., 2001}) and impact
flow dynamics. Despite these limitations, our results capture
first-order effects governing 410 structure.

\subsection{Implications for Subduction
Dynamics}\label{sec:implications-for-subduction-dynamics}

Three kinetic regimes identified in our simulations---quasi-equilibrium,
intermediate, and ultra-sluggish---arise from feedbacks between
kinetically controlled phase transitions and slab strength (Figure
\ref{fig:410-structure-comp}). The absence of widespread slab ponding at
the 410 in seismic tomography (\citeproc{ref-fukao2013}{Fukao \&
Obayashi, 2013}) constrains plausible kinetic conditions: the
ultra-sluggish regime (\(\dot{X}\) \(\lesssim\) 0.07 Ma\(^{-1}\)),
producing complete stagnation, appears inconsistent with global
observations of continuous slab descent through the 410. Most subduction
zones must therefore experience sufficiently rapid olivine
\(\Leftrightarrow\) wadsleyite transformation to avoid complete
stagnation at the 410.

However, deep earthquakes attributed to transformational faulting
(\citeproc{ref-green1995}{Green \& Houston, 1995};
\citeproc{ref-ishii2021}{Ishii \& Ohtani, 2021};
\citeproc{ref-kirby1996}{Kirby et al., 1996};
\citeproc{ref-ohuchi2022}{Ohuchi et al., 2022};
\citeproc{ref-sindhusuta2025}{Sindhusuta et al., 2025}) require
significant metastable olivine persistence, suggesting the intermediate
kinetic regime (0.07 \(\lesssim\) \(\dot{X}\) \(\lesssim\) 2.2
Ma\(^{-1}\)) characterizes many subduction zones. This regime generates
localized buoyant regions that resist but do not prevent downward flow.
Assuming that transformational faulting is primarily responsible for
deep earthquakes, rather than alternative mechanisms (e.g.,
\citeproc{ref-zhan2020}{Zhan, 2020}), coexistence of deep seismicity
with continued slab penetration requires a delicate balance: reaction
rates must be slow enough to sustain metastable volumes for
transformational faulting, yet fast enough to permit 410 penetration.

Rheological contrasts further modulate kinetic effects by controlling
slab trajectory and descent rate. Strong slabs (\(B\) \(\gtrsim\) 6)
maintain coherence and penetrate sub-horizontally, spending more time
within the phase transition zone where olivine transforms more
completely. Weaker slabs (\(B\) \(\lesssim\) 6) deform internally and
descend vertically, traversing the transition zone quickly with less
time for sluggish reactions. Thus under equivalent kinetic conditions,
stronger slabs experience more slab pull and continuous descent, while
weaker slabs accumulate more metastable olivine, amplifying buoyancy
forces that may slow or arrest descent.

A critical kinetic threshold near \(\dot{X}\) \(\sim\) 0.07 Ma\(^{-1}\)
marks where buoyancy forces from incomplete olivine \(\Leftrightarrow\)
wadsleyite transformation either overwhelm or permit slab pull. This
narrow threshold shifts with slab strength, where coherent slabs
penetrate the 410 at reaction rates that would stall weaker slabs
(Figure \ref{fig:410-structure-comp}). These dynamic sensitivities
suggest individual subduction zones could oscillate between penetration
and temporary stagnation as kinematic and PT conditions evolve (e.g.,
\citeproc{ref-agrusta2017}{Agrusta et al., 2017}).

Regional variability in slab behavior (\citeproc{ref-fukao2013}{Fukao \&
Obayashi, 2013}) could therefore reflect diversity in both kinetics and
rheology, in addition to slab age and convergence rate. Young, hot slabs
with lower viscosity contrasts descend steeply and accumulate moderate
metastable olivine despite warm thermal structures, while old, cold
slabs with high viscosity contrasts flatten and sink slowly, allowing
near-complete transformation under moderately sluggish kinetics. These
patterns distinguish plumes from slabs: in hot upwellings, elevated
temperatures suppress sensitivity to both kinetics and rheology, while
in cold slabs, kinetics govern reaction rates and rheology dictates
reaction duration by regulating slab kinematics. Accurately modeling
slab morphology, material exchange, and deep stress conditions in
geodynamic simulations therefore requires incorporating both effects.

\subsection{Implications for 410
Detectability}\label{sec:410-detectability}

Sharp interfaces with widths of a few kilometers are readily detected
with SS precursors and receiver function stacks
(\citeproc{ref-chambers2005}{Chambers et al., 2005};
\citeproc{ref-deuss2009}{Deuss, 2009};
\citeproc{ref-shearer2000}{Shearer, 2000}), and regional high-frequency
approaches can resolve structures as thin as \textasciitilde5 km
(\citeproc{ref-dokht2016}{Dokht et al., 2016};
\citeproc{ref-frazer2023}{Frazer \& Park, 2023};
\citeproc{ref-helffrich1996}{Helffrich \& Wood, 1996};
\citeproc{ref-wei2017}{Wei \& Shearer, 2017}). Our plume simulations
predict consistently thin 410 discontinuities (2--3 km widths,
\textasciitilde24 km displacements) across large rheological variations
and seven orders of magnitude in \(\dot{X}\), except under
ultra-sluggish kinetics where widths broaden up to 21 km. These sharp
discontinuities should be readily detectable, consistent with
observations of well-defined 410s beneath hotspots
(\citeproc{ref-deuss2009}{Deuss, 2009};
\citeproc{ref-lawrence2008}{Lawrence \& Shearer, 2008}).

Slab simulations, in contrast, imply that 410 detectability emerges from
joint kinetic and rheological influences (Figure
\ref{fig:410-structure-comp}). Strong slabs (\(B\) \(\gtrsim\) 6)
descending slowly along sub-horizontal trajectories maximize residence
time in the phase transition zone, producing sharp, detectable 410
signals due to near-complete olivine \(\Leftrightarrow\) wadsleyite
transformation. Weak slabs (\(B\) \(\lesssim\) 6) descending steeply
with reduced residence times amplify kinetic inhibition, producing
broader reaction fronts. Since tomography shows most slabs descending
steeply (\citeproc{ref-fukao2013}{Fukao \& Obayashi, 2013}),
moderate-to-low viscosity contrasts (\(B\) \(\lesssim\) 6) appear
common, suggesting kinetic effects should be amplified in many natural
subduction zones.

These systematic behaviors offer a quantitative framework for
interpreting observed seismic heterogeneity. Reported 410 thickness
variations of \textasciitilde5--30 km in Pacific regions
(\citeproc{ref-alex2004}{Alex Song et al., 2004};
\citeproc{ref-schmerr2007}{Schmerr \& Garnero, 2007}) can arise from
spatial variations in effective reaction rates, rheological contrasts,
or both, superimposed on thermal and compositional heterogeneity. Broad,
weakened signals beneath some subduction zones
(\citeproc{ref-han2021}{Han et al., 2021};
\citeproc{ref-jiang2015}{Jiang et al., 2015}; \citeproc{ref-lee2014}{Lee
et al., 2014}; \citeproc{ref-shen2020}{Shen \& Zhan, 2020};
\citeproc{ref-vanstiphout2019}{Van Stiphout et al., 2019}) are
consistent with intermediate kinetic conditions in relatively weak slabs
where steep, rapid descent amplifies metastability. Sharp 410s in cold
slabs suggest either quasi-equilibrium kinetics or slow sub-horizontal
descent permitting complete transformation despite sluggish rates.

The intermediate kinetic regime presents a particular challenge for
detecting the 410. Where 410 widths exceed \textasciitilde10--20 km,
gradual density and velocity gradients produce weak or absent signals in
SS precursors and receiver function stacks, despite substantial olivine
\(\Leftrightarrow\) transformation. Such ``invisible'' 410s could be
misinterpreted as compositional anomalies or unusual thermal structures
when they actually reflect kinetic inhibition. High-resolution
tomography imaging continuous velocity gradients may better detect these
diffuse transition zones, though distinguishing kinetic versus
rheological contributions requires independent constraints from
complementary geophysical observations.

\subsection{Implications for Constraining Kinetic and Rheological
Parameters from Seismic
Observations}\label{sec:constraining-kinetic-and-rheological-parameters-from-seismic-observations}

Contrasting sensitivities of plume and slab 410 structures suggest
different strategies for extracting parameter constraints from seismic
observations. For plumes, near-independence from both \(\dot{X}\) and
\(B\) limits utility for constraining either parameter. Consistent 410
widths (2--3 km) and displacements (\textasciitilde24 km) are primarily
controlled by thermal structure rather than kinetic rates or rheological
contrasts. Only ultra-sluggish kinetics produce distinguishable effects,
but this regime represents extreme inhibition unlikely to be widespread
in hot upwellings. Consequently, seismic observations beneath hotspots
primarily constrain plume temperature and geometry rather than
micro-scale kinetic or rheological parameters.

For slabs, three distinct kinetic regimes provide diagnostic signatures
that initially appear to offer stronger constraints. Widespread slab
penetration through the 410 in global tomography
(\citeproc{ref-fukao2013}{Fukao \& Obayashi, 2013}) provides a critical
first-order observation: the threshold near \(\dot{X}\) \(\sim\) 0.07
Ma\(^{-1}\) separating penetration from ponding requires that kinetic
conditions permit continuous descent. However, rheological effects
introduce ambiguity. Slabs with identical kinetic conditions can produce
dramatically different seismic signatures depending solely on their
vertical descent rates, which are controlled by rheological strength
contrasts (see Supplementary Information for examples).

This kinetic-rheological coupling creates parameter degeneracy.
Moderately sluggish kinetics (\(\dot{X}\) \(\gtrsim\) 0.07 Ma\(^{-1}\))
with stronger slabs (\(B\) \(\sim\) 8) may produce 410 structures
comparable to faster kinetics (\(\dot{X}\) \(\sim\) 2.2 Ma\(^{-1}\))
with weaker slabs (\(B\) \(\sim\) 4), as extended residence time
compensates for slower reaction rates. Regional 410 topography
variations could therefore reflect variations in kinetics (through
temperature, composition, water content, or grain size), variations in
rheology (through additional effects like accumulated strain), or
compensating variations in both. Without independent constraints,
seismic observations of 410 structure alone cannot uniquely separate
these effects.

Breaking this degeneracy requires combining multiple constraints.
High-resolution tomography (\citeproc{ref-fukao2013}{Fukao \& Obayashi,
2013}) together with kinematic and seismic data
(\citeproc{ref-lallemand2005}{Lallemand et al., 2005}) can independently
constrain dip angle and descent velocity, with steep continuous descent
suggesting weak rheology (\(B\) \(\lesssim\) 6) and thus brief residence
times that condition kinetic interpretations. Deep earthquakes
attributed to transformational faulting (\citeproc{ref-green1995}{Green
\& Houston, 1995}; \citeproc{ref-ishii2021}{Ishii \& Ohtani, 2021};
\citeproc{ref-kirby1996}{Kirby et al., 1996};
\citeproc{ref-ohuchi2022}{Ohuchi et al., 2022};
\citeproc{ref-sindhusuta2025}{Sindhusuta et al., 2025}) provide direct
evidence for metastable olivine, likely placing seismogenic regions in
the intermediate kinetic regime. Comparing 410 structure in seismogenic
versus aseismic segments with similar descent geometries can isolate
kinetic variations while controlling for rheological effects. Systematic
variations in slab thermal structure, age, and hydration across
subduction zones provide additional natural experiments
(\citeproc{ref-agius2017}{Agius et al., 2017};
\citeproc{ref-schmandt2012}{Schmandt, 2012};
\citeproc{ref-vanstiphout2019}{Van Stiphout et al., 2019})---if 410
structure varies with slab age inconsistently with compensating
kinetic-rheological effects, the dominant control can be identified.

Forward modeling using our framework (Figure
\ref{fig:410-structure-comp}) can test whether specific parameter
combinations match observed 410 structure from receiver functions or SS
precursors (e.g., \citeproc{ref-chambers2005}{Chambers et al., 2005};
\citeproc{ref-deuss2009}{Deuss, 2009}; \citeproc{ref-deuss2001}{Deuss \&
Woodhouse, 2001}; \citeproc{ref-houser2010}{Houser \& Williams, 2010};
\citeproc{ref-lawrence2008}{Lawrence \& Shearer, 2008};
\citeproc{ref-schmerr2007}{Schmerr \& Garnero, 2007}) given these
independent geometric and thermal constraints. However, until mineral
physics experiments (e.g., \citeproc{ref-hosoya2005}{Hosoya et al.,
2005}; \citeproc{ref-kubo2004}{Kubo et al., 2004};
\citeproc{ref-ledoux2023}{{Ledoux et al.}, 2023};
\citeproc{ref-perrillat2013}{Perrillat et al., 2013},
\citeproc{ref-perrillat2016}{2016}) reduce uncertainties, seismic
observations provide order-of-magnitude estimates of effective
\(\dot{X}\) and relative rheological strength rather than precise
determinations. Nevertheless, the threshold behavior and scaling
relationships from our simulations demonstrate that such comprehensive
multi-observation approaches are essential for constraining micro-scale
processes governing phase transitions and their geodynamic consequences.

\section{Conclusions}\label{sec:conclusions}

The olivine \(\Leftrightarrow\) wadsleyite phase transition and
resulting 410 structure are strongly influenced by coupled effects of
reaction kinetics and rheological strength on flow dynamics. We
quantified these effects by integrating an interface-controlled growth
model with compressible simulations of mantle plumes and slabs,
systematically exploring kinetic factors spanning seven orders of
magnitude. Each simulation was evaluated across a large range of
viscosity contrasts and 410 structure was determined after 100 Ma.

Our results reveal fundamentally different responses in hot versus cold
environments. Plumes produce consistently sharp discontinuities (2--3 km
wide) across the entire parameter space, implying that seismic
observations beneath hotspots primarily constrain thermal structure near
thermodynamic equilibrium rather than kinetic or rheological parameters.
Slabs exhibit distinct threshold behavior across three kinetic
regimes---quasi-equilibrium, intermediate, and ultra-sluggish---that are
further modulated by viscosity contrasts controlling slab geometry and
transit time through the phase transition zone.

Widespread slab penetration of the 410 in seismic tomography
(\citeproc{ref-fukao2013}{Fukao \& Obayashi, 2013}) requires effective
reaction rates exceeding the ultra-sluggish kinetic regime (\(\dot{X}\)
\(\gtrsim\) 0.07 Ma\(^{-1}\)). This critical threshold shifts
systematically with rheological strength, revealing that modest
variations in either reaction rates or viscosity contrasts can produce
substantial diversity in observed 410 topography. Uniquely constraining
kinetic versus rheological contributions requires combining 410
structural observations with independent constraints from
high-resolution tomography and kinematic data (providing descent
geometry) and deep seismicity patterns (indicating metastable olivine
volumes).

The 410 can therefore serve as a seismological probe of kinetic
conditions in cold subduction environments where disequilibrium effects
are amplified. Realizing this potential requires reducing uncertainties
through targeted mineral physics experiments that better quantify
nucleation versus growth mechanisms, water and compositional effects on
reaction rates, and microstructural evolution during deformation.
Integrating such constraints with high-resolution seismic imaging and
the forward modeling framework presented here offers a pathway toward
understanding how micro-scale kinetic processes govern mantle convection
and shape Earth's interior seismic structure.

\section*{Acknowledgements}\label{acknowledgements}
\addcontentsline{toc}{section}{Acknowledgements}

This work was funded by the UKRI NERC Large Grant no. NE/V018477/1
awarded to John Wheeler at the University of Liverpool. All computations
were undertaken on Barkla2, part of the High Performance Computing
facilities at the University of Liverpool, who graciously provided
expert support. We thank the Computational Infrastructure for
Geodynamics (\url{https://geodynamics.org}) which is funded by the
National Science Foundation under award EAR-0949446 and EAR-1550901 for
supporting the development of ASPECT.

\section*{Data Availability}\label{sec:data-availability}
\addcontentsline{toc}{section}{Data Availability}

All data, code, and relevant information for reproducing this work can
be found at
\url{https://github.com/buchanankerswell/kerswell_et_al_dynp}, and at
\ldots, the official Open Science Framework data repository. All code
within these repositories is MIT Licensed and free for use and
distribution (see license details). ASPECT version 3.0.0,
(\citeproc{ref-aspect-doi-v3.0.0}{Bangerth et al., 2024a},
\citeproc{ref-aspectmanual}{2024b};
\citeproc{ref-clevenger2021}{Clevenger \& Heister, 2021};
\citeproc{ref-fraters2019}{{Fraters et al.}, 2019};
\citeproc{ref-fraters2020}{Fraters, 2020};
\citeproc{ref-gassmoller2018}{Gassmöller et al., 2018};
\citeproc{ref-heister2017}{Heister et al., 2017};
\citeproc{ref-kronbichler2012}{Kronbichler et al., 2012}) used in these
computations is freely available under the GPL v2.0 or later license
through its software landing page
\url{https://geodynamics.org/resources/aspect} or
\url{https://aspect.geodynamics.org} and is being actively developed on
GitHub and can be accessed via
\url{https://github.com/geodynamics/aspect}.

\clearpage

\section*{References}\label{sec:references}
\addcontentsline{toc}{section}{References}

\protect\phantomsection\label{refs}
\begin{CSLReferences}{1}{0}
\bibitem[\citeproctext]{ref-agius2017}
Agius, M., Rychert, C., Harmon, N., \& Laske, G. (2017). Mapping the
mantle transition zone beneath hawaii from ps receiver functions:
Evidence for a hot plume and cold mantle downwellings. \emph{Earth and
Planetary Science Letters}, \emph{474}, 226--236.

\bibitem[\citeproctext]{ref-agrusta2017}
Agrusta, R., Goes, S., \& Van Hunen, J. (2017). Subducting-slab
transition-zone interaction: Stagnation, penetration and mode switches.
\emph{Earth and Planetary Science Letters}, \emph{464}, 10--23.

\bibitem[\citeproctext]{ref-alex2004}
Alex Song, T., Helmberger, D., \& Grand, S. (2004). Low-velocity zone
atop the 410-km seismic discontinuity in the northwestern united states.
\emph{Nature}, \emph{427}(6974), 530--533.

\bibitem[\citeproctext]{ref-aspect-doi-v3.0.0}
Bangerth, W., Dannberg, J., Fraters, M., Gassmöller, R., Glerum, A.,
Heister, T., et al. (2024a, December). ASPECT v3.0.0 (Version v3.0.0).
Zenodo. \url{https://doi.org/10.5281/zenodo.14371679}

\bibitem[\citeproctext]{ref-aspectmanual}
Bangerth, W., Dannberg, J., Fraters, M., Gassmöller, R., Glerum, A.,
Heister, T., et al. (2024b, December). {{ASPECT}: Advanced Solver for
Planetary Evolution, Convection, and Tectonics, User Manual}.
\url{https://doi.org/10.6084/m9.figshare.4865333}

\bibitem[\citeproctext]{ref-cahn1956}
Cahn, J. (1956). The kinetics of grain boundary nucleated reactions.
\emph{Acta Metallurgica}, \emph{4}(5), 449--459.

\bibitem[\citeproctext]{ref-chambers2005}
Chambers, K., Deuss, A., \& Woodhouse, J. (2005). Reflectivity of the
410-km discontinuity from PP and SS precursors. \emph{Journal of
Geophysical Research: Solid Earth}, \emph{110}(B2).

\bibitem[\citeproctext]{ref-clevenger2021}
Clevenger, T., \& Heister, T. (2021). Comparison between algebraic and
matrix-free geometric multigrid for a stokes problem on adaptive meshes
with variable viscosity. \emph{Numerical Linear Algebra with
Applications}, \emph{28}(5), e2375.

\bibitem[\citeproctext]{ref-connolly2009}
Connolly, J. (2009). The geodynamic equation of state: What and how.
\emph{Geochemistry, Geophysics, Geosystems}, \emph{10}(10).

\bibitem[\citeproctext]{ref-cottaar2016}
Cottaar, S., \& Deuss, A. (2016). Large-scale mantle discontinuity
topography beneath europe: Signature of akimotoite in subducting slabs.
\emph{Journal of Geophysical Research: Solid Earth}, \emph{121}(1),
279--292.

\bibitem[\citeproctext]{ref-cottaar2014}
Cottaar, S., Heister, T., Rose, I., \& Unterborn, C. (2014). BurnMan: A
lower mantle mineral physics toolkit. \emph{Geochemistry, Geophysics,
Geosystems}, \emph{15}(4), 1164--1179.

\bibitem[\citeproctext]{ref-dassler1996a}
Däßler, R., \& Yuen, D. (1996). The metastable olivine wedge in fast
subducting slabs: Constraints from thermo-kinetic coupling. \emph{Earth
and Planetary Science Letters}, \emph{137}(1-4), 109--118.

\bibitem[\citeproctext]{ref-dassler1996b}
Däßler, R., Yuen, D., Karato, S., \& Riedel, M. (1996). Two-dimensional
thermo-kinetic model for the olivine-spinel phase transition in
subducting slabs. \emph{Physics of the Earth and Planetary Interiors},
\emph{94}(3-4), 217--239.

\bibitem[\citeproctext]{ref-deuss2009}
Deuss, A. (2009). Global observations of mantle discontinuities using SS
and PP precursors. \emph{Surveys in Geophysics}, \emph{30}(4), 301--326.

\bibitem[\citeproctext]{ref-deuss2001}
Deuss, A., \& Woodhouse, J. (2001). Seismic observations of splitting of
the mid-transition zone discontinuity in earth's mantle. \emph{Science},
\emph{294}(5541), 354--357.

\bibitem[\citeproctext]{ref-dokht2016}
Dokht, R., Gu, Y., \& Sacchi, M. (2016). Waveform inversion of SS
precursors: An investigation of the northwestern pacific subduction
zones and intraplate volcanoes in china. \emph{Gondwana Research},
\emph{40}, 77--90.

\bibitem[\citeproctext]{ref-faccenda2017}
Faccenda, M., \& Dal Zilio, L. (2017). The role of solid--solid phase
transitions in mantle convection. \emph{Lithos}, \emph{268}, 198--224.

\bibitem[\citeproctext]{ref-fraters2020}
Fraters, M. (2020, June). The geodynamic world builder (Version v0.3.0).
Zenodo. \url{https://doi.org/10.5281/zenodo.3900603}

\bibitem[\citeproctext]{ref-fraters2019}
{Fraters, M., Thieulot, C., van den Berg, A., \& Spakman, W.} (2019).
The geodynamic world builder: A solution for complex initial conditions
in numerical modeling. \emph{Solid Earth}, \emph{10}(5), 1785--1807.

\bibitem[\citeproctext]{ref-frazer2023}
Frazer, W., \& Park, J. (2023). High-resolution mid-mantle imaging with
multiple-taper SS-precursor estimates. \emph{Geophysical Journal
International}, \emph{233}(2), 1356--1371.

\bibitem[\citeproctext]{ref-fukao2013}
Fukao, Y., \& Obayashi, M. (2013). Subducted slabs stagnant above,
penetrating through, and trapped below the 660 km discontinuity.
\emph{Journal of Geophysical Research: Solid Earth}, \emph{118}(11),
5920--5938.

\bibitem[\citeproctext]{ref-gassmoller2018}
Gassmöller, R., Lokavarapu, H., Heien, E., Puckett, E., \& Bangerth, W.
(2018). Flexible and scalable particle-in-cell methods with adaptive
mesh refinement for geodynamic computations. \emph{Geochemistry,
Geophysics, Geosystems}, \emph{19}(9), 3596--3604.

\bibitem[\citeproctext]{ref-gassmoller2020}
Gassmöller, R., Dannberg, J., Bangerth, W., Heister, T., \& Myhill, R.
(2020). On formulations of compressible mantle convection.
\emph{Geophysical Journal International}, \emph{221}(2), 1264--1280.

\bibitem[\citeproctext]{ref-glasgow2024}
Glasgow, M., Zhang, H., Schmandt, B., Zhou, W., \& Zhang, J. (2024).
Global variability of the composition and temperature at the 410-km
discontinuity from receiver function analysis of dense arrays.
\emph{Earth and Planetary Science Letters}, \emph{643}, 118889.

\bibitem[\citeproctext]{ref-goes2022}
Goes, S., Yu, C., Ballmer, M., Yan, J., \& Hilst, R. van der. (2022).
Compositional heterogeneity in the mantle transition zone. \emph{Nature
Reviews Earth \& Environment}, \emph{3}(8), 533--550.

\bibitem[\citeproctext]{ref-green1979}
Green, D., Jaques, L., \& Hibberson, W. (1979). Petrogenesis of
mid-ocean ridge basalts. In \emph{The earth: Its origin, structure and
evolution} (pp. 265--300). Academic Press.

\bibitem[\citeproctext]{ref-green1995}
Green, H., \& Houston, H. (1995). The mechanics of deep earthquakes.
\emph{Annual Review Of Earth And Planetary Sciences, Volume 23, Pp.
169-214.}, \emph{23}, 169--214.

\bibitem[\citeproctext]{ref-guest2004}
Guest, A., Schubert, G., \& Gable, C. (2004). Stresses along the
metastable wedge of olivine in a subducting slab: Possible explanation
for the tonga double seismic layer. \emph{Physics of the Earth and
Planetary Interiors}, \emph{141}(4), 253--267.

\bibitem[\citeproctext]{ref-han2021}
Han, G., Li, J., Guo, G., Mooney, W., Karato, S., \& Yuen, D. (2021).
Pervasive low-velocity layer atop the 410-km discontinuity beneath the
northwest pacific subduction zone: Implications for rheology and
geodynamics. \emph{Earth and Planetary Science Letters}, \emph{554},
116642.

\bibitem[\citeproctext]{ref-heister2017}
Heister, T., Dannberg, J., Gassmöller, R., \& Bangerth, W. (2017). High
accuracy mantle convection simulation through modern numerical
methods--II: Realistic models and problems. \emph{Geophysical Journal
International}, \emph{210}(2), 833--851.

\bibitem[\citeproctext]{ref-helffrich1996}
Helffrich, G., \& Wood, B. (1996). 410 km discontinuity sharpness and
the form of the olivine \(\alpha\)-\(\beta\) phase diagram: Resolution
of apparent seismic contradictions. \emph{Geophysical Journal
International}, \emph{126}(2), F7--F12.

\bibitem[\citeproctext]{ref-hosoya2005}
Hosoya, T., Kubo, T., Ohtani, E., Sano, A., \& Funakoshi, K. (2005).
Water controls the fields of metastable olivine in cold subducting
slabs. \emph{Geophysical Research Letters}, \emph{32}(17).

\bibitem[\citeproctext]{ref-houser2010}
Houser, C., \& Williams, Q. (2010). Reconciling pacific 410 and 660 km
discontinuity topography, transition zone shear velocity patterns, and
mantle phase transitions. \emph{Earth and Planetary Science Letters},
\emph{296}(3-4), 255--266.

\bibitem[\citeproctext]{ref-ishii2021}
Ishii, T., \& Ohtani, E. (2021). Dry metastable olivine and slab
deformation in a wet subducting slab. \emph{Nature Geoscience},
\emph{14}(7), 526--530.

\bibitem[\citeproctext]{ref-jarvis1980}
Jarvis, G., \& Mckenzie, D. (1980). Convection in a compressible fluid
with infinite prandtl number. \emph{Journal of Fluid Mechanics},
\emph{96}(3), 515--583.

\bibitem[\citeproctext]{ref-jenkins2016}
Jenkins, J., Cottaar, S., White, R., \& Deuss, A. (2016). Depressed
mantle discontinuities beneath iceland: Evidence of a garnet controlled
660 km discontinuity? \emph{Earth and Planetary Science Letters},
\emph{433}, 159--168.

\bibitem[\citeproctext]{ref-jiang2015}
Jiang, G., Zhao, D., \& Zhang, G. (2015). Detection of metastable
olivine wedge in the western pacific slab and its geodynamic
implications. \emph{Physics of the Earth and Planetary Interiors},
\emph{238}, 1--7.

\bibitem[\citeproctext]{ref-karato2008}
Karato, S. (2008). Deformation of earth materials. \emph{An Introduction
to the Rheology of Solid Earth}, \emph{463}.

\bibitem[\citeproctext]{ref-karato2011}
Karato, S. (2011). Water distribution across the mantle transition zone
and its implications for global material circulation. \emph{Earth and
Planetary Science Letters}, \emph{301}(3-4), 413--423.

\bibitem[\citeproctext]{ref-karato2001}
Karato, S., Riedel, M., \& Yuen, D. (2001). Rheological structure and
deformation of subducted slabs in the mantle transition zone:
Implications for mantle circulation and deep earthquakes. \emph{Physics
of the Earth and Planetary Interiors}, \emph{127}(1-4), 83--108.

\bibitem[\citeproctext]{ref-karato1984}
Karato, S.-I. (1984). Grain-size distribution and rheology of the upper
mantle. \emph{Tectonophysics}, \emph{104}(1-2), 155--176.

\bibitem[\citeproctext]{ref-katsura2004}
{Katsura, T., Yamada, H., Nishikawa, O., Song, M., Kubo, A., Shinmei,
T., et al.} (2004). Olivine-wadsleyite transition in the system (mg, fe)
2SiO4. \emph{Journal of Geophysical Research: Solid Earth},
\emph{109}(B2).

\bibitem[\citeproctext]{ref-kirby1996}
Kirby, S., Stein, S., Okal, E., \& Rubie, D. (1996). Metastable mantle
phase transformations and deep earthquakes in subducting oceanic
lithosphere. \emph{Reviews of Geophysics}, \emph{34}(2), 261--306.

\bibitem[\citeproctext]{ref-kronbichler2012}
Kronbichler, M., Heister, T., \& Bangerth, W. (2012). High accuracy
mantle convection simulation through modern numerical methods.
\emph{Geophysical Journal International}, \emph{191}(1), 12--29.

\bibitem[\citeproctext]{ref-kubo2004}
Kubo, T., Ohtani, E., \& Funakoshi, K. (2004). Nucleation and growth
kinetics of the \(\alpha\)-\(\beta\) transformation in Mg2SiO4determined
by in situ synchrotron powder x-ray diffraction. \emph{American
Mineralogist}, \emph{89}(2-3), 285--293.

\bibitem[\citeproctext]{ref-lallemand2005}
Lallemand, S., Heuret, A., \& Boutelier, D. (2005). On the relationships
between slab dip, back-arc stress, upper plate absolute motion, and
crustal nature in subduction zones. \emph{Geochemistry, Geophysics,
Geosystems}, \emph{6}(9).

\bibitem[\citeproctext]{ref-lawrence2008}
Lawrence, J., \& Shearer, P. (2008). Imaging mantle transition zone
thickness with SdS-SS finite-frequency sensitivity kernels.
\emph{Geophysical Journal International}, \emph{174}(1), 143--158.

\bibitem[\citeproctext]{ref-ledoux2023}
{Ledoux, E., Krug, M., Gay, J., Chantel, J., Hilairet, N., Bykov, M., et
al.} (2023). In-situ study of microstructures induced by the olivine to
wadsleyite transformation at conditions of the 410 km depth
discontinuity. \emph{American Mineralogist}, \emph{108}(12), 2283--2293.

\bibitem[\citeproctext]{ref-lee2014}
Lee, S., Rhie, J., Park, Y., \& Kim, K. (2014). Topography of the 410
and 660 km discontinuities beneath the korean peninsula and southwestern
japan using teleseismic receiver functions. \emph{Journal of Geophysical
Research: Solid Earth}, \emph{119}(9), 7245--7257.

\bibitem[\citeproctext]{ref-liu1998}
Liu, M., Kerschhofer, L., Mosenfelder, J., \& Rubie, D. (1998). The
effect of strain energy on growth rates during the olivine-spinel
transformation and implications for olivine metastability in subducting
slabs. \emph{Journal of Geophysical Research: Solid Earth},
\emph{103}(B10), 23897--23909.

\bibitem[\citeproctext]{ref-myhill2023}
Myhill, R., Cottaar, S., Heister, T., Rose, I., Unterborn, C., Dannberg,
J., \& Gassmöller, R. (2023). BurnMan--a python toolkit for planetary
geophysics, geochemistry and thermodynamics.

\bibitem[\citeproctext]{ref-ohuchi2022}
Ohuchi, T., Higo, Y., Tange, Y., Sakai, T., Matsuda, K., \& Irifune, T.
(2022). In situ x-ray and acoustic observations of deep seismic faulting
upon phase transitions in olivine. \emph{Nature Communications},
\emph{13}(1), 5213.

\bibitem[\citeproctext]{ref-perrillat2013}
Perrillat, J., Daniel, I., Bolfan-Casanova, N., Chollet, M., Morard, G.,
\& Mezouar, M. (2013). Mechanism and kinetics of the
\(\alpha\)--\(\beta\) transition in san carlos olivine Mg1. 8Fe0. 2SiO4.
\emph{Journal of Geophysical Research: Solid Earth}, \emph{118}(1),
110--119.

\bibitem[\citeproctext]{ref-perrillat2016}
Perrillat, J., Chollet, M., Durand, S., De Moortele, B. van, Chambat,
F., Mezouar, M., \& Daniel, I. (2016). Kinetics of the
olivine--ringwoodite transformation and seismic attenuation in the
earth's mantle transition zone. \emph{Earth and Planetary Science
Letters}, \emph{433}, 360--369.

\bibitem[\citeproctext]{ref-ranalli1995}
Ranalli, G. (1995). \emph{Rheology of the earth}. Springer Science \&
Business Media.

\bibitem[\citeproctext]{ref-ringwood1975}
Ringwood, A. (1975). Composition and petrology of the earth's mantle.
\emph{MacGraw-Hill}, \emph{618}.

\bibitem[\citeproctext]{ref-rubie1994}
Rubie, D., \& Ross II, C. (1994). Kinetics of the olivine-spinel
transformation in subducting lithosphere: Experimental constraints and
implications for deep slab processes. \emph{Physics of the Earth and
Planetary Interiors}, \emph{86}(1-3), 223--243.

\bibitem[\citeproctext]{ref-saikia2008}
Saikia, A., Frost, D., \& Rubie, D. (2008). Splitting of the
520-kilometer seismic discontinuity and chemical heterogeneity in the
mantle. \emph{Science}, \emph{319}(5869), 1515--1518.

\bibitem[\citeproctext]{ref-schmandt2012}
Schmandt, B. (2012). Mantle transition zone shear velocity gradients
beneath USArray. \emph{Earth and Planetary Science Letters}, \emph{355},
119--130.

\bibitem[\citeproctext]{ref-schmeling1999}
Schmeling, H., Monz, R., \& Rubie, D. (1999). The influence of olivine
metastability on the dynamics of subduction. \emph{Earth and Planetary
Science Letters}, \emph{165}(1), 55--66.

\bibitem[\citeproctext]{ref-schmerr2007}
Schmerr, N., \& Garnero, E. (2007). Upper mantle discontinuity
topography from thermal and chemical heterogeneity. \emph{Science},
\emph{318}(5850), 623--626.

\bibitem[\citeproctext]{ref-schubert2001}
Schubert, G., Turcotte, D., \& Olson, P. (2001). \emph{Mantle convection
in the earth and planets}. Cambridge University Press.

\bibitem[\citeproctext]{ref-shearer2000}
Shearer, P. (2000). Upper mantle seismic discontinuities.
\emph{Geophysical Monograph-American Geophysical Union}, \emph{117},
115--132.

\bibitem[\citeproctext]{ref-shen2020}
Shen, Z., \& Zhan, Z. (2020). Metastable olivine wedge beneath the japan
sea imaged by seismic interferometry. \emph{Geophysical Research
Letters}, \emph{47}(6), e2019GL085665.

\bibitem[\citeproctext]{ref-sindhusuta2025}
Sindhusuta, S., Chi, S., Foster, C., Officer, T., \& Wang, Y. (2025).
Numerical investigation into mechanical behavior of metastable olivine
during phase transformation: Implications for deep-focus earthquakes.
\emph{Journal of Geophysical Research: Solid Earth}, \emph{130}(2),
e2024JB030557.

\bibitem[\citeproctext]{ref-smyth1987}
Smyth, J. (1987). Beta-Mg2 SiO4; a potential host for water in the
mantle? \emph{American Mineralogist}, \emph{72}(11-12), 1051--1055.

\bibitem[\citeproctext]{ref-smyth2002}
Smyth, J., \& Frost, D. (2002). The effect of water on the 410-km
discontinuity: An experimental study. \emph{Geophysical Research
Letters}, \emph{29}(10), 123--1.

\bibitem[\citeproctext]{ref-stixrude2022}
Stixrude, L., \& Lithgow-Bertelloni, C. (2022). Thermal expansivity,
heat capacity and bulk modulus of the mantle. \emph{Geophysical Journal
International}, \emph{228}(2), 1119--1149.

\bibitem[\citeproctext]{ref-tauzin2017}
Tauzin, B., Kim, S., \& Kennett, B. (2017). Pervasive seismic
low-velocity zones within stagnant plates in the mantle transition zone:
Thermal or compositional origin? \emph{Earth and Planetary Science
Letters}, \emph{477}, 1--13.

\bibitem[\citeproctext]{ref-vanstiphout2019}
Van Stiphout, A., Cottaar, S., \& Deuss, A. (2019). Receiver function
mapping of mantle transition zone discontinuities beneath alaska using
scaled 3-d velocity corrections. \emph{Geophysical Journal
International}, \emph{219}(2), 1432--1446.

\bibitem[\citeproctext]{ref-wei2017}
Wei, S., \& Shearer, P. (2017). A sporadic low-velocity layer atop the
410 km discontinuity beneath the pacific ocean. \emph{Journal of
Geophysical Research: Solid Earth}, \emph{122}(7), 5144--5159.

\bibitem[\citeproctext]{ref-wheeler2014}
Wheeler, J. (2014). Dramatic effects of stress on metamorphic reactions.
\emph{Geology}, \emph{42}(8), 647--650.

\bibitem[\citeproctext]{ref-wheeler2018}
Wheeler, J. (2018). The effects of stress on reactions in the earth:
Sometimes rather mean, usually normal, always important. \emph{Journal
of Metamorphic Geology}, \emph{36}(4), 439--461.

\bibitem[\citeproctext]{ref-wheeler2020}
Wheeler, J. (2020). A unifying basis for the interplay of stress and
chemical processes in the earth: Support from diverse experiments.
\emph{Contributions to Mineralogy and Petrology}, \emph{175}(12), 116.

\bibitem[\citeproctext]{ref-zhan2020}
Zhan, Z. (2020). Mechanisms and implications of deep earthquakes.
\emph{Annual Review of Earth and Planetary Sciences}, \emph{48}(1),
147--174.

\end{CSLReferences}

\end{document}