%%
% Copyright (c) 2017 - 2024, Pascal Wagler;
% Copyright (c) 2014 - 2024, John MacFarlane
%
% All rights reserved.
%
% Redistribution and use in source and binary forms, with or without
% modification, are permitted provided that the following conditions
% are met:
%
% - Redistributions of source code must retain the above copyright
% notice, this list of conditions and the following disclaimer.
%
% - Redistributions in binary form must reproduce the above copyright
% notice, this list of conditions and the following disclaimer in the
% documentation and/or other materials provided with the distribution.
%
% - Neither the name of John MacFarlane nor the names of other
% contributors may be used to endorse or promote products derived
% from this software without specific prior written permission.
%
% THIS SOFTWARE IS PROVIDED BY THE COPYRIGHT HOLDERS AND CONTRIBUTORS
% "AS IS" AND ANY EXPRESS OR IMPLIED WARRANTIES, INCLUDING, BUT NOT
% LIMITED TO, THE IMPLIED WARRANTIES OF MERCHANTABILITY AND FITNESS
% FOR A PARTICULAR PURPOSE ARE DISCLAIMED. IN NO EVENT SHALL THE
% COPYRIGHT OWNER OR CONTRIBUTORS BE LIABLE FOR ANY DIRECT, INDIRECT,
% INCIDENTAL, SPECIAL, EXEMPLARY, OR CONSEQUENTIAL DAMAGES (INCLUDING,
% BUT NOT LIMITED TO, PROCUREMENT OF SUBSTITUTE GOODS OR SERVICES;
% LOSS OF USE, DATA, OR PROFITS; OR BUSINESS INTERRUPTION) HOWEVER
% CAUSED AND ON ANY THEORY OF LIABILITY, WHETHER IN CONTRACT, STRICT
% LIABILITY, OR TORT (INCLUDING NEGLIGENCE OR OTHERWISE) ARISING IN
% ANY WAY OUT OF THE USE OF THIS SOFTWARE, EVEN IF ADVISED OF THE
% POSSIBILITY OF SUCH DAMAGE.
%%

%%
% This is the Eisvogel pandoc LaTeX template.
%
% For usage information and examples visit the official GitHub page:
% https://github.com/Wandmalfarbe/pandoc-latex-template
%%

% Options for packages loaded elsewhere
\PassOptionsToPackage{unicode}{hyperref}
\PassOptionsToPackage{hyphens}{url}
\PassOptionsToPackage{dvipsnames,svgnames,x11names,table}{xcolor}
%
\documentclass[
  paper=a4,
  ,captions=tableheading
]{scrartcl}
\usepackage{amsmath,amssymb}
% Use setspace anyway because we change the default line spacing.
% The spacing is changed early to affect the titlepage and the TOC.
\usepackage{setspace}
\setstretch{1.2}
\usepackage{iftex}
\ifPDFTeX
  \usepackage[T1]{fontenc}
  \usepackage[utf8]{inputenc}
  \usepackage{textcomp} % provide euro and other symbols
\else % if luatex or xetex
  \usepackage{unicode-math} % this also loads fontspec
  \defaultfontfeatures{Scale=MatchLowercase}
  \defaultfontfeatures[\rmfamily]{Ligatures=TeX,Scale=1}
\fi
\usepackage{lmodern}
\ifPDFTeX\else
  % xetex/luatex font selection
\fi
% Use upquote if available, for straight quotes in verbatim environments
\IfFileExists{upquote.sty}{\usepackage{upquote}}{}
\IfFileExists{microtype.sty}{% use microtype if available
  \usepackage[]{microtype}
  \UseMicrotypeSet[protrusion]{basicmath} % disable protrusion for tt fonts
}{}
\makeatletter
\@ifundefined{KOMAClassName}{% if non-KOMA class
  \IfFileExists{parskip.sty}{%
    \usepackage{parskip}
  }{% else
    \setlength{\parindent}{0pt}
    \setlength{\parskip}{6pt plus 2pt minus 1pt}}
}{% if KOMA class
  \KOMAoptions{parskip=half}}
\makeatother
\usepackage{xcolor}
\definecolor{default-linkcolor}{HTML}{A50000}
\definecolor{default-filecolor}{HTML}{A50000}
\definecolor{default-citecolor}{HTML}{4077C0}
\definecolor{default-urlcolor}{HTML}{4077C0}
\usepackage[margin=2.5cm,includehead=true,includefoot=true,centering,]{geometry}
\usepackage{listings}
\newcommand{\passthrough}[1]{#1}
\renewcommand\lstlistingname{Code Block}
\renewcommand\lstlistlistingname{Code Block}
\lstset{defaultdialect=[5.3]Lua}
\lstset{defaultdialect=[x86masm]Assembler}
\usepackage{longtable,booktabs,array}
\usepackage{calc} % for calculating minipage widths
% Correct order of tables after \paragraph or \subparagraph
\usepackage{etoolbox}
\makeatletter
\patchcmd\longtable{\par}{\if@noskipsec\mbox{}\fi\par}{}{}
\makeatother
% Allow footnotes in longtable head/foot
\IfFileExists{footnotehyper.sty}{\usepackage{footnotehyper}}{\usepackage{footnote}}
\makesavenoteenv{longtable}
% add backlinks to footnote references, cf. https://tex.stackexchange.com/questions/302266/make-footnote-clickable-both-ways
\usepackage{footnotebackref}
\usepackage{graphicx}
\makeatletter
\newsavebox\pandoc@box
\newcommand*\pandocbounded[1]{% scales image to fit in text height/width
  \sbox\pandoc@box{#1}%
  \Gscale@div\@tempa{\textheight}{\dimexpr\ht\pandoc@box+\dp\pandoc@box\relax}%
  \Gscale@div\@tempb{\linewidth}{\wd\pandoc@box}%
  \ifdim\@tempb\p@<\@tempa\p@\let\@tempa\@tempb\fi% select the smaller of both
  \ifdim\@tempa\p@<\p@\scalebox{\@tempa}{\usebox\pandoc@box}%
  \else\usebox{\pandoc@box}%
  \fi%
}
% Set default figure placement to htbp
% Make use of float-package and set default placement for figures to H.
% The option H means 'PUT IT HERE' (as  opposed to the standard h option which means 'You may put it here if you like').
\usepackage{float}
\floatplacement{figure}{H}
\makeatother
\setlength{\emergencystretch}{3em} % prevent overfull lines
\providecommand{\tightlist}{%
  \setlength{\itemsep}{0pt}\setlength{\parskip}{0pt}}
\setcounter{secnumdepth}{5}
% definitions for citeproc citations
\NewDocumentCommand\citeproctext{}{}
\NewDocumentCommand\citeproc{mm}{%
  \begingroup\def\citeproctext{#2}\cite{#1}\endgroup}
\makeatletter
 % allow citations to break across lines
 \let\@cite@ofmt\@firstofone
 % avoid brackets around text for \cite:
 \def\@biblabel#1{}
 \def\@cite#1#2{{#1\if@tempswa , #2\fi}}
\makeatother
\newlength{\cslhangindent}
\setlength{\cslhangindent}{1.5em}
\newlength{\csllabelwidth}
\setlength{\csllabelwidth}{3em}
\newenvironment{CSLReferences}[2] % #1 hanging-indent, #2 entry-spacing
  {\begin{list}{}{%
   \setlength{\itemindent}{0pt}
   \setlength{\leftmargin}{0pt}
   \setlength{\parsep}{0pt}
   % turn on hanging indent if param 1 is 1
   \ifodd #1
    \setlength{\leftmargin}{\cslhangindent}
    \setlength{\itemindent}{-1\cslhangindent}
   \fi
   % set entry spacing
   \setlength{\itemsep}{#2\baselineskip}}}
  {\end{list}}
\usepackage{calc}
\newcommand{\CSLBlock}[1]{\hfill\break\parbox[t]{\linewidth}{\strut\ignorespaces#1\strut}}
\newcommand{\CSLLeftMargin}[1]{\parbox[t]{\csllabelwidth}{\strut#1\strut}}
\newcommand{\CSLRightInline}[1]{\parbox[t]{\linewidth - \csllabelwidth}{\strut#1\strut}}
\newcommand{\CSLIndent}[1]{\hspace{\cslhangindent}#1}
\ifLuaTeX
\usepackage[bidi=basic]{babel}
\else
\usepackage[bidi=default]{babel}
\fi
\babelprovide[main,import]{american}
% get rid of language-specific shorthands (see #6817):
\let\LanguageShortHands\languageshorthands
\def\languageshorthands#1{}
\makeatletter
\newcounter{none}
\renewcommand{\thenone}{}
\makeatother
\makeatletter
\@ifpackageloaded{subcaption}{}{\usepackage{subcaption}}
\@ifpackageloaded{caption}{}{\usepackage{caption}}
\captionsetup[subfigure]{margin=0.5em}
\AtBeginDocument{%
\renewcommand*\figurename{Figure}
\renewcommand*\tablename{Table}
}
\AtBeginDocument{%
\renewcommand*\listfigurename{List of Figures}
\renewcommand*\listtablename{List of Tables}
}
\newcounter{pandoccrossref@subfigures@footnote@counter}
\newenvironment{pandoccrossrefsubfigures}{%
\setcounter{pandoccrossref@subfigures@footnote@counter}{0}
\begin{figure}\centering%
\gdef\global@pandoccrossref@subfigures@footnotes{}%
\DeclareRobustCommand{\footnote}[1]{\footnotemark%
\stepcounter{pandoccrossref@subfigures@footnote@counter}%
\ifx\global@pandoccrossref@subfigures@footnotes\empty%
\gdef\global@pandoccrossref@subfigures@footnotes{{##1}}%
\else%
\g@addto@macro\global@pandoccrossref@subfigures@footnotes{, {##1}}%
\fi}}%
{\end{figure}%
\addtocounter{footnote}{-\value{pandoccrossref@subfigures@footnote@counter}}
\@for\f:=\global@pandoccrossref@subfigures@footnotes\do{\stepcounter{footnote}\footnotetext{\f}}%
\gdef\global@pandoccrossref@subfigures@footnotes{}}
\newcommand*\listoflistings\lstlistoflistings
\AtBeginDocument{%
\renewcommand*{\lstlistlistingname}{List of Listings}
}
\makeatother
\usepackage{bookmark}
\IfFileExists{xurl.sty}{\usepackage{xurl}}{} % add URL line breaks if available
\urlstyle{same}
\hypersetup{
  pdftitle={Displaced and Faded 410s},
  pdfauthor={Kerswell B.; Wheeler J.; Gassmöller R.},
  pdflang={en-US},
  pdfsubject={Mantle Convection},
  pdfkeywords={mantle convection, phase changes, geodynamics, numerical
modeling},
  colorlinks=true,
  linkcolor={default-linkcolor},
  filecolor={default-filecolor},
  citecolor={default-citecolor},
  urlcolor={default-urlcolor},
  breaklinks=true,
  pdfcreator={LaTeX via pandoc with the Eisvogel template}}
\title{Displaced and Faded 410s}
\usepackage{etoolbox}
\makeatletter
\providecommand{\subtitle}[1]{% add subtitle to \maketitle
  \apptocmd{\@title}{\par {\large #1 \par}}{}{}
}
\makeatother
\subtitle{How micro-scale kinetics complicate mantle-scale seismic
structures and flow dynamics}
\author{Kerswell B. \and Wheeler J. \and Gassmöller R.}
\date{28 October 2025}



%%
%% added
%%


%
% for the background color of the title page
%
\usepackage{pagecolor}
\usepackage{afterpage}
\usepackage[margin=2.5cm,includehead=true,includefoot=true,centering]{geometry}

%
% break urls
%
\PassOptionsToPackage{hyphens}{url}

%
% When using babel or polyglossia with biblatex, loading csquotes is recommended
% to ensure that quoted texts are typeset according to the rules of your main language.
%
\usepackage{csquotes}

%
% captions
%
\definecolor{caption-color}{HTML}{777777}
\usepackage[font={stretch=1.2}, textfont={color=caption-color}, position=top, skip=4mm, labelfont=bf, singlelinecheck=false, justification=justified]{caption}
\setcapindent{0em}

%
% blockquote
%
\definecolor{blockquote-border}{RGB}{221,221,221}
\definecolor{blockquote-text}{RGB}{119,119,119}
\usepackage{mdframed}
\newmdenv[rightline=false,bottomline=false,topline=false,linewidth=3pt,linecolor=blockquote-border,skipabove=\parskip]{customblockquote}
\renewenvironment{quote}{\begin{customblockquote}\list{}{\rightmargin=0em\leftmargin=0em}%
\item\relax\color{blockquote-text}\ignorespaces}{\unskip\unskip\endlist\end{customblockquote}}

%
% Source Sans Pro as the default font family
% Source Code Pro for monospace text
%
% 'default' option sets the default
% font family to Source Sans Pro, not \sfdefault.
%
\ifnum 0\ifxetex 1\fi\ifluatex 1\fi=0 % if pdftex
    \usepackage[default]{sourcesanspro}
  \usepackage{sourcecodepro}
  \else % if not pdftex
    \usepackage[default]{sourcesanspro}
  \usepackage{sourcecodepro}

  % XeLaTeX specific adjustments for straight quotes: https://tex.stackexchange.com/a/354887
  % This issue is already fixed (see https://github.com/silkeh/latex-sourcecodepro/pull/5) but the
  % fix is still unreleased.
  % TODO: Remove this workaround when the new version of sourcecodepro is released on CTAN.
  \ifxetex
    \makeatletter
    \defaultfontfeatures[\ttfamily]
      { Numbers   = \sourcecodepro@figurestyle,
        Scale     = \SourceCodePro@scale,
        Extension = .otf }
    \setmonofont
      [ UprightFont    = *-\sourcecodepro@regstyle,
        ItalicFont     = *-\sourcecodepro@regstyle It,
        BoldFont       = *-\sourcecodepro@boldstyle,
        BoldItalicFont = *-\sourcecodepro@boldstyle It ]
      {SourceCodePro}
    \makeatother
  \fi
  \fi

%
% heading color
%
\definecolor{heading-color}{RGB}{40,40,40}
\addtokomafont{section}{\color{heading-color}}
% When using the classes report, scrreprt, book,
% scrbook or memoir, uncomment the following line.
%\addtokomafont{chapter}{\color{heading-color}}

%
% variables for title, author and date
%
\usepackage{titling}
\title{Displaced and Faded 410s}
\author{Kerswell B., Wheeler J., Gassmöller R.}
\date{28 October 2025}

%
% tables
%

\definecolor{table-row-color}{HTML}{F5F5F5}
\definecolor{table-rule-color}{HTML}{999999}

%\arrayrulecolor{black!40}
\arrayrulecolor{table-rule-color}     % color of \toprule, \midrule, \bottomrule
\setlength\heavyrulewidth{0.3ex}      % thickness of \toprule, \bottomrule
\renewcommand{\arraystretch}{1.3}     % spacing (padding)


%
% remove paragraph indentation
%
\setlength{\parindent}{0pt}
\setlength{\parskip}{6pt plus 2pt minus 1pt}
\setlength{\emergencystretch}{3em}  % prevent overfull lines

%
%
% Listings
%
%


%
% general listing colors
%
\definecolor{listing-background}{HTML}{E5E5E5}
\definecolor{listing-rule}{HTML}{B3B2B3}
\definecolor{listing-numbers}{HTML}{B3B2B3}
\definecolor{listing-text-color}{HTML}{000000}
\definecolor{listing-keyword}{HTML}{435489}
\definecolor{listing-keyword-2}{HTML}{994d00}
\definecolor{listing-keyword-3}{HTML}{9137CB}
\definecolor{listing-identifier}{HTML}{5F2E3A}
\definecolor{listing-symbol}{HTML}{990000}
\definecolor{listing-string}{HTML}{006B6B}
\definecolor{listing-comment}{HTML}{3A5F2E}

\lstdefinestyle{eisvogel_listing_style}{
  numbers          = left,
  xleftmargin      = 2.7em,
  framexleftmargin = 2.5em,
  backgroundcolor  = \color{listing-background},
  basicstyle       = \color{listing-text-color}\linespread{1.0}%
                      \lst@ifdisplaystyle%
                      \small%
                      \fi\ttfamily{},
  breaklines       = true,
  frame            = single,
  framesep         = 0.19em,
  rulecolor        = \color{listing-rule},
  frameround       = ffff,
  tabsize          = 4,
  aboveskip        = 1.0em,
  belowskip        = 0.1em,
  abovecaptionskip = 0em,
  belowcaptionskip = 1.0em,
  sensitive        = true,
  showstringspaces = false,
  escapeinside     = {/*@}{@*/}, % Allow LaTeX inside these special comments
  numberstyle      = \color{listing-numbers}
}

\lstdefinelanguage{bash}{
  morekeywords    = {if, fi, then, else, elif, for, while, do, done, case, esac,
                     function, select, set, until, readonly, declare, let, eval, exit,
                     break, continue, shift, return, subsection, end},
  sensitive       = true,
  morecomment     = [l]{\#},
  morestring      = [b]",
  morestring      = [b]',
  stringstyle     = \color{listing-string},
  commentstyle    = \color{listing-comment},
  keywordstyle    = \color{listing-keyword}\bfseries,
  keywordstyle    = {[2]\color{listing-keyword-2}\bfseries},
  literate        =
    % symbols
    {,}{{\textcolor{listing-symbol}{,}}}1
    {;}{{\textcolor{listing-symbol}{;}}}1
    {!}{{\textcolor{listing-symbol}{!}}}1
    {\\}{{\textcolor{listing-symbol}{\textbackslash}}}1
    {=}{{\textcolor{listing-symbol}{=}}}1
    {~}{{\textcolor{listing-symbol}{\textasciitilde}}}1
    {=~}{{\textcolor{listing-symbol}{=\textasciitilde}}}2
    {[}{{\textcolor{listing-symbol}{[}}}1
    {]}{{\textcolor{listing-symbol}{]}}}1
    {\{}{{\textcolor{listing-symbol}{\{}}}1
    {\}}{{\textcolor{listing-symbol}{\}}}}1
    {|}{{\textcolor{listing-symbol}{|}}}1
    {<}{{\textcolor{listing-symbol}{<}}}1
    {>}{{\textcolor{listing-symbol}{>}}}1
    {*}{{\textcolor{listing-symbol}{*}}}1
    % keywords
    {\&>}{{\textcolor{listing-keyword-2}{\&>}}}2
    {/dev/null}{{\textcolor{listing-keyword-2}{/dev/null}}}9
    {\&}{{\textcolor{listing-keyword-2}{\&}}}1
    {--}{{\textcolor{listing-keyword-2}{--}}}2
    {~}{{\textcolor{listing-keyword-2}{~}}}1
    {-}{{\textcolor{listing-keyword-2}{-}}}1
    {-e}{{\textcolor{listing-keyword-2}{-e}}}2
    {-j}{{\textcolor{listing-keyword-2}{-j}}}2
    {-d}{{\textcolor{listing-keyword-2}{-d}}}2
    {-D}{{\textcolor{listing-keyword-2}{-D}}}2
    {-x}{{\textcolor{listing-keyword-2}{-x}}}2
    {-f}{{\textcolor{listing-keyword-2}{-f}}}2
    {-q}{{\textcolor{listing-keyword-2}{-q}}}2
    {-y}{{\textcolor{listing-keyword-2}{-y}}}2
    {-r}{{\textcolor{listing-keyword-2}{-r}}}2
    {-v}{{\textcolor{listing-keyword-2}{-v}}}2
    {-rf}{{\textcolor{listing-keyword-2}{-rf}}}3
    {-type}{{\textcolor{listing-keyword-2}{-type}}}5
    {-name}{{\textcolor{listing-keyword-2}{-name}}}5
    {á}{{\'a}}1 {é}{{\'e}}1 {í}{{\'i}}1 {ó}{{\'o}}1 {ú}{{\'u}}1
    {Á}{{\'A}}1 {É}{{\'E}}1 {Í}{{\'I}}1 {Ó}{{\'O}}1 {Ú}{{\'U}}1
    {à}{{\`a}}1 {è}{{\`e}}1 {ì}{{\`i}}1 {ò}{{\`o}}1 {ù}{{\`u}}1
    {À}{{\`A}}1 {È}{{\`E}}1 {Ì}{{\`I}}1 {Ò}{{\`O}}1 {Ù}{{\`U}}1
    {ä}{{\"a}}1 {ë}{{\"e}}1 {ï}{{\"i}}1 {ö}{{\"o}}1 {ü}{{\"u}}1
    {Ä}{{\"A}}1 {Ë}{{\"E}}1 {Ï}{{\"I}}1 {Ö}{{\"O}}1 {Ü}{{\"U}}1
    {â}{{\^a}}1 {ê}{{\^e}}1 {î}{{\^i}}1 {ô}{{\^o}}1 {û}{{\^u}}1
    {Â}{{\^A}}1 {Ê}{{\^E}}1 {Î}{{\^I}}1 {Ô}{{\^O}}1 {Û}{{\^U}}1
    {œ}{{\oe}}1 {Œ}{{\OE}}1 {æ}{{\ae}}1 {Æ}{{\AE}}1 {ß}{{\ss}}1
    {ç}{{\c c}}1 {Ç}{{\c C}}1 {ø}{{\o}}1 {å}{{\r a}}1 {Å}{{\r A}}1
    {€}{{\EUR}}1 {£}{{\pounds}}1 {«}{{\guillemotleft}}1
    {»}{{\guillemotright}}1 {ñ}{{\~n}}1 {Ñ}{{\~N}}1 {¿}{{?`}}1
    {…}{{\ldots}}1 {≥}{{>=}}1 {≤}{{<=}}1 {„}{{\glqq}}1 {“}{{\grqq}}1
    {”}{{''}}1
}

\lstset{style=eisvogel_listing_style}


%
% header and footer
%
\usepackage[headsepline,footsepline]{scrlayer-scrpage}

\newpairofpagestyles{eisvogel-header-footer}{
  \clearpairofpagestyles
  \ihead*{Displaced and Faded 410s}
  \chead*{}
  \ohead*{28 October 2025}
  \ifoot*{Kerswell B., Wheeler J., Gassmöller R.}
  \cfoot*{}
  \ofoot*{\thepage}
  \addtokomafont{pageheadfoot}{\upshape}
}
\pagestyle{eisvogel-header-footer}



%%
%% end added
%%

\begin{document}

%%
%% begin titlepage
%%
\begin{titlepage}
\newgeometry{left=6cm}
\definecolor{titlepage-color}{HTML}{2E7A40}
\newpagecolor{titlepage-color}\afterpage{\restorepagecolor}
\newcommand{\colorRule}[3][black]{\textcolor[HTML]{#1}{\rule{#2}{#3}}}
\begin{flushleft}
\noindent
\\[-1em]
\color[HTML]{FFFFFF}
\makebox[0pt][l]{\colorRule[FFFFFF]{1.3\textwidth}{2pt}}
\par
\noindent

{
  \setstretch{1.4}
  \vfill
  \noindent {\huge \textbf{\textsf{Displaced and Faded 410s}}}
    \vskip 1em
  {\Large \textsf{How micro-scale kinetics complicate mantle-scale
seismic structures and flow dynamics}}
    \vskip 2em
  \noindent {\Large \textsf{Kerswell B., Wheeler J., Gassmöller R.}}
  \vfill
}


\textsf{28 October 2025}
\end{flushleft}
\end{titlepage}
\restoregeometry
\pagenumbering{arabic}

%%
%% end titlepage
%%

% \maketitle


{
\hypersetup{linkcolor=}
\setcounter{tocdepth}{3}
\tableofcontents
\newpage
}
\section*{Abstract}\label{sec:abstract}
\addcontentsline{toc}{section}{Abstract}

The seismic expression of Earth's 410 km discontinuity (the ``410'')
varies substantially across different tectonic settings, from sharp,
high-amplitude interfaces to broad, diffuse transitions---patterns that
cannot be explained by equilibrium thermodynamics alone. Laboratory
experiments demonstrate that the olivine \(\Leftrightarrow\) wadsleyite
phase transition responsible for the 410 is strongly rate-limited, yet
quantitative links between micro-scale kinetics and mantle-scale seismic
structures and flow dynamics remain poorly understood. Here we
systematically investigate these relationships by coupling an
interface-controlled growth model to compressible simulations of mantle
plumes and subducting slabs using ASPECT. We vary kinetic parameters
across seven orders of magnitude and quantify the resulting 410
displacements and widths. Our results reveal a fundamental asymmetry
between hot and cold mantle environments. In plumes, high temperatures
suppress olivine metastability, producing consistently sharp 410s (2--5
km wide) that remain insensitive to kinetic variations across the
explored parameter space. In slabs, by contrast, kinetics exert
first-order control on 410 structure and flow dynamics. We identify
three distinct kinetic regimes in these cold environments: (1)
quasi-equilibrium behavior at high phase transition rates (\(\dot{X}\)
\textgreater{} 10\(^1\) Ma\(^{-1}\)) enabling continuous slab
penetration with narrow, positively displaced 410s; (2) intermediate
phase transition rates (10\(^{-1.5}\) \textless{} \(\dot{X}\)
\textless{} 10\(^1\) Ma\(^{-1}\)) generating progressively broader,
deeper 410s and metastable olivine wedges that resist slab descent
without preventing it; and (3) ultra-sluggish phase transition rates
(\(\dot{X}\) \textless{} 10\(^{-1.5}\) Ma\(^{-1}\)) causing complete
slab stagnation with re-sharpened but deeply displaced 410s
(\textgreater{} 100 km). These findings demonstrate that phase
transition rates strongly influence 410 structure in subduction zones
and establish the 410 as a potential seismological constraint on kinetic
processes operating in Earth's upper mantle, particularly in cold
environments where disequilibrium effects are amplified.

\cleardoublepage

\section*{Definition of Symbols}\label{sec:symbols}
\addcontentsline{toc}{section}{Definition of Symbols}

{\def\LTcaptype{none} % do not increment counter
\begin{longtable}[]{@{}
  >{\raggedright\arraybackslash}p{(\linewidth - 6\tabcolsep) * \real{0.4688}}
  >{\raggedright\arraybackslash}p{(\linewidth - 6\tabcolsep) * \real{0.2188}}
  >{\raggedright\arraybackslash}p{(\linewidth - 6\tabcolsep) * \real{0.1562}}
  >{\raggedright\arraybackslash}p{(\linewidth - 6\tabcolsep) * \real{0.1562}}@{}}
\toprule\noalign{}
\begin{minipage}[b]{\linewidth}\raggedright
Parameter
\end{minipage} & \begin{minipage}[b]{\linewidth}\raggedright
Symbol
\end{minipage} & \begin{minipage}[b]{\linewidth}\raggedright
Unit
\end{minipage} & \begin{minipage}[b]{\linewidth}\raggedright
Equations
\end{minipage} \\
\midrule\noalign{}
\endhead
\bottomrule\noalign{}
\endlastfoot
Activation enthalpy & \(H^{\ast}\) & J mol\(^{-1}\) &
\ref{eq:growth-rate}, \ref{eq:phase-transition-rate} \\
Activation volume & \(V^{\ast}\) & m\(^3\) mol\(^{-1}\) &
\ref{eq:growth-rate}, \ref{eq:phase-transition-rate} \\
Compressibility (reference) & \(\bar{\beta}\) & Pa\(^{-1}\) &
\ref{eq:density-ala} \\
Density & \(\rho\) & kg m\(^{-3}\) &
\ref{eq:navier-stokes-no-inertia}--\ref{eq:continuity-expanded},
\ref{eq:density-ala-expansion}--\ref{eq:density-ala} \\
Density (reference) & \(\bar{\rho}\) & kg m\(^{-3}\) &
\ref{eq:adiabatic-pressure}--\ref{eq:density-ala} \\
Density (dynamic) & \(\hat{\rho}\) & kg m\(^{-3}\) & - \\
Deviatoric stress tensor & \(\sigma^{\prime}\) & Pa &
\ref{eq:navier-stokes-no-inertia}, \ref{eq:energy} \\
Deviatoric strain rate tensor & \(\dot{\epsilon}^{\prime}\) & s\(^{-1}\)
& \ref{eq:energy} \\
Gas constant & \(R\) & J K\(^{-1}\) mol\(^{-1}\) & \ref{eq:growth-rate},
\ref{eq:phase-transition-rate} \\
Grain size & \(d\) & 1 m & \ref{eq:growth-rate},
\ref{eq:phase-transition-rate} \\
Gravitational acceleration & \(g\) & m s\(^{-2}\) &
\ref{eq:navier-stokes-no-inertia},
\ref{eq:adiabatic-temperature}--\ref{eq:adiabatic-pressure} \\
Growth rate & \(\dot{x}\) & m s\(^{-1}\) & \ref{eq:volume-fraction},
\ref{eq:growth-rate} \\
Kinetic prefactor & \(A\) & m s\(^{-1}\) K\(^{-1}\)
ppm\(_\mathrm{OH}^{-n}\) & \ref{eq:growth-rate} \\
Latent heat & \(Q_L\) & J kg\(^{-1}\) & \ref{eq:energy} \\
Molar entropy & \(\bar{S}\) & J mol\(^{-1}\) K\(^{-1}\) &
\ref{eq:gibbs} \\
Molar Gibbs free energy & \(\bar{G}\) & J mol\(^{-1}\) &
\ref{eq:gibbs} \\
Molar volume & \(\bar{V}\) & m\(^{3}\) mol\(^{-1}\) & \ref{eq:gibbs} \\
Nucleation site factor & \(N\) & m\(^{-1}\) & \ref{eq:volume-fraction},
\ref{eq:phase-transition-rate-short} \\
Phase transition rate & \(\frac{dX}{dt}\),
\(\frac{\partial X}{\partial t}\), \(\dot{X}\) & s\(^{-1}\) &
\ref{eq:phase-transition-rate-short}--\ref{eq:composition} \\
Pressure & \(P\) & Pa & \ref{eq:navier-stokes-no-inertia},
\ref{eq:energy}, \ref{eq:growth-rate}, \ref{eq:phase-transition-rate} \\
Pressure (reference) & \(\bar{P}\) & K & \ref{eq:adiabatic-pressure} \\
Pressure (dynamic) & \(\hat{P}\) & Pa & \ref{eq:density-ala},
\ref{eq:gibbs} \\
Specific heat capacity (reference) & \(\bar{C}_p\) & J kg\(^{-1}\)
K\(^{-1}\) & \ref{eq:energy}, \ref{eq:adiabatic-temperature} \\
Temperature & \(T\) & K & \ref{eq:energy}, \ref{eq:growth-rate},
\ref{eq:phase-transition-rate} \\
Temperature (reference) & \(\bar{T}\) & K &
\ref{eq:adiabatic-temperature}, \ref{eq:rheological-model} \\
Temperature (dynamic) & \(\hat{T}\) & K & \ref{eq:density-ala},
\ref{eq:gibbs}, \ref{eq:rheological-model} \\
Thermal conductivity (reference) & \(\bar{k}\) & W m\(^{-1}\) K\(^{-1}\)
& \ref{eq:energy} \\
Thermal expansivity (reference) & \(\bar{\alpha}\) & Pa\(^{-1}\) &
\ref{eq:energy}, \ref{eq:adiabatic-temperature}, \ref{eq:density-ala} \\
Thermal viscosity exponent factor & \(B\) & - &
\ref{eq:rheological-model} \\
Time & \(t\) & s &
\ref{eq:continuity-compressible}--\ref{eq:continuity-expanded},
\ref{eq:volume-fraction},
\ref{eq:phase-transition-rate-short}--\ref{eq:composition} \\
Velocity & \(\vec{u}\) & m s\(^{-1}\) &
\ref{eq:continuity-compressible}--\ref{eq:continuity-expanded},
\ref{eq:composition} \\
Viscosity & \(\eta\) & Pa s & \ref{eq:rheological-model} \\
Viscosity (reference) & \(\bar{\eta}\) & Pa s &
\ref{eq:rheological-model} \\
Volume fraction & \(X\) & - & \ref{eq:volume-fraction},
\ref{eq:phase-transition-rate-short}--\ref{eq:composition} \\
Water content & \(C_\mathrm{OH}\) & ppm & \ref{eq:growth-rate} \\
Water content exponent & \(n\) & - & \ref{eq:growth-rate} \\
\end{longtable}
}

\cleardoublepage

\section{Introduction}\label{sec:introduction}

Earth's mantle transition zone hosts two prominent seismic
discontinuities near 410 and 660 km depth, attributed to polymorphic
phase transitions of olivine (\citeproc{ref-katsura2004}{{Katsura et
al.}, 2004}; \citeproc{ref-ringwood1975}{Ringwood, 1975}). While these
discontinuities are observed globally, their detailed seismic
characteristics---depth, sharpness, amplitude, and lateral
continuity---vary substantially between tectonic settings
(\citeproc{ref-deuss2009}{Deuss, 2009}; \citeproc{ref-fukao2013}{Fukao
\& Obayashi, 2013}; \citeproc{ref-lawrence2008}{Lawrence \& Shearer,
2008}; \citeproc{ref-schmerr2007}{Schmerr \& Garnero, 2007}). Some
regions display sharp, high-amplitude reflectors consistent with abrupt
mineralogical boundaries, while others exhibit broad, weakened, or
laterally variable signals. Such heterogeneity cannot be explained by
equilibrium thermodynamics alone, which relates discontinuity topography
mainly to temperature-dependent phase boundaries defined by Clapeyron
slopes. Additional physical processes---including phase transition
kinetics and dynamic pressure effects---likely contribute to the
observed variability (\citeproc{ref-faccenda2017}{Faccenda \& Dal Zilio,
2017}; \citeproc{ref-rubie1994}{Rubie \& Ross II, 1994}).

Laboratory studies provide crucial insights into the mechanisms
controlling the olivine \(\Leftrightarrow\) wadsleyite phase transition
at 410 km depth (referred to hereafter as the nominal ``410''). Mineral
physics experiments consistently demonstrate that this transition is
strongly rate-limited, with kinetics governed by temperature, pressure,
water content, bulk chemical composition, grain size, and
microstructural evolution (\citeproc{ref-hosoya2005}{Hosoya et al.,
2005}; \citeproc{ref-kubo2004}{Kubo et al., 2004};
\citeproc{ref-ledoux2023}{{Ledoux et al.}, 2023};
\citeproc{ref-perrillat2013}{Perrillat et al., 2013};
\citeproc{ref-rubie1994}{Rubie \& Ross II, 1994}). In cold subducting
slabs, sluggish reaction rates can allow metastable olivine to persist
tens of kilometers below its thermodynamic stability limit, promoting
slab stagnation and triggering deep earthquakes via transformational
faulting (\citeproc{ref-green1995}{Green \& Houston, 1995};
\citeproc{ref-ishii2021}{Ishii \& Ohtani, 2021};
\citeproc{ref-kirby1996}{Kirby et al., 1996};
\citeproc{ref-ohuchi2022}{Ohuchi et al., 2022};
\citeproc{ref-rubie1994}{Rubie \& Ross II, 1994};
\citeproc{ref-sindhusuta2025}{Sindhusuta et al., 2025}). In hot
upwellings beneath mantle plumes, slow kinetics may broaden and uplift
the discontinuity, possibly explaining reduced seismic amplitudes
observed beneath some hotspots (\citeproc{ref-chambers2005}{Chambers et
al., 2005}). However, because published kinetic models remain poorly
constrained, with parameters spanning several orders of magnitude (e.g.,
\citeproc{ref-hosoya2005}{Hosoya et al., 2005}), the effects of
micro-scale kinetic processes on mantle-scale dynamics and seismic
observables are ambiguous.

Bridging the gap between laboratory-derived kinetic rate laws and
mantle-scale seismic observations requires numerical models that couple
phase transition kinetics to realistic treatments of mantle convection.
Previous modeling efforts have demonstrated qualitatively that kinetics
can strongly influence mantle flow (\citeproc{ref-agrusta2017}{Agrusta
et al., 2017}; \citeproc{ref-faccenda2017}{Faccenda \& Dal Zilio,
2017}), but systematic investigations quantifying the sensitivity of 410
structure to kinetic parameters remain limited. Moreover, most prior
studies employ simplified treatments of mantle compressibility that may
inadequately capture feedbacks among density changes, pressure
perturbations, and phase transition rates.

This study seeks to clarify these unresolved issues by implementing an
interface-controlled growth model (after
\citeproc{ref-hosoya2005}{Hosoya et al., 2005}) within the geodynamic
software ASPECT. We systematically explore how thermodynamics and
kinetics interact to control the structure of the 410 within a
compressible treatment of mantle flow. Specifically, we address the
following questions:

\begin{enumerate}
\def\labelenumi{\arabic{enumi}.}
\tightlist
\item
  How do coupled feedbacks among thermodynamic driving forces, phase
  transition rates, and compressible flow shape 410 structure?
\item
  What are the characteristic timescales and length scales over which
  these feedbacks operate?
\item
  How do kinetic effects differ between plume and slab environments?
\item
  Can seismic observations of 410 structure constrain effective kinetic
  parameters in Earth's mantle?
\end{enumerate}

To investigate these questions, we analyze a suite of numerical
experiments that vary kinetic parameters across seven orders of
magnitude, encompassing a wide range of water contents and grain sizes.
For each experiment, we quantify 410 displacement and width, enabling
direct comparisons with seismological observations. Our simulations
reveal that plumes and slabs respond differently to kinetic variations
and establish quantitative relationships between phase transition rates,
flow dynamics, and 410 structure. More broadly, our results demonstrate
that realistic treatment of phase transition kinetics is essential for
accurately modeling subduction dynamics and interpreting seismic
structures.

\cleardoublepage

\section{Methods}\label{sec:methods}

\subsection{Governing Equations for Compressible Mantle
Flow}\label{sec:governing-equations}

Mantle flow is simulated using the finite-element geodynamic code ASPECT
(v3.0.0, \citeproc{ref-aspect-doi-v3.0.0}{Bangerth et al., 2024a},
\citeproc{ref-aspectmanual}{2024b};
\citeproc{ref-clevenger2021}{Clevenger \& Heister, 2021};
\citeproc{ref-fraters2019}{{Fraters et al.}, 2019};
\citeproc{ref-fraters2020}{Fraters, 2020};
\citeproc{ref-gassmoller2018}{Gassmöller et al., 2018};
\citeproc{ref-heister2017}{Heister et al., 2017};
\citeproc{ref-kronbichler2012}{Kronbichler et al., 2012}) to find the
velocity \(\vec{u}\), pressure \(P\), and temperature \(T\) fields that
satisfy the following equations:

\begin{equation}
  \nabla P - \nabla \cdot \sigma^{\prime} = \rho\, g
  \label{eq:navier-stokes-no-inertia}
\end{equation}

\begin{equation}
  \frac{\partial \rho}{\partial t} + \nabla \cdot (\rho\, \vec{u}) = 0
  \label{eq:continuity-compressible}
\end{equation}

\begin{equation}
  \rho\, \bar{C}_p \left(\frac{\partial T}{\partial t} + \vec{u} \cdot \nabla T \right) - \nabla \cdot \left(\bar{k}\, \nabla T \right) = \sigma^{\prime} : \dot{\epsilon}^{\prime} + \bar{\alpha}\, T \left(\vec{u} \cdot \nabla P \right) + Q_L
  \label{eq:energy}
\end{equation}

where \(\sigma^{\prime}\) is the deviatoric stress tensor, \(\rho\) is
density, \(g\) is gravitational acceleration, \(t\) is time,
\(\bar{C}_p\), \(\bar{k}\), \(\bar{\alpha}\) are the reference specific
heat capacity, thermal conductivity, and thermal expansivity,
respectively (see Section \ref{sec:adiabatic-reference-conditions}), and
\(Q_L\) is the latent heat released or absorbed during phase
transitions. Equations \ref{eq:navier-stokes-no-inertia} and
\ref{eq:continuity-compressible} together describe the buoyancy-driven
flow of an isotropic fluid with negligible inertia and Equation
\ref{eq:energy} describes the conduction, advection, and production (or
consumption) of thermal energy (\citeproc{ref-schubert2001}{Schubert et
al., 2001}). Note that the pressure \(P\) in this context is equal to
the mean normal stress and is positive under compression:
\(P = - \frac{\sigma_{xx} + \sigma_{yy}}{2}\) (see Appendix
\ref{sec:momentum-derivation}).

The compressible form of the continuity equation (Equation
\ref{eq:continuity-compressible}) can result in dynamic feedbacks
between density changes and pressures that cause numerical oscillations
when the advection timestep is less than the viscous relaxation
timescale (\citeproc{ref-curbelo2019}{Curbelo et al., 2019}). Therefore
the continuity equation is reformulated using the \emph{projected
density approximation} (PDA, \citeproc{ref-gassmoller2020}{Gassmöller et
al., 2020}) by applying the product rule to
\(\nabla \cdot (\rho\, \vec{u})\) and multiplying both sides of Equation
\ref{eq:continuity-compressible} by \(\frac{1}{\rho}\) to obtain the
following expression:

\begin{equation}
  \frac{1}{\rho} \frac{\partial \rho}{\partial t} + \nabla \cdot \vec{u} + \left(\frac{1}{\rho} \nabla \rho \right) \cdot \vec{u} = 0
  \label{eq:continuity-expanded}
\end{equation}

This PDA formulation maintains the inherent coupling among density,
pressure, temperature, and phase transitions while mitigating the
numerical instabilities associated with compressible flow
(\citeproc{ref-gassmoller2020}{Gassmöller et al., 2020}). These coupled
feedbacks would otherwise be neglected by incompressible formulations of
the continuity equation (e.g., Boussinesq). This makes the PDA approach
particularly suitable for our numerical experiments, which incorporate
density changes due to dynamic PT effects and the olivine
\(\Leftrightarrow\) wadsleyite phase transition.

\subsection{Numerical Setup}\label{sec:numerical-setup}

\subsubsection{Adiabatic Reference
Conditions}\label{sec:adiabatic-reference-conditions}

To ensure numerical convergence, we initialized our ASPECT simulations
with reasonable initial estimates of the pressure-temperature (PT)
fields and material properties in Earth's upper mantle
(\citeproc{ref-aspectmanual}{Bangerth et al., 2024b};
\citeproc{ref-heister2017}{Heister et al., 2017};
\citeproc{ref-kronbichler2012}{Kronbichler et al., 2012}). We began by
evaluating entropy changes over a PT range of 1573--1973 K and 0.001--25
GPa (Figure \ref{fig:isentrope}) using the Gibbs free energy
minimization software Perple\_X (v.7.0.9,
\citeproc{ref-connolly2009}{Connolly, 2009}). We assumed a dry pyrolitic
bulk composition after Green et al. (\citeproc{ref-green1979}{1979}) and
phase equilibria were evaluated in the
Na\(_2\)O‐CaO‐FeO‐MgO‐Al\(_2\)O\(_3\)‐SiO\(_2\) (NCFMAS) chemical system
with thermodynamic data and solution models of Stixrude \&
Lithgow-Bertelloni (\citeproc{ref-stixrude2022}{2022}). Equations of
state were included for solid solution phases: olivine, plagioclase,
spinel, clinopyroxene, wadsleyite, ringwoodite, perovskite,
ferropericlase, high‐pressure C2/c pyroxene, orthopyroxene, akimotoite,
post‐perovskite, Ca‐ferrite, garnet, and Na‐Al phase.

\begin{figure}
\centering
\includegraphics[width=0.7\linewidth,height=\textheight,keepaspectratio,alt={Entropy (a) and density (b) changes in Earth's upper mantle under thermodynamic equilibrium and hydrostatic stress conditions. Material properties were computed with Perple\_X using the equations of state and thermodynamic data of Stixrude \& Lithgow-Bertelloni (2022). The black box indicates the approximate PT range of our ASPECT simulations, while the white line indicates the isentropic adiabat used to calculate reference material properties.}]{../figs/PYR-material-table.png}
\caption{Entropy (a) and density (b) changes in Earth's upper mantle
under thermodynamic equilibrium and hydrostatic stress conditions.
Material properties were computed with Perple\_X using the equations of
state and thermodynamic data of Stixrude \& Lithgow-Bertelloni
(\citeproc{ref-stixrude2022}{2022}). The black box indicates the
approximate PT range of our ASPECT simulations, while the white line
indicates the isentropic adiabat used to calculate reference material
properties.}\label{fig:isentrope}
\end{figure}

We then determined the mantle adiabat by applying the Newton--Raphson
algorithm to find temperatures corresponding to each pressure that
maintain constant entropy (white line in Figure \ref{fig:isentrope}).
Material properties were evaluated at each PT point along the isentrope
to construct the adiabatic reference conditions shown in Figure
\ref{fig:material-property-profile}. These reference conditions serve
three main purposes: 1) initializing the PT fields and material
properties in our ASPECT simulations (see Section
\ref{sec:initialization-and-boundary-conditions}), 2) updating the
material model during the simulations (see Section
\ref{sec:material-model}), and 3) serving as a basis for computing
``dynamic'' quantities, such as the dynamic temperature
\(\hat{T} = T - \bar{T}\), dynamic pressure \(\hat{P} = P - \bar{P}\),
and dynamic density \(\hat{\rho} = \rho - \bar{\rho}\), that quantify
how much the approximate numerical solution deviates from the reference
adiabatic conditions (i.e., a non-convecting ambient mantle).

\begin{figure}
\centering
\pandocbounded{\includegraphics[keepaspectratio,alt={Reference material properties used in our ASPECT simulations. Profiles were computed using the BurnMan software (Cottaar et al., 2014; Myhill et al., 2023) and were based on the equations of state and thermodynamic data of Stixrude \& Lithgow-Bertelloni (2022) for pure Mg olivine (ol) and wadsleyite (wd).}]{../figs/material-property-profile.png}}
\caption{Reference material properties used in our ASPECT simulations.
Profiles were computed using the BurnMan software
(\citeproc{ref-cottaar2014}{Cottaar et al., 2014};
\citeproc{ref-myhill2023}{Myhill et al., 2023}) and were based on the
equations of state and thermodynamic data of Stixrude \&
Lithgow-Bertelloni (\citeproc{ref-stixrude2022}{2022}) for pure Mg
olivine (ol) and wadsleyite (wd).}\label{fig:material-property-profile}
\end{figure}

\subsubsection{Initialization and Boundary
Conditions}\label{sec:initialization-and-boundary-conditions}

Our ASPECT simulations used a 900 \(\times\) 600 km rectangular model
domain initialized with pure Mg olivine and wadsleyite (Figure
\ref{fig:initial-setup}). ``Surface'' PT conditions of 10 GPa and 1706 K
were applied at the top boundary such that the olivine
\(\Leftrightarrow\) wadsleyite transition occurs at approximately 130 km
from the top boundary. The initial PT fields were then computed by
numerically integrating the following equations:

\begin{equation}
  \frac{d\bar{T}}{dy} = \frac{\bar{\alpha}\, \bar{T}\, g}{\bar{C}_p}
  \label{eq:adiabatic-temperature}
\end{equation}

\begin{equation}
  \frac{d\bar{P}}{dy} = \bar{\rho}\, g
  \label{eq:adiabatic-pressure}
\end{equation}

where \(d\bar{P}/dy\) and \(d\bar{T}/dy\) are adiabatic PT profiles
applied uniformly across the model domain, \(g\) is gravitational
acceleration, and the density \(\bar{\rho}\), thermal expansivity
\(\bar{\alpha}\), and specific heat capacity \(\bar{C}_p\) were
determined from the adiabatic reference conditions shown in Figure
\ref{fig:material-property-profile}. Gaussian-shaped thermal anomalies
of \(\pm\) 500 K were then applied along lines of length \(L\) extending
from the top and bottom boundaries for slab and plume simulations,
respectively.

\begin{figure}
\centering
\includegraphics[width=0.5\linewidth,height=\textheight,keepaspectratio,alt={Initial setup for slab (top) and plume (bottom) simulations. The free-slip (left and right), open (top or bottom), and prescribed inflow (top or bottom) boundary conditions for slab and plume simulations were essentially mirrored. Prescribed inflow velocities (\textbackslash vec\{v\} = 5 cm/yr) act parallel to the thermal anomalies, which are held at a fixed temperature at the boundary. The normal hydrostatic stress component (\textbackslash sigma\_\{yy\}) at the open boundary ensures that outflows are driven by dynamic pressures. The top boundary has a constant ``surface'' PT of 10 GPa and 1706 K such that the olivine \textbackslash Leftrightarrow wadsleyite phase transition (dashed line) occurs at 130 km from the top boundary.}]{../figs/initial-setup.png}
\caption{Initial setup for slab (top) and plume (bottom) simulations.
The free-slip (left and right), open (top or bottom), and prescribed
inflow (top or bottom) boundary conditions for slab and plume
simulations were essentially mirrored. Prescribed inflow velocities
(\(\vec{v}\) = 5 cm/yr) act parallel to the thermal anomalies, which are
held at a fixed temperature at the boundary. The normal hydrostatic
stress component (\(\sigma_{yy}\)) at the open boundary ensures that
outflows are driven by dynamic pressures. The top boundary has a
constant ``surface'' PT of 10 GPa and 1706 K such that the olivine
\(\Leftrightarrow\) wadsleyite phase transition (dashed line) occurs at
130 km from the top boundary.}\label{fig:initial-setup}
\end{figure}

Velocity and stress boundary conditions were set to ensure that constant
prescribed inflows of slab or plume material were balanced by free
outflows at the opposite open boundaries. Prescribed inflow velocities
of \(\vec{u}\) = 5 cm/yr were applied along the top (slabs) or bottom
(plumes) boundaries parallel to the thermal anomalies, decaying smoothly
to \(\vec{u}\) = 0 cm/yr at \(\pm\) 15 km from the thermal anomaly
centers. The left and right boundaries are free-slip (\(\sigma_{xy}\) =
\(\bar{P}\), \(\vec{u}_x\) = 0), and the top (slab) or bottom (plume)
boundaries have constant normal stress component \(\sigma_{yy}\) equal
to the initial hydrostatic pressure \(\bar{P}\) as determined by
numerical integration of Equation \ref{eq:adiabatic-pressure}. The
hydrostatic stress conditions at the open boundaries (bottom for slabs;
top for plumes) ensure that outflows are driven by dynamic pressures due
ot convection and/or volume changes during the olivine
\(\Leftrightarrow\) wadsleyite phase transition.

\subsubsection{Material Model}\label{sec:material-model}

\paragraph{Material Properties}\label{sec:material-properties}

Material properties were updated during our ASPECT simulations by
referencing the adiabatic reference conditions shown in Figure
\ref{fig:material-property-profile}. Except for density, material
properties received no PT corrections, effectively assuming that
deviations from a non-convecting ambient mantle were negligible. For
density, however, we applied a dynamic PT correction through a
first-order Taylor expansion (\citeproc{ref-gassmoller2020}{Gassmöller
et al., 2020}; \citeproc{ref-jarvis1980}{Jarvis \& Mckenzie, 1980}):

\begin{equation}
  \rho \approx \bar{\rho} + \left(\frac{\partial \bar{\rho}}{\partial P} \right)_T \Delta P + \left(\frac{\partial \bar{\rho}}{\partial T} \right)_P \Delta T
  \label{eq:density-ala-expansion}
\end{equation}

Equation \ref{eq:density-ala-expansion} is rewritten using standard
thermodynamic relations
\(\beta = \frac{1}{\rho} \left(\frac{\partial \rho}{\partial P}\right)_T\)
and
\(\alpha = -\frac{1}{\rho} \left(\frac{\partial \rho}{\partial T}\right)_P\)
to obtain the expression:

\begin{equation}
  \rho = \bar{\rho} \left(1 + \bar{\beta}\, \hat{P} - \bar{\alpha}\, \hat{T} \right)
  \label{eq:density-ala}
\end{equation}

where \(\bar{\rho}\), \(\, \bar{\beta}\), \(\, \bar{\alpha}\), are the
adiabatic reference density, compressibility, and thermal expansivity,
respectively, and \(\Delta P = \hat{P} = P - \bar{P}\) and
\(\Delta T = \hat{T} = T - \bar{T}\) are the dynamic PT. Note that the
reference thermal conductivity \(\bar{k}\) = 4.0 Wm\(^{-1}\)K\(^{-1}\)
is constant in all our numerical experiments.

\paragraph{Phase Transition
Kinetics}\label{sec:phase-transition-kinetics}

The kinetics of the olivine \(\Leftrightarrow\) wadsleyite phase
transition were governed entirely by interface-controlled growth, as
nucleation was assumed to saturate rapidly and did not limit the phase
transition (\citeproc{ref-cahn1956}{Cahn, 1956}). Following Faccenda \&
Dal Zilio (\citeproc{ref-faccenda2017}{2017}), the transformed volume
fraction is given by:

\begin{equation}
  X = 1 - \exp\left(-N\, \dot{x}\, t \right)
  \label{eq:volume-fraction}
\end{equation}

where \(X\) is the volume fraction of the product phase (olivine or
wadsleyite), \(N\) is a geometric factor that accounts for nucleation
sites, \(\dot{x}\) is the growth rate, and \(t\) is the elapsed time
after site saturation. For inter-crystalline grain-boundary controlled
growth, \(N = 6.67/d\), where \(d\) is grain size.

Since we assumed interface-controlled growth kinetics, the following
expression determined the overall phase transition rate
(\citeproc{ref-hosoya2005}{Hosoya et al., 2005}):

\begin{equation}
  \dot{x} = A\, T\, C_\mathrm{OH}^n\, \exp\left(-\frac{H^{\ast} + P V^{\ast}}{R\, T}\right) \left(1 - \exp\left[-\frac{\Delta G}{R\, T}\right] \right)
  \label{eq:growth-rate}
\end{equation}

where \(A\) is a kinetic prefactor, \(C_\mathrm{OH}\) is the
concentration of water in the reactant phase, \(n\) is the water content
exponent, \(H^{\ast}\) is activation enthalpy, \(V^{\ast}\) is
activation volume, \(P\) is pressure, \(T\) is temperature, \(R\) is the
gas constant, and \(\Delta G\) is the Gibbs free energy difference
between olivine and wadsleyite, which is approximated by:

\begin{equation}
  \Delta G \approx \Delta \bar{G} + \hat{P}\, \Delta \bar{V} - \hat{T}\, \Delta \bar{S}
  \label{eq:gibbs}
\end{equation}

where \(\Delta \bar{G}\), \(\Delta \bar{V}\), and \(\Delta \bar{S}\) are
the molar Gibbs free energy, volume, and entropy differences between
olivine and wadsleyite along the adiabatic reference profile (Figure
\ref{fig:thermodynamic-property-profile}), respectively, and \(\hat{P}\)
and \(\hat{T}\) are the dynamic PT.

\begin{figure}
\centering
\includegraphics[width=0.8\linewidth,height=\textheight,keepaspectratio,alt={Reference thermodynamic properties used in our ASPECT simulations. Profiles were computed using the same methods as described in Figure  (see Section ).}]{../figs/thermodynamic-property-profile.png}
\caption{Reference thermodynamic properties used in our ASPECT
simulations. Profiles were computed using the same methods as described
in Figure \ref{fig:material-property-profile} (see Section
\ref{sec:adiabatic-reference-conditions}).}\label{fig:thermodynamic-property-profile}
\end{figure}

In this formulation, the time evolution of the olivine
\(\Leftrightarrow\) wadsleyite phase transition is fully described by
the interplay of pressure, temperature, and kinetic parameters applied
to the interface-controlled growth model, without explicit consideration
of nucleation kinetics (\citeproc{ref-faccenda2017}{Faccenda \& Dal
Zilio, 2017}; \citeproc{ref-hosoya2005}{Hosoya et al., 2005}). The
macro-scale olivine \(\Leftrightarrow\) wadsleyite phase transition rate
was therefore computed by taking the time derivative of Equation
\ref{eq:volume-fraction}:

\begin{equation}
  \frac{dX}{dt} = \dot{X} = N\, \dot{x}\, \left(1 - X \right)
  \label{eq:phase-transition-rate-short}
\end{equation}

To simplify our numerical implementation of Equation
\ref{eq:phase-transition-rate-short}, we combined the parameters \(N\),
\(A\), and \(C_\mathrm{OH}^n\) into a single kinetic factor
\(Z = \frac{6.67}{d}\, A\, C_\mathrm{OH}^n\). Thus, the full expression
for the phase transition rate is:

\begin{equation}
  \dot{X} = Z\, T\, \exp\left(-\frac{H^{\ast} + P V^{\ast}}{R\, T}\right) \left(1 - \exp\left[-\frac{\Delta G}{R\, T}\right] \right)\, \left(1 - X \right)
  \label{eq:phase-transition-rate}
\end{equation}

The range of kinetic factors \(Z\) used in our numerical experiments
(3.0e0--7.0e7 K s\(^{-1}\)) is consistent with the kinetic growth rate
parameters reported in Hosoya et al. (\citeproc{ref-hosoya2005}{2005}).
We determined the range of \(Z\) by holding \(A\) = \(e^{-18}\) m
s\(^{-1}\) K\(^{-1}\) ppm\(_\mathrm{OH}^{-n}\), \(H^{\ast}\) = 274 kJ
mol\(^{-1}\), \(V^{\ast}\) = 3.0e-6 m\(^3\) mol\(^{-1}\), and \(n\) =
3.2 constant, while varying water content \(C_\mathrm{OH}\) from
50--5000ppm and grain size \(d\) from 1--10 mm.

\paragraph{Operator Splitting}\label{sec:operator-splitting}

Since the phase transition rate \(\dot{X}\) is faster than the advection
timescale in our ASPECT simulations, we employ a first-order operator
splitting scheme to decouple advection from interface-controlled growth
kinetics. In this approach, the phase fraction \(X\) is updated in two
sequential steps within each overall time step \(\Delta t\):

\begin{equation}
  \frac{\partial X}{\partial t} + \vec{u} \cdot \nabla X = 0
  \label{eq:composition}
\end{equation}

\begin{enumerate}
\def\labelenumi{\arabic{enumi}.}
\tightlist
\item
  \textbf{Advection step:} Solve the transport of material
  \(\left(\vec{u} \cdot \nabla X \right)\) without phase changes over
  the time interval \(\Delta t\) to yield an intermediate composition
  \(X^\ast\)
\item
  \textbf{Reaction step:} Starting from \(X^\ast\), integrate Equation
  \ref{eq:phase-transition-rate} over the same time interval using a
  smaller sub-step \(\delta t \le \Delta t\) to obtain the updated
  composition \(X^{n+1}\)
\end{enumerate}

This operator splitting scheme ensures numerical stability while
accurately capturing fast kinetics without restricting the convective
timestep (\citeproc{ref-aspectmanual}{Bangerth et al., 2024b}).

\subsubsection{Rheological Model}\label{sec:rheological-model}

Our ASPECT simulations use a simple rheological model where mantle
viscosity is modified by a depth-dependent piecewise function:

\begin{equation}
  \bar{\eta} =
  \begin{cases}
    1\, \eta_0, & y \leq 130\, \text{km} \\
    1\, \eta_0, & y > 130\, \text{km}
  \end{cases}
  \label{eq:piecewise-viscosity-function}
\end{equation}

where \(\eta_0\) = 1e21 is the nominal background viscosity of the upper
mantle (\citeproc{ref-karato2008}{Karato, 2008};
\citeproc{ref-ranalli1995}{Ranalli, 1995}), which we assume is identical
for pure olivine and wadsleyite. A thermal dependency is implemented
through an exponential term:

\begin{equation}
  \eta = \bar{\eta} \exp \left(-B\, \frac{\hat{T}}{\bar{T}} \right)
  \label{eq:rheological-model}
\end{equation}

where \(B\) = 1 is the thermal viscosity exponent factor, and
\(\hat{T}\) is the dynamic temperature.

\cleardoublepage

\section{Results}\label{sec:results}

\subsection{Simulation Snapshots: Slabs and
Plumes}\label{sec:simulation-snapshots}

Figures \ref{fig:slab-composition-set2} and
\ref{fig:plume-composition-set2} illustrate how thermodynamic driving
forces, phase transition rates, and compressible flow interact to shape
the 410 discontinuity. These snapshots, taken after 100 Ma of evolution,
provide visual context for the quantitative analysis in Section
\ref{sec:410-displacement-width}.

In slab simulations, ultra-sluggish kinetics (Figure
\ref{fig:slab-composition-set2}a--c) allow metastable olivine to persist
deep into the transition zone. This inhibition causes the slab to
stagnate and pond, depressing the 410. Within the cold, metastable
olivine region, Gibbs free energy accumulates and wadsleyite saturation
remains low until the thermodynamic driving force overcomes kinetic
barriers. Once this threshold is reached, the olivine
\(\Leftrightarrow\) wadsleyite reaction rapidly completes, producing a
sharp 410 that is displaced downwards by tens of kilometers.

At intermediate phase transition rates (Figure
\ref{fig:slab-composition-set2}d--f), the olivine \(\Leftrightarrow\)
wadsleyite reaction still lags but is fast enough to limit widespread
olivine metastability and avoid total slab stagnation. The resulting 410
is broad and diffuse, as density and seismic velocity contrasts
gradually fade with depth. This moderately-sluggish kinetic regime
produces complex 410 structures through intermediate phase transition
rates, incomplete slab stagnation, and deflected flow patterns. However,
when phase transition rates are sufficiently fast to maintain
quasi-equilibrium conditions (Figure
\ref{fig:slab-composition-set2}g--i), the 410 sharpens and rapid
wadsleyite growth within the slab allows continuous slab descent through
the 410 without hesitation.

In plume simulations, thermodynamics dominate mantle flow dynamics and
410 structure. Even under ultra-sluggish kinetics (Figure
\ref{fig:plume-composition-set2}a--c), the high temperatures of
upwellings prevent significant olivine metastablility. The olivine
\(\Leftrightarrow\) wadsleyite transition proceeds rapidly, maintaining
thin, sharp 410 interfaces and strong density and seismic contrasts.
Although ultra-sluggish kinetics slightly broaden and uplift the 410,
reducing buoyancy contrasts, plume structures remain vertically coherent
across the full range of tested kinetic factors (Figure
\ref{fig:plume-composition-set2}).

Altogether, these simulations demonstrate that in cold environments,
kinetics strongly influence slab dynamics and control whether the 410
appears as a diffuse, low-amplitude feature or as a sharp, high-contrast
seismic boundary. In contrast, thermodynamics dominate in hot plume
environments, producing stable, sharply defined 410s that are largely
independent of kinetic factors.

\begin{figure}
\centering
\pandocbounded{\includegraphics[keepaspectratio,alt={Slab simulation within ultra-sluggish (a--c: Z = 3.0e0 K s\^{}\{-1\}), intermediate (d--f: Z = 4.7e2 K s\^{}\{-1\}), and quasi-equilibrium (g--i: Z = 7.0e7 K s\^{}\{-1\}) kinetic regimes after 100 Ma evolution. Panels show dynamic temperature \textbackslash hat\{T\} (left column), dynamic density \textbackslash hat\{\textbackslash rho\} (middle column), and pressure-wave velocity V\_p (right column). Additional sets of visualized outputs are shown in Appendix .}]{../figs/simulation/compositions/slab-3.0e0-4.7e2-7.0e7-set2-composition-0100.png}}
\caption{Slab simulation within ultra-sluggish (a--c: \(Z\) = 3.0e0 K
s\(^{-1}\)), intermediate (d--f: \(Z\) = 4.7e2 K s\(^{-1}\)), and
quasi-equilibrium (g--i: \(Z\) = 7.0e7 K s\(^{-1}\)) kinetic regimes
after 100 Ma evolution. Panels show dynamic temperature \(\hat{T}\)
(left column), dynamic density \(\hat{\rho}\) (middle column), and
pressure-wave velocity \(V_p\) (right column). Additional sets of
visualized outputs are shown in Appendix
\ref{sec:simulation-snapshots-continued}.}\label{fig:slab-composition-set2}
\end{figure}

\begin{figure}
\centering
\pandocbounded{\includegraphics[keepaspectratio,alt={Plume simulation within ultra-sluggish (a--c: Z = 3.0e0 K s\^{}\{-1\}), intermediate-sluggish (d--f: Z = 4.7e2 K s\^{}\{-1\}), and quasi-equilibrium (g--i: Z = 7.0e7 K s\^{}\{-1\}) kinetic regimes after 100 Ma evolution. Panels show dynamic temperature \textbackslash hat\{T\} (left column), dynamic density \textbackslash hat\{\textbackslash rho\} (middle column), and pressure-wave velocity V\_p (right column). Additional sets of visualized outputs are shown in Appendix .}]{../figs/simulation/compositions/plume-3.0e0-4.7e2-7.0e7-set2-composition-0100.png}}
\caption{Plume simulation within ultra-sluggish (a--c: \(Z\) = 3.0e0 K
s\(^{-1}\)), intermediate-sluggish (d--f: \(Z\) = 4.7e2 K s\(^{-1}\)),
and quasi-equilibrium (g--i: \(Z\) = 7.0e7 K s\(^{-1}\)) kinetic regimes
after 100 Ma evolution. Panels show dynamic temperature \(\hat{T}\)
(left column), dynamic density \(\hat{\rho}\) (middle column), and
pressure-wave velocity \(V_p\) (right column). Additional sets of
visualized outputs are shown in Appendix
\ref{sec:simulation-snapshots-continued}.}\label{fig:plume-composition-set2}
\end{figure}

\cleardoublepage

\subsection{Structure of the 410: Displacement and
Width}\label{sec:410-displacement-width}

Figure \ref{fig:410-structure} summarizes the quantitative relationships
between 410 structure and the maximum phase transition rate
\(\dot{X}_{\mathrm{max}}\) evaluated in slab and plume simulations after
100 Ma of evolution. The results reveal fundamentally different
responses of plumes and slabs to thermodynamic driving forces, phase
transition rates, and compressible flow.

In plume simulations, the 410 shows little systematic variation with
\(\dot{X}_{\mathrm{max}}\). Its structure remains nearly constant across
seven orders of magnitude variation in \(\dot{X}_{\mathrm{max}}\), with
consistent displacements of -26 km and widths between 2--5 km. The only
exception is a few ultra-sluggish kinetic models where both displacement
and width increase slightly to -16 and 19 km, respectively (Table
\ref{tbl:depth-profile-summary}). The weak dependence of 410 structure
on \(\dot{X}_{\mathrm{max}}\) reflects the strong thermal control of the
reaction front in upwellings, where high temperatures promote rapid
wadsleyite \(\Leftrightarrow\) olivine transition---maintaining a sharp
discontinuity regardless of the kinetic factor \(Z\) applied to the
interface-controlled growth model (Equation
\ref{eq:phase-transition-rate}).

In slab simulations, the 410 exhibits distinct structural changes across
three kinetic regimes. At high phase transition rates (\(Z\) = 2.4e6 K
s\(^{-1}\); \(\dot{X}_{\mathrm{max}}\) \textgreater{} 10\(^1\)
Ma\(^{-1}\)), the olivine \(\Leftrightarrow\) wadsleyite transition
remains near thermodynamic equilibrium, producing a narrow 410
(\textless{} 5 km), displaced 36--39 km upwards within the slab's inner
core. As \(\dot{X}_{\mathrm{max}}\) decreases (4.7e2 \textless{} \(Z\)
\textless{} 2.4e6 K s\(^{-1}\); 10\(^{-1.5}\) \textless{}
\(\dot{X}_{\mathrm{max}}\) \textless{} 10\(^1\) Ma\(^{-1}\)), the 410
deepens and widens, forming a log-linear relationship where reductions
in \(\dot{X}_{\mathrm{max}}\) progressively broaden the reaction front
(Table \ref{tbl:depth-profile-summary}). This intermediate kinetic
regime corresponds to a partially inhibited olivine \(\Leftrightarrow\)
phase transition that proceeds slowly, hindering downward flow without
complete slab stagnation.

At the lowest phase transition rates (\(Z\) \textless{} 4.7e2 K
s\(^{-1}\); \(\dot{X}_{\mathrm{max}}\) \textless{} 10\(^{-1.5}\)
Ma\(^{-1}\)), a third kinetic regime emerges. While the 410 is displaced
downwards, its width narrows with further reductions in
\(\dot{X}_{\mathrm{max}}\). This ultra-sluggish kinetic regime reflects
a transition to strong disequilibrium conditions and complete slab
stagnation, where reaction progress becomes localized within the cold,
high-pressure regions of the ponding slab. The apparent sharpening of
the 410 under these conditions is due to a narrow portion of the phase
transition zone remaining reactive, while surrounding regions are
kinetically frozen.

In summary, 410 structure near plumes is regulated by thermal effects
near thermodynamic equilibrium, whereas 410 structure near slabs
exhibits distinct kinetic thresholds and non-linear scaling between its
width, displacement, and the phase transition rate \(\dot{X}\). These
contrasting behaviors underscore the differing roles of kinetics in hot
versus cold mantle environments and imply that the 410 beneath slabs can
transition abruptly between thermodynamically- and kinetically
controlled regimes as phase transition rates decrease.

\begin{figure}
\centering
\includegraphics[width=0.7\linewidth,height=\textheight,keepaspectratio,alt={Quantitative relationships between 410 structure and maximum phase transition rates evaluated in plume and slab simulations after 100 Ma. Structure of the 410 near plumes (left column) shows minimal dependence on \textbackslash dot\{X\}\_\{\textbackslash mathrm\{max\}\}, with both displacement and width remaining nearly constant across seven orders of magnitude variation in \textbackslash dot\{X\}\_\{\textbackslash mathrm\{max\}\}. In contrast, 410 structure near slabs (right column) changes distinctly across three kinetic regimes: (1) quasi-equilibrium at high \textbackslash dot\{X\}\_\{\textbackslash mathrm\{max\}\}, where 410 widths are narrow and displacements positive; (2) an intermediate regime where decreasing phase transition rates \textbackslash dot\{X\}\_\{\textbackslash mathrm\{max\}\} progressively widen and deepen the 410; and (3) an ultra-sluggish regime at low \textbackslash dot\{X\}\_\{\textbackslash mathrm\{max\}\}, where the 410 narrows while deepening, and slabs completely stall and pond above -100 km displacement.}]{../figs/410-structure.png}
\caption{Quantitative relationships between 410 structure and maximum
phase transition rates evaluated in plume and slab simulations after 100
Ma. Structure of the 410 near plumes (left column) shows minimal
dependence on \(\dot{X}_{\mathrm{max}}\), with both displacement and
width remaining nearly constant across seven orders of magnitude
variation in \(\dot{X}_{\mathrm{max}}\). In contrast, 410 structure near
slabs (right column) changes distinctly across three kinetic regimes:
(1) quasi-equilibrium at high \(\dot{X}_{\mathrm{max}}\), where 410
widths are narrow and displacements positive; (2) an intermediate regime
where decreasing phase transition rates \(\dot{X}_{\mathrm{max}}\)
progressively widen and deepen the 410; and (3) an ultra-sluggish regime
at low \(\dot{X}_{\mathrm{max}}\), where the 410 narrows while
deepening, and slabs completely stall and pond above -100 km
displacement.}\label{fig:410-structure}
\end{figure}

\cleardoublepage

\section{Discussion}\label{sec:discussion}

Our numerical simulations show that interface-controlled growth
kinetics, when coupled to a compressible treatment of mantle flow, exert
a first-order control on the geometry and seismic expression of the 410
discontinuity. The results presented in Section \ref{sec:results}
demonstrate that plume and slab flow dynamics respond in systematically
different ways: plumes are insensitive to kinetics due to high
temperatures, whereas slabs show three distinct kinetic regimes with
thresholded behavior (Figure \ref{fig:410-structure} and Table
\ref{tbl:depth-profile-summary}). The implications of such contrasting
relationships in slabs versus plumes are discussed below.

\subsection{Uncertainties and Model
Limitations}\label{sec:uncertainties-and-model-limitations}

The primary quantitative uncertainty in our analysis arises from the
kinetic factor \(Z\) in the interface-controlled growth model governing
the olivine \(\Leftrightarrow\) wadsleyite phase transition (Equation
\ref{eq:phase-transition-rate}). This factor spans several orders of
magnitude, reflecting a large range of water contents (50--5000 ppm) and
grain sizes (1--10 mm) that impact phase transition rates. These ranges
reflect experimental conditions from Hosoya et al.
(\citeproc{ref-hosoya2005}{2005}), previous numerical studies of
metastable olivine wedges (\citeproc{ref-rubie1994}{Rubie \& Ross II,
1994}), and typical grain sizes of upper mantle xenoliths
(\textasciitilde3--10 mm, \citeproc{ref-karato1984}{Karato, 1984};
\citeproc{ref-karato2008}{Karato, 2008}). Although we explicitly hold
the other kinetic parameters in Equation \ref{eq:growth-rate} constant,
laboratory studies show large uncertainties for \(n\), \(A\),
\(H^{\ast}\), and \(V^{\ast}\) that depend strongly on water content,
grain size, Mg-Fe composition, and microstructural evolution
(\citeproc{ref-hosoya2005}{Hosoya et al., 2005};
\citeproc{ref-kubo2004}{Kubo et al., 2004};
\citeproc{ref-ledoux2023}{{Ledoux et al.}, 2023};
\citeproc{ref-perrillat2013}{Perrillat et al., 2013};
\citeproc{ref-rubie1994}{Rubie \& Ross II, 1994}). Our simulations
therefore explore only a limited subset of the potential phase
transition rates in Earth's upper mantle.

Our kinetic model also relies on a key simplification: we assume
instantaneous nucleation site saturation followed by
interface-controlled growth (Equations
\ref{eq:volume-fraction}--\ref{eq:phase-transition-rate}), thereby
neglecting nucleation kinetics. This approach is often justified because
nucleation rates typically occur too rapidly to be measured reliably
(\citeproc{ref-faccenda2017}{Faccenda \& Dal Zilio, 2017};
\citeproc{ref-hosoya2005}{Hosoya et al., 2005};
\citeproc{ref-kubo2004}{Kubo et al., 2004};
\citeproc{ref-perrillat2016}{Perrillat et al., 2016}). However, recent
in-situ X-ray and acoustic studies (\citeproc{ref-ledoux2023}{{Ledoux et
al.}, 2023}; \citeproc{ref-ohuchi2022}{Ohuchi et al., 2022}) document
complex nucleation-growth microstructures that can limit net phase
transition rates under some PT conditions. Therefore, our saturated
nucleation assumption generally overestimates phase transition rates and
consequently underestimates olivine metastability and its effects on
flow dynamics and 410 structure.

Finally, compositional and rheological simplifications make our
simulations less representative of natural behavior. Assuming pure
Mg-rich end-members neglects Fe-partitioning and minor-element effects
that can shift equilibrium depths by \textasciitilde10--20 km and alter
kinetics (\citeproc{ref-katsura2004}{{Katsura et al.}, 2004};
\citeproc{ref-perrillat2013}{Perrillat et al., 2013},
\citeproc{ref-perrillat2016}{2016}). We also neglect the direct effects
of non-hydrostatic stress on microstructures and solid-state reactions
(\citeproc{ref-wheeler2014}{Wheeler, 2014},
\citeproc{ref-wheeler2018}{2018}, \citeproc{ref-wheeler2020}{2020}).
Moreover, our use of a simple temperature-dependent viscosity neglects
crucial rheological complexities such as grain-size evolution, plastic
deformation, and stress-dependent rheologies---factors known to modify
strain localization (\citeproc{ref-karato2001}{Karato et al., 2001}) and
thus impact dynamic feedbacks controlling 410 structure. These omissions
merit future investigation but suggest that the quantitative thresholds
reported here capture the primary, first-order effects.

\subsection{Implications for Subduction
Dynamics}\label{sec:implications-for-subduction-dynamics}

The three kinetic regimes identified in our slab
simulations---quasi-equilibrium, intermediate, and
ultra-sluggish---provide a framework for understanding how phase
transition kinetics control slab penetration and deep earthquake
activity in the mantle transition zone.

The absence of widespread slab ponding at the 410 in seismic tomography
(\citeproc{ref-fukao2013}{Fukao \& Obayashi, 2013}) constrains the
permissible kinetic conditions in Earth's mantle. The ultra-sluggish
kinetic regime (\(\dot{X}\) \textless{} 10\(^{-1.5}\) Ma\(^{-1}\)),
which produces complete stagnation in our simulations, appears
inconsistent with tomographic images that show continuous slab descent
through the 410. Most subduction zones must therefore experience
sufficiently rapid olivine \(\Leftrightarrow\) wadsleyite phase
transition rates to avoid complete stagnation, placing an upper bound on
the degree of metastability that can develop during typical subduction.

Yet this constraint alone cannot fully explain subduction zone behavior.
Deep earthquakes attributed to transformational faulting
(\citeproc{ref-green1995}{Green \& Houston, 1995};
\citeproc{ref-ishii2021}{Ishii \& Ohtani, 2021};
\citeproc{ref-kirby1996}{Kirby et al., 1996};
\citeproc{ref-ohuchi2022}{Ohuchi et al., 2022};
\citeproc{ref-sindhusuta2025}{Sindhusuta et al., 2025}) require olivine
persistence well into the wadsleyite stability field---evidence that
metastability does occur to a significant degree. Our simulations
suggest this behavior is consistent with the intermediate kinetic regime
(10\(^{-1.5}\) \textless{} \(\dot{X}\) \textless{} 10\(^1\)
Ma\(^{-1}\)), which generates localized buoyant regions of metastable
olivine that resist, but do not prevent, downward flow. The coexistence
of deep seismicity with continued slab penetration therefore requires a
delicate balance: phase transition rates must be slow enough to sustain
metastable volumes sufficient for transformational faulting, yet fast
enough to permit overall slab descent through the 410.

The dynamic threshold near \(\dot{X}\) \textasciitilde{} 10\(^{-1.5}\)
Ma\(^{-1}\) therefore represents a critical boundary for mantle
convection. Below this threshold, buoyancy forces from incomplete phase
transition overwhelm slab pull, arresting descent. Above it, phase
transitions proceed rapidly enough to permit 410 penetration despite
temporary kinetic resistance. The sensitivity of this
threshold---occurring over less than one order of magnitude in
\(\dot{X}\)---suggests that individual subduction zones could oscillate
between penetration and temporary stagnation as thermal or kinematic
conditions evolve, potentially explaining temporarily stalled slabs that
subsequently resume descent (\citeproc{ref-agrusta2017}{Agrusta et al.,
2017}).

Observed variability in slab behavior beneath different subduction
zones, ranging from rapid penetration to temporary stalling
(\citeproc{ref-fukao2013}{Fukao \& Obayashi, 2013}), likely reflects
regional differences influencing effective phase transition rates. The
nonlinear scaling between 410 structure and \(\dot{X}\) in our
simulations implies that relatively modest variations in slab thermal
structure, descent velocity, water content, or grain size could position
different subduction zones in different kinetic regimes. Young, hot
slabs descending slowly may maintain quasi-equilibrium conditions, while
old, cold slabs descending rapidly may approach the intermediate regime
where metastability becomes significant. This framework suggests that
subduction zone diversity arises not only from differences in plate age
and convergence rate but also from kinetically controlled feedbacks
between phase transitions and flow dynamics.

The fundamental asymmetry between plume and slab responses to kinetics
has important implications for numerical modeling. Thermal dominance in
hot upwellings renders plume dynamics insensitive to kinetics, whereas
the low temperatures in slabs amplify kinetic effects to the point of
controlling flow regime. This asymmetry implies that numerical
geodynamic models assuming thermodynamic equilibrium systematically
underestimate the role of metastability in subduction. Realistic
simulations of whole-mantle convection and long-term plate tectonic
evolution must therefore account for kinetically controlled phase
transitions, particularly in cold downwelling environments where
disequilibrium effects are most pronounced.

\subsection{Implications for 410
Detectability}\label{sec:410-detectability}

The relationships between phase transition kinetics and 410 structure
established in our simulations provide testable predictions for seismic
detectability. Sharp interfaces with widths of a few kilometers are
readily detected with SS precursors and receiver function stacks
(\citeproc{ref-chambers2005}{Chambers et al., 2005};
\citeproc{ref-deuss2009}{Deuss, 2009};
\citeproc{ref-shearer2000}{Shearer, 2000}), though regional
high-frequency receiver function approaches can resolve structures as
thin as \textasciitilde5 km under favorable conditions
(\citeproc{ref-dokht2016}{Dokht et al., 2016};
\citeproc{ref-frazer2023}{Frazer \& Park, 2023};
\citeproc{ref-helffrich1996}{Helffrich \& Wood, 1996};
\citeproc{ref-wei2017}{Wei \& Shearer, 2017}).

Our plume simulations predict consistently thin, easily detectable
discontinuities in hot upwelling environments. Across seven orders of
magnitude variation in \(\dot{X}\), plumes maintain 410 widths of 2--5
km with displacements of approximately -26 km, except under
ultra-sluggish kinetics where widths broaden to \textasciitilde19 km
(Table \ref{tbl:depth-profile-summary}). These sharp discontinuities
beneath plume regions should be readily detectable with standard seismic
methods, consistent with observations of well-defined 410s beneath
hotspots (\citeproc{ref-deuss2009}{Deuss, 2009};
\citeproc{ref-lawrence2008}{Lawrence \& Shearer, 2008}). However,
composition, anisotropy, and imaging methods also influence observed
sharpness, complicating unique interpretations.

Slab simulations predict more variable seismic detectability across the
three kinetic regimes. In the quasi-equilibrium regime (\(\dot{X}\)
\textgreater{} 10\(^1\) Ma\(^{-1}\)), narrow 410s with positive
displacements produce sharp, high-amplitude, readily detectible seismic
signals. In the intermediate regime (10\(^{-1.5}\) \textless{}
\(\dot{X}\) \textless{} 10\(^1\) Ma\(^{-1}\)), progressive broadening
systematically reduces detectability as the phase transition front
becomes diffuse, potentially rendering the 410 invisible to
lower-frequency seismic methods. In the ultra-sluggish regime
(\(\dot{X}\) \textless{} 10\(^{-1.5}\) Ma\(^{-1}\)), re-sharpening
produces detectable 410s despite substantial deepening (\textgreater{}
100 km displacement), reflecting localized phase transitioning in ponded
slabs.

These predictions offer a quantitative framework for interpreting
observed seismic heterogeneity. Reported 410 thickness variations
ranging from \textasciitilde5--30 km in Pacific regions
(\citeproc{ref-alex2004}{Alex Song et al., 2004};
\citeproc{ref-schmerr2007}{Schmerr \& Garnero, 2007}) can arise from
spatial changes in effective phase transition rates superimposed on
thermal and compositional heterogeneity. Broad, weakened 410 signals
beneath some subduction zones (\citeproc{ref-han2021}{Han et al., 2021};
\citeproc{ref-jiang2015}{Jiang et al., 2015}; \citeproc{ref-lee2014}{Lee
et al., 2014}; \citeproc{ref-shen2020}{Shen \& Zhan, 2020};
\citeproc{ref-vanstiphout2019}{Van Stiphout et al., 2019}) are
consistent with intermediate kinetic conditions where partial
metastability produces diffuse reaction fronts. Conversely, sharp 410s
in cold slabs suggest either quasi-equilibrium maintained by rapid
kinetics or localized phase transitioning in ultra-sluggish
conditions---though the latter should be accompanied by slab stagnation,
which is rarely observed at the 410 (\citeproc{ref-fukao2013}{Fukao \&
Obayashi, 2013}).

The intermediate kinetic regime presents a particular challenge for
seismic detection. Where 410 widths exceed \textasciitilde10--20 km, the
gradual density and velocity gradients may produce weak or absent
reflections in SS precursor and receiver function studies, even during
substantial phase transitioning. Such ``invisible'' 410s could be
misinterpreted as evidence for compositional anomalies or unusual
thermal structures when they actually reflect kinetically controlled
broadening. High-resolution tomographic studies that image continuous
velocity gradients rather than discrete discontinuities may better
detect these diffuse transition zones.

Within this kinetic framework, discontinuities observed near 500--600 km
depth (\citeproc{ref-cottaar2016}{Cottaar \& Deuss, 2016};
\citeproc{ref-deuss2001}{Deuss \& Woodhouse, 2001};
\citeproc{ref-saikia2008}{Saikia et al., 2008};
\citeproc{ref-tauzin2017}{Tauzin et al., 2017}) are not interpreted as
metastable olivine persisting past the 410. These features likely
reflect akimotoite formation, the wadsleyite \(\Leftrightarrow\)
ringwoodite transition, or more complex compositional layering. Broad or
absent discontinuities within the mantle transition zone may indicate
sluggish kinetics in addition to thermal or compositional influences,
but distinguishing these effects requires independent temperature and
composition constraints from complementary geophysical observations.

\subsection{Implications for Constraining Kinetic Parameters from
Seismic
Observations}\label{sec:constraining-kinetic-parameters-from-seismic-observations}

The contrasting sensitivities of plume and slab 410 structures to phase
transition kinetics (Figure \ref{fig:410-structure}) suggest different
strategies for extracting kinetic constraints from seismic observations.

For plumes, the near-independence of 410 structure from \(\dot{X}\)
limits the utility of seismic observations for constraining kinetics.
The consistent 410 widths of 2--5 km and displacements of approximately
-26 km observed in our simulations are primarily controlled by thermal
structure rather than kinetic rates. Only in the ultra-sluggish regime,
where 410 widths broaden to \textasciitilde19 km and displacements
decrease to -16 km, do kinetic effects become seismically
distinguishable. However, this regime represents extreme kinetic
inhibition unlikely to be widespread in hot upwelling environments.

For slabs, the three distinct kinetic regimes (Figure
\ref{fig:410-structure}) provide exploitable diagnostic signatures. The
threshold behavior near \(\dot{X}\) \textasciitilde{} 10\(^{-1.5}\)
Ma\(^{-1}\), which separates slab penetration from ponding, offers a
critical constraint: widespread observation of slabs penetrating the 410
in global tomography (\citeproc{ref-fukao2013}{Fukao \& Obayashi, 2013})
require effective phase transition rates exceeding this threshold in
most subduction zones. Regional variations in 410 topography and
sharpness can then be interpreted within the intermediate kinetic regime
(10\(^{-1.5}\) \textless{} \(\dot{X}\) \textless{} 10\(^1\)
Ma\(^{-1}\)), where the log-linear relationship between 410 width and
\(\dot{X}\) provides a potential inversion target.

Operationally, constraining kinetic parameters requires combining
high-resolution seismic imaging with independent thermal constraints.
Regional receiver function or SS precursor studies mapping spatial
variations in 410 thickness and displacement
(\citeproc{ref-chambers2005}{Chambers et al., 2005};
\citeproc{ref-deuss2009}{Deuss, 2009}; \citeproc{ref-houser2010}{Houser
\& Williams, 2010}; \citeproc{ref-lawrence2008}{Lawrence \& Shearer,
2008}; \citeproc{ref-schmerr2007}{Schmerr \& Garnero, 2007}) can be
compared against tomographic temperature estimates and our simulation
results to infer order-of-magnitude bounds on \(\dot{X}\). Such
inversions can potentially discriminate between grain sizes or dry
versus wet environments following the formulation of Hosoya et al.
(\citeproc{ref-hosoya2005}{2005}), though quantitative constraints
require reducing uncertainties in kinetic parameters through targeted
mineral physics experiments (e.g., \citeproc{ref-hosoya2005}{Hosoya et
al., 2005}; \citeproc{ref-kubo2004}{Kubo et al., 2004};
\citeproc{ref-ledoux2023}{{Ledoux et al.}, 2023};
\citeproc{ref-perrillat2013}{Perrillat et al., 2013},
\citeproc{ref-perrillat2016}{2016}).

Regions where thermal structure is relatively well-constrained but
seismic expression varies systematically are particularly valuable---for
example, across slabs with different ages, descent velocities, or
hydration states (\citeproc{ref-agius2017}{Agius et al., 2017};
\citeproc{ref-schmandt2012}{Schmandt, 2012};
\citeproc{ref-vanstiphout2019}{Van Stiphout et al., 2019}). In such
settings, thermal and compositional effects can be approximately
controlled, allowing lateral variations in 410 structure to isolate
kinetic influences. Similarly, comparing 410 structure in hotspots
versus ambient mantle, where temperature contrasts are independently
estimated, can test predictions of kinetically controlled broadening and
displacement in plumes (\citeproc{ref-chambers2005}{Chambers et al.,
2005}; \citeproc{ref-glasgow2024}{Glasgow et al., 2024};
\citeproc{ref-jenkins2016}{Jenkins et al., 2016}).

The primary limitation of seismic inversions for kinetics remains the
uncertainties in the kinetic parameters themselves (Equation
\ref{eq:growth-rate}). Our simulations span a broad range of \(Z\)
values reflecting a range of water contents and grain sizes, but the
other critical kinetic parameters \(n\), \(A\), \(H^{\ast}\), and
\(V^{\ast}\) remain largely unconstrained and were not tested here.
Until mineral physics experiments reduce these uncertainties, seismic
observations can provide order-of-magnitude estimates of effective
\(\dot{X}\) rather than precise determinations of micro-scale kinetic
parameters. Nevertheless, the threshold behavior and scaling
relationships emerging from slab simulations clarify how phase
transition kinetics can account for seismic heterogeneity.

\cleardoublepage

\section{Conclusions}\label{sec:conclusions}

The olivine \(\Leftrightarrow\) wadsleyite phase transition in Earth's
upper mantle is strongly influenced by coupled feedbacks among
thermodynamic driving forces, phase transition rates, and compressible
flow. We quantified how these factors contribute to 410 expression by
coupling an interface-controlled growth model to compressible
simulations of mantle plumes and slabs, systematically exploring 410
structure sensitivity to kinetic factors spanning seven orders of
magnitude.

Our results reveal fundamentally different responses in hot versus cold
mantle environments. Plume simulations produce consistently sharp
discontinuities, implying that seismic observations of 410s beneath
hotspots primarily constrain thermal structure near thermodynamic
equilibrium rather than phase transition rates. Slab simulations, in
contrast, exhibit distinct threshold behavior across three kinetic
regimes: quasi-equilibrium, intermediate, and ultra-sluggish. Widespread
observations of slabs penetrating the 410 in seismic tomography suggest
effective phase transition rates exceed the ultra-sluggish threshold
(\(\dot{X}\) \textgreater{} 10\(^{-1.5}\) Ma\(^{-1}\)), while regional
variations in 410 topography and sharpness likely reflect intermediate
kinetic conditions (10\(^{-1.5}\) Ma\(^{-1}\) \textless{} \(\dot{X}\)
\textless{} 10\(^1\) Ma\(^{-1}\)) modulated by local thermal structure,
water content, and grain size.

These findings demonstrate that phase transition kinetics exert
first-order control on slab dynamics and 410 structure beneath
subduction zones. The threshold separating penetrating from ponded slabs
near \(\dot{X}\) \textasciitilde10\(^{-1.5}\) Ma\(^{-1}\) directly links
mantle flow to micro-scale kinetic parameters and suggests that modest
variations in effective growth rates can produce substantial diversity
in observed 410 topography. The 410 can therefore serve as a
seismological probe of upper mantle kinetics, particularly in cold
subduction environments where kinetic effects are amplified. Realizing
this potential requires reducing uncertainties in kinetic parameters
through targeted mineral physics experiments that better quantify
nucleation versus growth mechanisms, water and compositional effects,
and microstructural evolution during deformation. Integrating such
experimental constraints with high-resolution seismic imaging and the
forward modeling framework presented here offers a pathway toward
understanding how micro-scale kinetic processes shape Earth's
large-scale interior structure.

\cleardoublepage

\section*{Acknowledgements}\label{acknowledgements}
\addcontentsline{toc}{section}{Acknowledgements}

This work was funded by the UKRI NERC Large Grant no. NE/V018477/1
awarded to John Wheeler at the University of Liverpool. All computations
were undertaken on Barkla2, part of the High Performance Computing
facilities at the University of Liverpool, who graciously provided
expert support. We thank the Computational Infrastructure for
Geodynamics (\url{geodynamics.org}) which is funded by the National
Science Foundation under award EAR-0949446 and EAR-1550901 for
supporting the development of ASPECT.

\section*{Data Availability}\label{data-availability}
\addcontentsline{toc}{section}{Data Availability}

All data, code, and relevant information for reproducing this work can
be found at
\url{https://github.com/buchanankerswell/kerswell_et_al_dynp}, and at
\ldots, the official Open Science Framework data repository. All code is
MIT Licensed and free for use and distribution (see license details).
ASPECT version 3.0.0, (\citeproc{ref-aspect-doi-v3.0.0}{Bangerth et al.,
2024a}, \citeproc{ref-aspectmanual}{2024b};
\citeproc{ref-clevenger2021}{Clevenger \& Heister, 2021};
\citeproc{ref-fraters2019}{{Fraters et al.}, 2019};
\citeproc{ref-fraters2020}{Fraters, 2020};
\citeproc{ref-gassmoller2018}{Gassmöller et al., 2018};
\citeproc{ref-heister2017}{Heister et al., 2017};
\citeproc{ref-kronbichler2012}{Kronbichler et al., 2012}) used in these
computations is freely available under the GPL v2.0 or later license
through its software landing page
\url{https://geodynamics.org/resources/aspect} or
\url{https://aspect.geodynamics.org} and is being actively developed on
GitHub and can be accessed via
\url{https://github.com/geodynamics/aspect}.

\cleardoublepage

\section*{References}\label{references}
\addcontentsline{toc}{section}{References}

\protect\phantomsection\label{refs}
\begin{CSLReferences}{1}{0}
\bibitem[\citeproctext]{ref-agius2017}
Agius, M., Rychert, C., Harmon, N., \& Laske, G. (2017). Mapping the
mantle transition zone beneath hawaii from ps receiver functions:
Evidence for a hot plume and cold mantle downwellings. \emph{Earth and
Planetary Science Letters}, \emph{474}, 226--236.

\bibitem[\citeproctext]{ref-agrusta2017}
Agrusta, R., Goes, S., \& Van Hunen, J. (2017). Subducting-slab
transition-zone interaction: Stagnation, penetration and mode switches.
\emph{Earth and Planetary Science Letters}, \emph{464}, 10--23.

\bibitem[\citeproctext]{ref-alex2004}
Alex Song, T., Helmberger, D., \& Grand, S. (2004). Low-velocity zone
atop the 410-km seismic discontinuity in the northwestern united states.
\emph{Nature}, \emph{427}(6974), 530--533.

\bibitem[\citeproctext]{ref-aspect-doi-v3.0.0}
Bangerth, W., Dannberg, J., Fraters, M., Gassmöller, R., Glerum, A.,
Heister, T., et al. (2024a, December). ASPECT v3.0.0 (Version v3.0.0).
Zenodo. \url{https://doi.org/10.5281/zenodo.14371679}

\bibitem[\citeproctext]{ref-aspectmanual}
Bangerth, W., Dannberg, J., Fraters, M., Gassmöller, R., Glerum, A.,
Heister, T., et al. (2024b, December). {{ASPECT}: Advanced Solver for
Planetary Evolution, Convection, and Tectonics, User Manual}.
\url{https://doi.org/10.6084/m9.figshare.4865333}

\bibitem[\citeproctext]{ref-cahn1956}
Cahn, J. (1956). The kinetics of grain boundary nucleated reactions.
\emph{Acta Metallurgica}, \emph{4}(5), 449--459.

\bibitem[\citeproctext]{ref-chambers2005}
Chambers, K., Deuss, A., \& Woodhouse, J. (2005). Reflectivity of the
410-km discontinuity from PP and SS precursors. \emph{Journal of
Geophysical Research: Solid Earth}, \emph{110}(B2).

\bibitem[\citeproctext]{ref-clevenger2021}
Clevenger, T., \& Heister, T. (2021). Comparison between algebraic and
matrix-free geometric multigrid for a stokes problem on adaptive meshes
with variable viscosity. \emph{Numerical Linear Algebra with
Applications}, \emph{28}(5), e2375.

\bibitem[\citeproctext]{ref-connolly2009}
Connolly, J. (2009). The geodynamic equation of state: What and how.
\emph{Geochemistry, Geophysics, Geosystems}, \emph{10}(10).

\bibitem[\citeproctext]{ref-cottaar2016}
Cottaar, S., \& Deuss, A. (2016). Large-scale mantle discontinuity
topography beneath europe: Signature of akimotoite in subducting slabs.
\emph{Journal of Geophysical Research: Solid Earth}, \emph{121}(1),
279--292.

\bibitem[\citeproctext]{ref-cottaar2014}
Cottaar, S., Heister, T., Rose, I., \& Unterborn, C. (2014). BurnMan: A
lower mantle mineral physics toolkit. \emph{Geochemistry, Geophysics,
Geosystems}, \emph{15}(4), 1164--1179.

\bibitem[\citeproctext]{ref-curbelo2019}
Curbelo, J., Duarte, L., Alboussiere, T., Dubuffet, F., Labrosse, S., \&
Ricard, Y. (2019). Numerical solutions of compressible convection with
an infinite prandtl number: Comparison of the anelastic and anelastic
liquid models with the exact equations. \emph{Journal of Fluid
Mechanics}, \emph{873}, 646--687.

\bibitem[\citeproctext]{ref-deuss2009}
Deuss, A. (2009). Global observations of mantle discontinuities using SS
and PP precursors. \emph{Surveys in Geophysics}, \emph{30}(4), 301--326.

\bibitem[\citeproctext]{ref-deuss2001}
Deuss, A., \& Woodhouse, J. (2001). Seismic observations of splitting of
the mid-transition zone discontinuity in earth's mantle. \emph{Science},
\emph{294}(5541), 354--357.

\bibitem[\citeproctext]{ref-dokht2016}
Dokht, R., Gu, Y., \& Sacchi, M. (2016). Waveform inversion of SS
precursors: An investigation of the northwestern pacific subduction
zones and intraplate volcanoes in china. \emph{Gondwana Research},
\emph{40}, 77--90.

\bibitem[\citeproctext]{ref-faccenda2017}
Faccenda, M., \& Dal Zilio, L. (2017). The role of solid--solid phase
transitions in mantle convection. \emph{Lithos}, \emph{268}, 198--224.

\bibitem[\citeproctext]{ref-fraters2020}
Fraters, M. (2020, June). The geodynamic world builder (Version v0.3.0).
Zenodo. \url{https://doi.org/10.5281/zenodo.3900603}

\bibitem[\citeproctext]{ref-fraters2019}
{Fraters, M., Thieulot, C., van den Berg, A., \& Spakman, W.} (2019).
The geodynamic world builder: A solution for complex initial conditions
in numerical modeling. \emph{Solid Earth}, \emph{10}(5), 1785--1807.

\bibitem[\citeproctext]{ref-frazer2023}
Frazer, W., \& Park, J. (2023). High-resolution mid-mantle imaging with
multiple-taper SS-precursor estimates. \emph{Geophysical Journal
International}, \emph{233}(2), 1356--1371.

\bibitem[\citeproctext]{ref-fukao2013}
Fukao, Y., \& Obayashi, M. (2013). Subducted slabs stagnant above,
penetrating through, and trapped below the 660 km discontinuity.
\emph{Journal of Geophysical Research: Solid Earth}, \emph{118}(11),
5920--5938.

\bibitem[\citeproctext]{ref-gassmoller2018}
Gassmöller, R., Lokavarapu, H., Heien, E., Puckett, E., \& Bangerth, W.
(2018). Flexible and scalable particle-in-cell methods with adaptive
mesh refinement for geodynamic computations. \emph{Geochemistry,
Geophysics, Geosystems}, \emph{19}(9), 3596--3604.

\bibitem[\citeproctext]{ref-gassmoller2020}
Gassmöller, R., Dannberg, J., Bangerth, W., Heister, T., \& Myhill, R.
(2020). On formulations of compressible mantle convection.
\emph{Geophysical Journal International}, \emph{221}(2), 1264--1280.

\bibitem[\citeproctext]{ref-gerya2019}
Gerya, T. (2019). \emph{Introduction to numerical geodynamic modelling}.
Cambridge University Press.

\bibitem[\citeproctext]{ref-glasgow2024}
Glasgow, M., Zhang, H., Schmandt, B., Zhou, W., \& Zhang, J. (2024).
Global variability of the composition and temperature at the 410-km
discontinuity from receiver function analysis of dense arrays.
\emph{Earth and Planetary Science Letters}, \emph{643}, 118889.

\bibitem[\citeproctext]{ref-green1979}
Green, D., Jaques, L., \& Hibberson, W. (1979). Petrogenesis of
mid-ocean ridge basalts. In \emph{The earth: Its origin, structure and
evolution} (pp. 265--300). Academic Press.

\bibitem[\citeproctext]{ref-green1995}
Green, H., \& Houston, H. (1995). The mechanics of deep earthquakes.
\emph{Annual Review Of Earth And Planetary Sciences, Volume 23, Pp.
169-214.}, \emph{23}, 169--214.

\bibitem[\citeproctext]{ref-han2021}
Han, G., Li, J., Guo, G., Mooney, W., Karato, S., \& Yuen, D. (2021).
Pervasive low-velocity layer atop the 410-km discontinuity beneath the
northwest pacific subduction zone: Implications for rheology and
geodynamics. \emph{Earth and Planetary Science Letters}, \emph{554},
116642.

\bibitem[\citeproctext]{ref-heister2017}
Heister, T., Dannberg, J., Gassmöller, R., \& Bangerth, W. (2017). High
accuracy mantle convection simulation through modern numerical
methods--II: Realistic models and problems. \emph{Geophysical Journal
International}, \emph{210}(2), 833--851.

\bibitem[\citeproctext]{ref-helffrich1996}
Helffrich, G., \& Wood, B. (1996). 410 km discontinuity sharpness and
the form of the olivine \(\alpha\)-\(\beta\) phase diagram: Resolution
of apparent seismic contradictions. \emph{Geophysical Journal
International}, \emph{126}(2), F7--F12.

\bibitem[\citeproctext]{ref-hosoya2005}
Hosoya, T., Kubo, T., Ohtani, E., Sano, A., \& Funakoshi, K. (2005).
Water controls the fields of metastable olivine in cold subducting
slabs. \emph{Geophysical Research Letters}, \emph{32}(17).

\bibitem[\citeproctext]{ref-houser2010}
Houser, C., \& Williams, Q. (2010). Reconciling pacific 410 and 660 km
discontinuity topography, transition zone shear velocity patterns, and
mantle phase transitions. \emph{Earth and Planetary Science Letters},
\emph{296}(3-4), 255--266.

\bibitem[\citeproctext]{ref-ishii2021}
Ishii, T., \& Ohtani, E. (2021). Dry metastable olivine and slab
deformation in a wet subducting slab. \emph{Nature Geoscience},
\emph{14}(7), 526--530.

\bibitem[\citeproctext]{ref-jarvis1980}
Jarvis, G., \& Mckenzie, D. (1980). Convection in a compressible fluid
with infinite prandtl number. \emph{Journal of Fluid Mechanics},
\emph{96}(3), 515--583.

\bibitem[\citeproctext]{ref-jenkins2016}
Jenkins, J., Cottaar, S., White, R., \& Deuss, A. (2016). Depressed
mantle discontinuities beneath iceland: Evidence of a garnet controlled
660 km discontinuity? \emph{Earth and Planetary Science Letters},
\emph{433}, 159--168.

\bibitem[\citeproctext]{ref-jiang2015}
Jiang, G., Zhao, D., \& Zhang, G. (2015). Detection of metastable
olivine wedge in the western pacific slab and its geodynamic
implications. \emph{Physics of the Earth and Planetary Interiors},
\emph{238}, 1--7.

\bibitem[\citeproctext]{ref-karato2008}
Karato, S. (2008). Deformation of earth materials. \emph{An Introduction
to the Rheology of Solid Earth}, \emph{463}.

\bibitem[\citeproctext]{ref-karato2001}
Karato, S., Riedel, M., \& Yuen, D. (2001). Rheological structure and
deformation of subducted slabs in the mantle transition zone:
Implications for mantle circulation and deep earthquakes. \emph{Physics
of the Earth and Planetary Interiors}, \emph{127}(1-4), 83--108.

\bibitem[\citeproctext]{ref-karato1984}
Karato, S.-I. (1984). Grain-size distribution and rheology of the upper
mantle. \emph{Tectonophysics}, \emph{104}(1-2), 155--176.

\bibitem[\citeproctext]{ref-katsura2004}
{Katsura, T., Yamada, H., Nishikawa, O., Song, M., Kubo, A., Shinmei,
T., et al.} (2004). Olivine-wadsleyite transition in the system (mg, fe)
2SiO4. \emph{Journal of Geophysical Research: Solid Earth},
\emph{109}(B2).

\bibitem[\citeproctext]{ref-kirby1996}
Kirby, S., Stein, S., Okal, E., \& Rubie, D. (1996). Metastable mantle
phase transformations and deep earthquakes in subducting oceanic
lithosphere. \emph{Reviews of Geophysics}, \emph{34}(2), 261--306.

\bibitem[\citeproctext]{ref-kronbichler2012}
Kronbichler, M., Heister, T., \& Bangerth, W. (2012). High accuracy
mantle convection simulation through modern numerical methods.
\emph{Geophysical Journal International}, \emph{191}(1), 12--29.

\bibitem[\citeproctext]{ref-kubo2004}
Kubo, T., Ohtani, E., \& Funakoshi, K. (2004). Nucleation and growth
kinetics of the \(\alpha\)-\(\beta\) transformation in Mg2SiO4determined
by in situ synchrotron powder x-ray diffraction. \emph{American
Mineralogist}, \emph{89}(2-3), 285--293.

\bibitem[\citeproctext]{ref-lawrence2008}
Lawrence, J., \& Shearer, P. (2008). Imaging mantle transition zone
thickness with SdS-SS finite-frequency sensitivity kernels.
\emph{Geophysical Journal International}, \emph{174}(1), 143--158.

\bibitem[\citeproctext]{ref-ledoux2023}
{Ledoux, E., Krug, M., Gay, J., Chantel, J., Hilairet, N., Bykov, M., et
al.} (2023). In-situ study of microstructures induced by the olivine to
wadsleyite transformation at conditions of the 410 km depth
discontinuity. \emph{American Mineralogist}, \emph{108}(12), 2283--2293.

\bibitem[\citeproctext]{ref-lee2014}
Lee, S., Rhie, J., Park, Y., \& Kim, K. (2014). Topography of the 410
and 660 km discontinuities beneath the korean peninsula and southwestern
japan using teleseismic receiver functions. \emph{Journal of Geophysical
Research: Solid Earth}, \emph{119}(9), 7245--7257.

\bibitem[\citeproctext]{ref-myhill2023}
Myhill, R., Cottaar, S., Heister, T., Rose, I., Unterborn, C., Dannberg,
J., \& Gassmöller, R. (2023). BurnMan--a python toolkit for planetary
geophysics, geochemistry and thermodynamics.

\bibitem[\citeproctext]{ref-ohuchi2022}
Ohuchi, T., Higo, Y., Tange, Y., Sakai, T., Matsuda, K., \& Irifune, T.
(2022). In situ x-ray and acoustic observations of deep seismic faulting
upon phase transitions in olivine. \emph{Nature Communications},
\emph{13}(1), 5213.

\bibitem[\citeproctext]{ref-perrillat2013}
Perrillat, J., Daniel, I., Bolfan-Casanova, N., Chollet, M., Morard, G.,
\& Mezouar, M. (2013). Mechanism and kinetics of the
\(\alpha\)--\(\beta\) transition in san carlos olivine Mg1. 8Fe0. 2SiO4.
\emph{Journal of Geophysical Research: Solid Earth}, \emph{118}(1),
110--119.

\bibitem[\citeproctext]{ref-perrillat2016}
Perrillat, J., Chollet, M., Durand, S., De Moortele, B. van, Chambat,
F., Mezouar, M., \& Daniel, I. (2016). Kinetics of the
olivine--ringwoodite transformation and seismic attenuation in the
earth's mantle transition zone. \emph{Earth and Planetary Science
Letters}, \emph{433}, 360--369.

\bibitem[\citeproctext]{ref-ranalli1995}
Ranalli, G. (1995). \emph{Rheology of the earth}. Springer Science \&
Business Media.

\bibitem[\citeproctext]{ref-ringwood1975}
Ringwood, A. (1975). Composition and petrology of the earth's mantle.
\emph{MacGraw-Hill}, \emph{618}.

\bibitem[\citeproctext]{ref-rubie1994}
Rubie, D., \& Ross II, C. (1994). Kinetics of the olivine-spinel
transformation in subducting lithosphere: Experimental constraints and
implications for deep slab processes. \emph{Physics of the Earth and
Planetary Interiors}, \emph{86}(1-3), 223--243.

\bibitem[\citeproctext]{ref-saikia2008}
Saikia, A., Frost, D., \& Rubie, D. (2008). Splitting of the
520-kilometer seismic discontinuity and chemical heterogeneity in the
mantle. \emph{Science}, \emph{319}(5869), 1515--1518.

\bibitem[\citeproctext]{ref-schmandt2012}
Schmandt, B. (2012). Mantle transition zone shear velocity gradients
beneath USArray. \emph{Earth and Planetary Science Letters}, \emph{355},
119--130.

\bibitem[\citeproctext]{ref-schmerr2007}
Schmerr, N., \& Garnero, E. (2007). Upper mantle discontinuity
topography from thermal and chemical heterogeneity. \emph{Science},
\emph{318}(5850), 623--626.

\bibitem[\citeproctext]{ref-schubert2001}
Schubert, G., Turcotte, D., \& Olson, P. (2001). \emph{Mantle convection
in the earth and planets}. Cambridge University Press.

\bibitem[\citeproctext]{ref-shearer2000}
Shearer, P. (2000). Upper mantle seismic discontinuities.
\emph{Geophysical Monograph-American Geophysical Union}, \emph{117},
115--132.

\bibitem[\citeproctext]{ref-shen2020}
Shen, Z., \& Zhan, Z. (2020). Metastable olivine wedge beneath the japan
sea imaged by seismic interferometry. \emph{Geophysical Research
Letters}, \emph{47}(6), e2019GL085665.

\bibitem[\citeproctext]{ref-sindhusuta2025}
Sindhusuta, S., Chi, S., Foster, C., Officer, T., \& Wang, Y. (2025).
Numerical investigation into mechanical behavior of metastable olivine
during phase transformation: Implications for deep-focus earthquakes.
\emph{Journal of Geophysical Research: Solid Earth}, \emph{130}(2),
e2024JB030557.

\bibitem[\citeproctext]{ref-stixrude2022}
Stixrude, L., \& Lithgow-Bertelloni, C. (2022). Thermal expansivity,
heat capacity and bulk modulus of the mantle. \emph{Geophysical Journal
International}, \emph{228}(2), 1119--1149.

\bibitem[\citeproctext]{ref-tauzin2017}
Tauzin, B., Kim, S., \& Kennett, B. (2017). Pervasive seismic
low-velocity zones within stagnant plates in the mantle transition zone:
Thermal or compositional origin? \emph{Earth and Planetary Science
Letters}, \emph{477}, 1--13.

\bibitem[\citeproctext]{ref-vanstiphout2019}
Van Stiphout, A., Cottaar, S., \& Deuss, A. (2019). Receiver function
mapping of mantle transition zone discontinuities beneath alaska using
scaled 3-d velocity corrections. \emph{Geophysical Journal
International}, \emph{219}(2), 1432--1446.

\bibitem[\citeproctext]{ref-wei2017}
Wei, S., \& Shearer, P. (2017). A sporadic low-velocity layer atop the
410 km discontinuity beneath the pacific ocean. \emph{Journal of
Geophysical Research: Solid Earth}, \emph{122}(7), 5144--5159.

\bibitem[\citeproctext]{ref-wheeler2014}
Wheeler, J. (2014). Dramatic effects of stress on metamorphic reactions.
\emph{Geology}, \emph{42}(8), 647--650.

\bibitem[\citeproctext]{ref-wheeler2018}
Wheeler, J. (2018). The effects of stress on reactions in the earth:
Sometimes rather mean, usually normal, always important. \emph{Journal
of Metamorphic Geology}, \emph{36}(4), 439--461.

\bibitem[\citeproctext]{ref-wheeler2020}
Wheeler, J. (2020). A unifying basis for the interplay of stress and
chemical processes in the earth: Support from diverse experiments.
\emph{Contributions to Mineralogy and Petrology}, \emph{175}(12), 116.

\end{CSLReferences}

\cleardoublepage

\section*{Appendix}\label{appendix}
\addcontentsline{toc}{section}{Appendix}

\subsection*{Stress, Pressure, and The Momentum
Equation}\label{sec:momentum-derivation}
\addcontentsline{toc}{subsection}{Stress, Pressure, and The Momentum
Equation}

The momentum equation in the following form is referred to as the
Navier-Stokes equation and describes the flow of a compressible viscous
fluid primarily by buoyancy forces:

\begin{equation}
  \rho \left(\frac{\partial \vec{u}}{\partial t} \right) = \nabla \cdot \sigma^{\prime} - \nabla P + \rho g
  \label{eq:navier-stokes-compressible}
\end{equation}

where \(\rho\) is density, \(\vec{u}\) is velocity, \(P\) is pressure,
\(\sigma^{\prime}\) is the deviatoric stress tensor, and \(g\) is
gravitational acceleration. In terms of classical mechanics, the
left-hand side is analogous to mass times acceleration \(ma\), and the
right-hand side are the forces \(F\) that are acting on the fluid.
Hence, the equation describes a balance between force and momentum
\(ma = F\) (\citeproc{ref-gerya2019}{Gerya, 2019}).

The forces in Equation \ref{eq:navier-stokes-compressible} include the
pressure gradient \(\nabla P\) which acts to drive the fluid from high
pressure to low pressure, the viscous forces
\(\nabla \cdot \sigma^{\prime}\) that dissipate energy by resisting
flow, and the buoyancy forces \(\rho g\) that drive convection (denser
fluids sink while lighter fluids rise). Because the flow of Earth's
mantle occurs at such slow rates, however, the inertial term
\(\left(\frac{\partial \vec{u}}{\partial t} \right)\) on the left-hand
side of Equation \ref{eq:navier-stokes-compressible} can be ignored, and
the momentum equation simplifies to:

\begin{equation}
  \nabla P - \nabla \cdot \sigma^{\prime} = \rho g
  \label{eq:navier-stokes-no-inertia-appendix}
\end{equation}

Equation \ref{eq:navier-stokes-no-inertia-appendix} describes a balance
between the buoyancy force and the pressure gradient minus the energy
dissipation due to deformation. Since the deviatoric stress tensor
\(\sigma^{\prime}\) can be described in terms of velocity (see Equation
\ref{eq:stress-deviatoric-component}), the primary unknowns in Equation
\ref{eq:navier-stokes-compressible} are pressure and velocity.

The complete stress tensor can be written as:

\begin{equation}
  \sigma_{ij} = \sigma^{\prime}_{ij} - P \delta_{ij}
  \label{eq:stress-complete}
\end{equation}

where \(\sigma_{ij}\) is total stress, \(\sigma^{\prime}_{ij}\) is the
deviatoric (non-hydrostatic) component of stress,
\(P = - \frac{\sigma_{xx} + \sigma_{yy} + \sigma_{zz}}{3}\) is the
hydrostatic component of stress, and \(\delta_{ij}\) is the Kronecker
delta:

\begin{equation}
  \delta_{ij} =
  \begin{cases}
    1, & \text{if} i = j \\
    0, & \text{if} i \neq j
  \end{cases}
\end{equation}

The hydrostatic component of stress acts equally in all directions and
therefore affects the fluid's volume (density) but does not change its
shape or cause it to flow. Note that the negative sign in Equation
\ref{eq:stress-complete} implies that pressure is positive under
compression (negative normal stress). This is a convention used in
geodynamics that differs from material sciences and other fields.

The deviatoric part of the stress tensor is responsible for deformation
and flow of the fluid and is equal to the total stress without the
hydrostatic stress component,
\(\sigma^{\prime}_{ij} = \sigma_{ij} + P \delta_{ij}\), or in full
matrix form:

\begin{equation}
  \begin{pmatrix}
  \sigma^{\prime}_{xx} & \sigma^{\prime}_{xy} & \sigma^{\prime}_{xz} \\
  \sigma^{\prime}_{yx} & \sigma^{\prime}_{yy} & \sigma^{\prime}_{yz} \\
  \sigma^{\prime}_{zx} & \sigma^{\prime}_{zy} & \sigma^{\prime}_{zz}
  \end{pmatrix} =
  \begin{pmatrix}
  \sigma_{xx} & \sigma_{xy} & \sigma_{xz} \\
  \sigma_{yx} & \sigma_{yy} & \sigma_{yz} \\
  \sigma_{zx} & \sigma_{zy} & \sigma_{zz}
  \end{pmatrix} +
  \begin{pmatrix}
  -\frac{\sigma_{xx} + \sigma_{yy} + \sigma_{zz}}{3} & 0 & 0 \\
  0 & -\frac{\sigma_{xx} + \sigma_{yy} + \sigma_{zz}}{3} & 0 \\
  0 & 0 & -\frac{\sigma_{xx} + \sigma_{yy} + \sigma_{zz}}{3}
  \end{pmatrix}
  \label{eq:stress-deviatoric}
\end{equation}

In practice, the deviatoric stress tensor \(\sigma^{\prime}\) is
computed by applying a constitutive relationship between stress and
strain to express \(\sigma^{\prime}\) in terms of velocity. In the
present case, we apply a generalized linear model that combines shear
deformation without rotation and volumetric deformation (dilation):

\begin{equation}
  \sigma^{\prime} = \eta \left(\nabla \vec{u} + \left(\nabla \vec{u} \right)^\intercal \right) - \left(\frac{2}{3} \eta - \zeta \right) \left(\nabla \cdot \vec{u} \right) I
\end{equation}

where \(\vec{u}\) is velocity, \(\eta\) is shear viscosity, and
\(\zeta\) is bulk viscosity. For nearly-incompressible fluids (very
small \(\zeta\)), the expression for deviatoric stress simplifies to:

\begin{equation}
  \sigma^{\prime} = 2 \eta \dot{\epsilon}^{\prime}
\end{equation}

where
\(\dot{\epsilon}^{\prime} = \frac{1}{2} \left(\nabla \vec{u} + \left(\nabla \vec{u} \right)^\intercal \right) - \frac{1}{3} \left(\nabla \cdot \vec{u} \right) I\)
is the deviatoric strain rate tensor. In full component form the
deviatoric stress tensor is:

\begin{equation}
  \sigma^{\prime}_{ij} = \eta \left(\frac{\partial u_i}{\partial x_j} + \frac{\partial u_j}{\partial x_i} \right) - \frac{2}{3} \eta \left(\frac{\partial u_x}{\partial x} + \frac{\partial u_y}{\partial y} + \frac{\partial u_z}{\partial z} \right) \delta_{ij}
  \label{eq:stress-deviatoric-component}
\end{equation}

Note that the deviatoric stress and strain rate tensors are symmetric
such that \(\sigma^{\prime}_{ij} = \sigma^{\prime}_{ji}\) and
\(\dot{\epsilon}^{\prime}_{ij} = \dot{\epsilon}^{\prime}_{ji}\), which
implies that there is zero rigid-body rotation in the fluid flow.
Because of this symmetry, the full matrix form the deviatoric stress
tensor can be written as:

\begin{equation}
  \sigma^{\prime} = \begin{pmatrix}
  2 \eta \frac{\partial u_x}{\partial x} - \frac{2}{3} \eta \left(\frac{\partial u_x}{\partial x} + \frac{\partial u_y}{\partial y} + \frac{\partial u_z}{\partial z} \right) & \eta \left(\frac{\partial u_x}{\partial y} + \frac{\partial u_y}{\partial x} \right) & \eta \left(\frac{\partial u_x}{\partial z} + \frac{\partial u_z}{\partial x} \right) \\
  \eta \left(\frac{\partial u_y}{\partial x} + \frac{\partial u_x}{\partial y} \right) & 2 \eta \frac{\partial u_y}{\partial y} - \frac{2}{3} \eta \left(\frac{\partial u_x}{\partial x} + \frac{\partial u_y}{\partial y} + \frac{\partial u_z}{\partial z} \right) & \eta \left(\frac{\partial u_y}{\partial z} + \frac{\partial u_z}{\partial y} \right) \\
  \eta \left(\frac{\partial u_z}{\partial x} + \frac{\partial u_x}{\partial z} \right) & \eta \left(\frac{\partial u_z}{\partial y} + \frac{\partial u_y}{\partial z} \right) & 2 \eta \frac{\partial u_z}{\partial z} - \frac{2}{3} \eta \left(\frac{\partial u_x}{\partial x} + \frac{\partial u_y}{\partial y} + \frac{\partial u_z}{\partial z} \right)
  \end{pmatrix}
\end{equation}

It is often useful to visualize the \emph{second invariant} of the
deviatoric stress tensor, which is independent of the coordinate
reference frame and quantifies the local deviation of stress from a
hydrostatic (non-convecting) state:

\begin{equation}
  \sigma_{\text{II}} = \sqrt{\frac{1}{2} \left(\text{tr}(\sigma^{\prime 2}) - \text{tr}(\sigma^{\prime})^2 \right)}
  \label{eq:second-invariant-definition}
\end{equation}

where \(\text{tr}(\sigma^{\prime 2}) = \Sigma \sigma^{\prime 2}_{ij}\)
and
\(\text{tr}(\sigma^{\prime})^2 = (\sigma^{\prime}_{xx} + \sigma^{\prime}_{yy} + \sigma^{\prime}_{zz})^2\).
Note that Equation \ref{eq:second-invariant-definition} uses the
convention that compressive stress is positive. It follows from Equation
\ref{eq:stress-complete} that the normal deviatoric stresses are:

\begin{equation}
  \begin{aligned}
    \sigma^{\prime}_{xx} &= \sigma_{xx} + P \\
    \sigma^{\prime}_{yy} &= \sigma_{yy} + P \\
    \sigma^{\prime}_{zz} &= \sigma_{zz} + P
  \end{aligned}
\end{equation}

and thus by definition
\(\text{tr}(\sigma^{\prime}) = \text{tr}(\sigma) + 3P = 0\), since
\(\text{tr}(\sigma) = -3P\). By this definition, Equation
\ref{eq:second-invariant-definition} can be written as:

\begin{equation}
  \begin{aligned}
    \sigma_{\text{II}} &= \sqrt{\frac{1}{2} \left(\text{tr}(\sigma^{\prime 2}) - 0 \right)} = \sqrt{\frac{1}{2} \text{tr}(\sigma^{\prime 2})} = \sqrt{\frac{1}{2} \sum_{i, j}\sigma^{\prime 2}_{ij}} \\
    \sigma_{\text{II}} &= \sqrt{\frac{1}{2} (\sigma^{\prime 2}_{xx} + \sigma^{\prime 2}_{yy} + \sigma^{\prime 2}_{zz} + \sigma^{\prime 2}_{xy} + \sigma^{\prime 2}_{yx} + \sigma^{\prime 2}_{xz} + \sigma^{\prime 2}_{zx} + \sigma^{\prime 2}_{yz} + \sigma^{\prime 2}_{zy})} \\
    \sigma_{\text{II}} &= \sqrt{\frac{1}{2} (\sigma^{\prime 2}_{xx} + \sigma^{\prime 2}_{yy} + \sigma^{\prime 2}_{zz}) + \sigma^{\prime 2}_{xy} + \sigma^{\prime 2}_{xz} + \sigma^{\prime 2}_{yz}}
  \end{aligned}
\end{equation}

Note also that many engineering applications use the von Mises stress:

\begin{equation}
  \sigma_{\text{vm}} = \sqrt{\frac{3}{2} \sum_{i, j}\sigma^{\prime 2}_{ij}}
\end{equation}

which is proportional to the second invariant of the deviatoric stress
tensor by a factor of \(\sqrt{3}\):

\begin{equation}
  \sigma_{\text{vm}} = \sqrt{3} \, \sigma_{\text{II}}
\end{equation}

\cleardoublepage

\subsection*{Simulation Snapshots: Slabs and Plumes
Continued}\label{sec:simulation-snapshots-continued}
\addcontentsline{toc}{subsection}{Simulation Snapshots: Slabs and Plumes
Continued}

\begin{figure}
\centering
\pandocbounded{\includegraphics[keepaspectratio,alt={Slab simulation with ultra-sluggish kinetics (Z = 3.0e0 K s\^{}\{-1\}) after 100 Ma evolution. Panels show dynamic temperature \textbackslash hat\{T\} (a), dynamic pressure \textbackslash hat\{P\} (b), dynamic density \textbackslash hat\{\textbackslash rho\} (c), Arrhenius term \textbackslash exp\textbackslash left(-H\^{}\{\textbackslash ast\} + PV\^{}\{\textbackslash ast\}/RT\textbackslash right) (d), thermodynamic term \textbackslash left(1 - \textbackslash left{[}\textbackslash Delta G/R\textbackslash,T\textbackslash right{]}\textbackslash right) (e), phase transition rate \textbackslash dot\{X\} (f), volume fraction of wadsleyite X (g), pressure-wave velocity V\_p (h), and shear-wave velocity V\_s (i).}]{../figs/simulation/compositions/slab-3.0e0-full-set-composition-0100.png}}
\caption{Slab simulation with ultra-sluggish kinetics (\(Z\) = 3.0e0 K
s\(^{-1}\)) after 100 Ma evolution. Panels show dynamic temperature
\(\hat{T}\) (a), dynamic pressure \(\hat{P}\) (b), dynamic density
\(\hat{\rho}\) (c), Arrhenius term
\(\exp\left(-H^{\ast} + PV^{\ast}/RT\right)\) (d), thermodynamic term
\(\left(1 - \left[\Delta G/R\,T\right]\right)\) (e), phase transition
rate \(\dot{X}\) (f), volume fraction of wadsleyite \(X\) (g),
pressure-wave velocity \(V_p\) (h), and shear-wave velocity \(V_s\)
(i).}\label{fig:slab-composition-slow}
\end{figure}

\begin{figure}
\centering
\pandocbounded{\includegraphics[keepaspectratio,alt={Slab simulation with moderately-sluggish kinetics (Z = 4.7e2 K s\^{}\{-1\}) after 100 Ma evolution. Panels show dynamic temperature \textbackslash hat\{T\} (a), dynamic pressure \textbackslash hat\{P\} (b), dynamic density \textbackslash hat\{\textbackslash rho\} (c), Arrhenius term \textbackslash exp\textbackslash left(-H\^{}\{\textbackslash ast\} + PV\^{}\{\textbackslash ast\}/RT\textbackslash right) (d), thermodynamic term \textbackslash left(1 - \textbackslash left{[}\textbackslash Delta G/R\textbackslash,T\textbackslash right{]}\textbackslash right) (e), phase transition rate \textbackslash dot\{X\} (f), volume fraction of wadsleyite X (g), pressure-wave velocity V\_p (h), and shear-wave velocity V\_s (i).}]{../figs/simulation/compositions/slab-4.7e2-full-set-composition-0100.png}}
\caption{Slab simulation with moderately-sluggish kinetics (\(Z\) =
4.7e2 K s\(^{-1}\)) after 100 Ma evolution. Panels show dynamic
temperature \(\hat{T}\) (a), dynamic pressure \(\hat{P}\) (b), dynamic
density \(\hat{\rho}\) (c), Arrhenius term
\(\exp\left(-H^{\ast} + PV^{\ast}/RT\right)\) (d), thermodynamic term
\(\left(1 - \left[\Delta G/R\,T\right]\right)\) (e), phase transition
rate \(\dot{X}\) (f), volume fraction of wadsleyite \(X\) (g),
pressure-wave velocity \(V_p\) (h), and shear-wave velocity \(V_s\)
(i).}\label{fig:slab-composition-moderately-slow}
\end{figure}

\begin{figure}
\centering
\pandocbounded{\includegraphics[keepaspectratio,alt={Slab simulation with quasi-equilibrium kinetics (Z = 7.0e7 K s\^{}\{-1\}) after 100 Ma evolution. Panels show dynamic temperature \textbackslash hat\{T\} (a), dynamic pressure \textbackslash hat\{P\} (b), dynamic density \textbackslash hat\{\textbackslash rho\} (c), Arrhenius term \textbackslash exp\textbackslash left(-H\^{}\{\textbackslash ast\} + PV\^{}\{\textbackslash ast\}/RT\textbackslash right) (d) thermodynamic term \textbackslash left(1 - \textbackslash left{[}\textbackslash Delta G/R\textbackslash,T\textbackslash right{]}\textbackslash right) (e), phase transition rate \textbackslash dot\{X\} (f), volume fraction of wadsleyite X (g), pressure-wave velocity V\_p (h), and shear-wave velocity V\_s (i).}]{../figs/simulation/compositions/slab-7.0e7-full-set-composition-0100.png}}
\caption{Slab simulation with quasi-equilibrium kinetics (\(Z\) = 7.0e7
K s\(^{-1}\)) after 100 Ma evolution. Panels show dynamic temperature
\(\hat{T}\) (a), dynamic pressure \(\hat{P}\) (b), dynamic density
\(\hat{\rho}\) (c), Arrhenius term
\(\exp\left(-H^{\ast} + PV^{\ast}/RT\right)\) (d) thermodynamic term
\(\left(1 - \left[\Delta G/R\,T\right]\right)\) (e), phase transition
rate \(\dot{X}\) (f), volume fraction of wadsleyite \(X\) (g),
pressure-wave velocity \(V_p\) (h), and shear-wave velocity \(V_s\)
(i).}\label{fig:slab-composition-fast}
\end{figure}

\begin{figure}
\centering
\pandocbounded{\includegraphics[keepaspectratio,alt={Plume simulation with ultra-sluggish kinetics (Z = 3.0e0 K s\^{}\{-1\}) after 100 Ma evolution. Panels show dynamic temperature \textbackslash hat\{T\} (a), dynamic pressure \textbackslash hat\{P\} (b), dynamic density \textbackslash hat\{\textbackslash rho\} (c), Arrhenius term \textbackslash exp\textbackslash left(-H\^{}\{\textbackslash ast\} + PV\^{}\{\textbackslash ast\}/RT\textbackslash right) (d) thermodynamic term \textbackslash left(1 - \textbackslash left{[}\textbackslash Delta G/R\textbackslash,T\textbackslash right{]}\textbackslash right) (e), phase transition rate \textbackslash dot\{X\} (f), volume fraction of olivine X (g), pressure-wave velocity V\_p (h), and shear-wave velocity V\_s (i).}]{../figs/simulation/compositions/plume-3.0e0-full-set-composition-0100.png}}
\caption{Plume simulation with ultra-sluggish kinetics (\(Z\) = 3.0e0 K
s\(^{-1}\)) after 100 Ma evolution. Panels show dynamic temperature
\(\hat{T}\) (a), dynamic pressure \(\hat{P}\) (b), dynamic density
\(\hat{\rho}\) (c), Arrhenius term
\(\exp\left(-H^{\ast} + PV^{\ast}/RT\right)\) (d) thermodynamic term
\(\left(1 - \left[\Delta G/R\,T\right]\right)\) (e), phase transition
rate \(\dot{X}\) (f), volume fraction of olivine \(X\) (g),
pressure-wave velocity \(V_p\) (h), and shear-wave velocity \(V_s\)
(i).}\label{fig:plume-composition-slow}
\end{figure}

\begin{figure}
\centering
\pandocbounded{\includegraphics[keepaspectratio,alt={Plume simulation with moderately-sluggish kinetics (Z = 4.7e2 K s\^{}\{-1\}) after 100 Ma evolution. Panels show dynamic temperature \textbackslash hat\{T\} (a), dynamic pressure \textbackslash hat\{P\} (b), dynamic density \textbackslash hat\{\textbackslash rho\} (c), Arrhenius term \textbackslash exp\textbackslash left(-H\^{}\{\textbackslash ast\} + PV\^{}\{\textbackslash ast\}/RT\textbackslash right) (d) thermodynamic term \textbackslash left(1 - \textbackslash left{[}\textbackslash Delta G/R\textbackslash,T\textbackslash right{]}\textbackslash right) (e), phase transition rate \textbackslash dot\{X\} (f), volume fraction of olivine X (g), pressure-wave velocity V\_p (h), and shear-wave velocity V\_s (i).}]{../figs/simulation/compositions/plume-4.7e2-full-set-composition-0100.png}}
\caption{Plume simulation with moderately-sluggish kinetics (\(Z\) =
4.7e2 K s\(^{-1}\)) after 100 Ma evolution. Panels show dynamic
temperature \(\hat{T}\) (a), dynamic pressure \(\hat{P}\) (b), dynamic
density \(\hat{\rho}\) (c), Arrhenius term
\(\exp\left(-H^{\ast} + PV^{\ast}/RT\right)\) (d) thermodynamic term
\(\left(1 - \left[\Delta G/R\,T\right]\right)\) (e), phase transition
rate \(\dot{X}\) (f), volume fraction of olivine \(X\) (g),
pressure-wave velocity \(V_p\) (h), and shear-wave velocity \(V_s\)
(i).}\label{fig:plume-composition-moderately-slow}
\end{figure}

\begin{figure}
\centering
\pandocbounded{\includegraphics[keepaspectratio,alt={Plume simulation with quasi-equilibrium kinetics (Z = 7.0e7 K s\^{}\{-1\}) after 100 Ma evolution. Panels show dynamic temperature \textbackslash hat\{T\} (a), dynamic pressure \textbackslash hat\{P\} (b), dynamic density \textbackslash hat\{\textbackslash rho\} (c), Arrhenius term \textbackslash exp\textbackslash left(-H\^{}\{\textbackslash ast\} + PV\^{}\{\textbackslash ast\}/RT\textbackslash right) (d) thermodynamic term \textbackslash left(1 - \textbackslash left{[}\textbackslash Delta G/R\textbackslash,T\textbackslash right{]}\textbackslash right) (e), phase transition rate \textbackslash dot\{X\} (f), volume fraction of olivine X (g), pressure-wave velocity V\_p (h), and shear-wave velocity V\_s (i).}]{../figs/simulation/compositions/plume-7.0e7-full-set-composition-0100.png}}
\caption{Plume simulation with quasi-equilibrium kinetics (\(Z\) = 7.0e7
K s\(^{-1}\)) after 100 Ma evolution. Panels show dynamic temperature
\(\hat{T}\) (a), dynamic pressure \(\hat{P}\) (b), dynamic density
\(\hat{\rho}\) (c), Arrhenius term
\(\exp\left(-H^{\ast} + PV^{\ast}/RT\right)\) (d) thermodynamic term
\(\left(1 - \left[\Delta G/R\,T\right]\right)\) (e), phase transition
rate \(\dot{X}\) (f), volume fraction of olivine \(X\) (g),
pressure-wave velocity \(V_p\) (h), and shear-wave velocity \(V_s\)
(i).}\label{fig:plume-composition-fast}
\end{figure}

\cleardoublepage

\subsection*{Structure of the 410: Displacement and Width
Continued}\label{sec:410-displacement-width-continued}
\addcontentsline{toc}{subsection}{Structure of the 410: Displacement and
Width Continued}

The 410 width and displacement were evaluated from the phase fraction
field \(X\) along a vertical profile that intersected the widest 410
found within the central third of model domain (\(x\) = 450 \(\pm\) 140
km), where 410 width was defined as the difference between the depths at
\(X\) = 0.9 and \(X\) = 0.1, and 410 displacement was defined as the
offset between the nominal equilibrium reaction depth and the depth at
\(X\) = 0.9. The maximum phase transition rate,
\(\dot{X}_{\mathrm{max}}\), was evaluated from the phase transition rate
field \(\dot{X}\) along the same vertical profile.

\begin{longtable}[]{@{}
  >{\raggedright\arraybackslash}p{(\linewidth - 10\tabcolsep) * \real{0.1458}}
  >{\raggedleft\arraybackslash}p{(\linewidth - 10\tabcolsep) * \real{0.0729}}
  >{\raggedleft\arraybackslash}p{(\linewidth - 10\tabcolsep) * \real{0.1458}}
  >{\raggedleft\arraybackslash}p{(\linewidth - 10\tabcolsep) * \real{0.0729}}
  >{\raggedleft\arraybackslash}p{(\linewidth - 10\tabcolsep) * \real{0.1771}}
  >{\raggedleft\arraybackslash}p{(\linewidth - 10\tabcolsep) * \real{0.3854}}@{}}
\caption{Summary of the kinetic factor \(Z\), 410 displacement, 410
width, maximum vertical velocity \(\vec{u}_y\), and maximum phase
transition rate \(\dot{X}\) evaluated in plume and slab simulations
after 100 Ma of evolution. Units are \(Z\): K s\(^{-1}\), 410
displacement: km, 410 width: km, \(\vec{u}_y\): cm/yr, Log\(_{10}\)
\(\dot{X}_{\mathrm{max}}\): Log\(_{10}\)
Ma\(^{-1}\).}\label{tbl:depth-profile-summary}\tabularnewline
\toprule\noalign{}
\begin{minipage}[b]{\linewidth}\raggedright
Simulation
\end{minipage} & \begin{minipage}[b]{\linewidth}\raggedleft
\(Z\)
\end{minipage} & \begin{minipage}[b]{\linewidth}\raggedleft
Displacement
\end{minipage} & \begin{minipage}[b]{\linewidth}\raggedleft
Width
\end{minipage} & \begin{minipage}[b]{\linewidth}\raggedleft
Max \(\vec{u}_y\)
\end{minipage} & \begin{minipage}[b]{\linewidth}\raggedleft
Log\(_{10}\) \(\dot{X}_{\mathrm{max}}\)
\end{minipage} \\
\midrule\noalign{}
\endfirsthead
\toprule\noalign{}
\begin{minipage}[b]{\linewidth}\raggedright
Simulation
\end{minipage} & \begin{minipage}[b]{\linewidth}\raggedleft
\(Z\)
\end{minipage} & \begin{minipage}[b]{\linewidth}\raggedleft
Displacement
\end{minipage} & \begin{minipage}[b]{\linewidth}\raggedleft
Width
\end{minipage} & \begin{minipage}[b]{\linewidth}\raggedleft
Max \(\vec{u}_y\)
\end{minipage} & \begin{minipage}[b]{\linewidth}\raggedleft
Log\(_{10}\) \(\dot{X}_{\mathrm{max}}\)
\end{minipage} \\
\midrule\noalign{}
\endhead
\bottomrule\noalign{}
\endlastfoot
plume & 3.0e0 & -16 & 19 & 1.92 & -2.87 \\
plume & 7.0e0 & -22 & 14 & 4.69 & -2.22 \\
plume & 1.6e1 & -23 & 10 & 4.69 & -1.65 \\
plume & 3.7e1 & -24 & 8 & 4.69 & -0.98 \\
plume & 8.7e1 & -25 & 5 & 4.69 & -0.31 \\
plume & 2.0e2 & -25 & 4 & 4.64 & 0.26 \\
plume & 4.7e2 & -25 & 4 & 4.69 & 0.81 \\
plume & 1.1e3 & -26 & 4 & 4.49 & 1.39 \\
plume & 2.6e3 & -26 & 3 & 4.64 & 2.49 \\
plume & 6.0e3 & -26 & 3 & 4.69 & 3.37 \\
plume & 1.4e4 & -26 & 2 & 4.69 & 3.84 \\
plume & 3.3e4 & -26 & 3 & 4.49 & 3.46 \\
plume & 7.8e4 & -26 & 2 & 4.49 & 4.59 \\
plume & 1.8e5 & -26 & 2 & 4.69 & 4.95 \\
plume & 4.3e5 & -26 & 3 & 3.19 & 4.84 \\
plume & 1.0e6 & -26 & 2 & 4.69 & 5.60 \\
plume & 2.4e6 & -26 & 2 & 4.69 & 5.97 \\
plume & 5.6e6 & -26 & 3 & 4.69 & 5.48 \\
plume & 1.3e7 & -26 & 3 & 4.69 & 5.73 \\
plume & 3.0e7 & -26 & 3 & 4.69 & 6.27 \\
plume & 7.0e7 & -26 & 3 & 4.69 & 6.93 \\
slab & 3.0e0 & -83 & 7 & 1.68 & -2.07 \\
slab & 7.0e0 & -66 & 8 & 1.66 & -1.74 \\
slab & 1.6e1 & -49 & 10 & 2.17 & -1.59 \\
slab & 3.7e1 & -32 & 16 & 0.61 & -1.46 \\
slab & 8.7e1 & -42 & 24 & 2.33 & -1.42 \\
slab & 2.0e2 & -52 & 63 & 0.61 & -1.55 \\
slab & 4.7e2 & -145 & 168 & 1.77 & -1.42 \\
slab & 1.1e3 & -120 & 142 & 1.91 & -0.86 \\
slab & 2.6e3 & -76 & 100 & 2.06 & -0.53 \\
slab & 6.0e3 & -42 & 68 & 2.21 & -0.39 \\
slab & 1.4e4 & -17 & 47 & 2.43 & -0.21 \\
slab & 3.3e4 & 1 & 32 & 2.66 & -0.02 \\
slab & 7.8e4 & 14 & 22 & 2.82 & 0.14 \\
slab & 1.8e5 & 23 & 15 & 2.95 & 0.33 \\
slab & 4.3e5 & 29 & 9 & 2.99 & 0.64 \\
slab & 1.0e6 & 33 & 8 & 2.98 & 0.66 \\
slab & 2.4e6 & 36 & 5 & 3.01 & 0.89 \\
slab & 5.6e6 & 37 & 4 & 3.10 & 1.10 \\
slab & 1.3e7 & 37 & 3 & 3.11 & 1.91 \\
slab & 3.0e7 & 39 & 2 & 3.23 & 3.00 \\
slab & 7.0e7 & 39 & 2 & 3.25 & 3.37 \\
\end{longtable}

\end{document}